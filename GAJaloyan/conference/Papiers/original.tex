  \documentclass{beamer}
  \usepackage[utf8]{inputenc}
  \usepackage{tabularx}
  \usetheme{Frankfurt}
  \usepackage{listings}



  \title{DBpedia - A Crystallization Point for the Web of Data}
  \author{C. Bizer, J. Lehmann, G. Kobilarov, S. Auer, C. Becker, R. Cyganiak, S. Hellmann}
\date{May 24, 2009}

\begin{document}

\begin{frame}
	
	\begin{center}
	\includegraphics[height=1.3cm]{DBlogo.png} \text{                     } \includegraphics[height=2cm]{Wikipedia-logo-fr-big.png}
	\end{center}
	\titlepage
\end{frame}

 	\section{Introduction}
\begin{frame}
	\frametitle{Introduction}
	\text{DBPedia is a project that :}
	\begin{itemize}
	\item extracts automatically information from Wikipedia pages and turns it into a knowledge base
	\item has more than 3.4 million entries (2010), each of them having an ID referring to a Wikipedia article
	\item works on 30 languages, covering individuals, places, music, films, games, companies, species (and diseases). 
	\item has a totaly french version developped by INRIA : fr.dbpedia.org
	\end{itemize}
\end{frame}


 \section{Extraction}
	\begin{frame}
	\includegraphics[height=9.7cm]{Wikipedia00.png}
\end{frame}
	
\begin{frame}
	\includegraphics[height=9.7cm]{Wikipedia01.png}
\end{frame}

\begin{frame}
	\frametitle{Extraction}
	\text{What can be extracted on a Wikipedia article :}
	\begin{itemize}
	\item title, infobox, categories, interwikis, redirections, external links
	\item but also geo-coordinates, images, disambiguation pages
	\end{itemize}
\end{frame}

\begin{frame}
	\frametitle{Extraction}
	\text{The data is stored in RDF triples which are in the form of}
	\begin{block}{}
	(sujet, prédicat, objet)
	\end{block}
\end{frame}

\begin{frame}[fragile]
	\frametitle{Extraction}
	\text{A language is made for requests on such databases, named SPARQL :}
	\begin{lstlisting}[language=SQL,frame=single]
	SELECT ?name ?mbox
	WHERE
	  { ?x foaf:name ?name.
	    ?x foaf:mbox ?mbox }
	\end{lstlisting}

	\begin{lstlisting}[frame=single]
	<001>  foaf:name   "Johnny Lee Outlaw" .
	<001>  foaf:mbox   <mailto:jlow@jlow.me> .
	<002>  foaf:name   "Peter Goodguy" .
	<002>  foaf:mbox   <mailto:peter@peter.me> .
	<003>  foaf:mbox   <mailto:carol@carol.me> .
	\end{lstlisting}
\end{frame}

\begin{frame}[fragile]
	\frametitle{Extraction}
	\text{A language is made for requests on such databases, named SPARQL :}
	\begin{lstlisting}[language=SQL,frame=single]
	SELECT ?name ?mbox
	WHERE
	  { ?x foaf:name ?name.
	    ?x foaf:mbox ?mbox }
	\end{lstlisting}

	\begin{lstlisting}[frame=single]
	<001>  foaf:name   "Johnny Lee Outlaw" .
	<001>  foaf:mbox   <mailto:jlow@jlow.me> .
	<002>  foaf:name   "Peter Goodguy" .
	<002>  foaf:mbox   <mailto:peter@peter.me> .
	-
	\end{lstlisting}
\end{frame}

\begin{frame}[fragile]
	\frametitle{Extraction}
	\text{Returns}
	\begin{lstlisting}[frame=single]
	"Johnny Lee Outlaw"	<mailto:jlow@jlow.me>
	"Peter Goodguy"	<mailto:peter@peter.me>
	\end{lstlisting}
\end{frame}

\begin{frame}[fragile]
	\includegraphics[height=8cm]{Structure00.png}
\end{frame}

\begin{frame}[fragile]
	\frametitle{11 extractors}

	\begin{itemize}
	\item title : \texttt{rdfs:label}
	\item first paragraph : \texttt{rdfs:comment}  \\ introduction : \texttt{dbpedia:abstract}
	\item interlanguages links for the abstracts of different languages
	\item images : \texttt{foaf:depiction}
	\item redirects : to find \textbf{synonyms}
	\item disambiguation for \textbf{homonyms} : \texttt{dbpedia:disambiguates}
	\item external links : \texttt{dbpedia:reference}
	\item pagelinks : \texttt{dbpedia:wikilink}
	\item homepages : \texttt{foaf:homepage}
	\item categories : \texttt{skos:concepts} and category relations are \texttt{skos:broader}
	\item geo-coordinates : GeoRSS Simple encoding of W3C which allow aerial filtering
	\end{itemize}
\end{frame}

\begin{frame}[fragile]
	\frametitle{Generic infobox}
	\text{Creates RDF triples of the following form :}
	\begin{block}{}
	(URI of the article, property, value)
	\end{block}

	\text{Examples from fr.dbpedia.org :}
	\begin{block}{}
	(dbpedia-fr:France, dbpedia-owl:capital, dbpedia-fr:Paris)
	(dbpedia-fr:France, dbpedia-owl:phonePrefix, 33 (xsd:integer))
	(dbpedia-fr:France, dbpedia-owl:topLevelDomain, .fr)
	\end{block}
\end{frame}


\begin{frame}[fragile]
	\frametitle{Generic infobox extraction algorithm}
	From : S. Auer, J. Lehmann, \textit{What have innsbruck and leipzig in common? extracting
semantics from wiki content}, in: Proceedings of the 4th European Semantic Web Conference, 2007
	\begin{itemize}
	\item Select pages wich contain \{\{template \textbar param1\textbar param2\}\}
	\item Extract all templates that have more than 3 parameters.
	\item Parse each template and generate RDF triples. 
	\item Post-process the values : detect units (no conversion performed), detect lists, ...
	\item Determine class membership for the page processed : name of the template, categories.
	\end{itemize}
	Result : 1 hour on 2.80 GHz (Xeon) with 1GB of memory for 1.5 million articles.
\end{frame}


\begin{frame}[fragile]
	\frametitle{Mapping-based Infobox}
	\text{Problem : different property names for the same attribute}

	\begin{columns}
	\begin{column}{0.5\textwidth}
	\begin{lstlisting}[frame=single]
{{Infobox1
 | nom               = 
 | nom local         = 
 | image             = 
 | taille image      =
 | district          =
 | ligne             = 
 | ancien nom        =
 | date              = 
 | architecte        =
 | profondeur        = 
}}
	\end{lstlisting}
	\end{column}


	\begin{column}{0.5\textwidth}
	\begin{lstlisting}[frame=single]
{{Infobox2
| nom = 
| nomination = 
| photo = 
| inauguration = 
| situation = 
| ligne = 
| type = 
| commune = 
| latitude =   
| longitude = 
}}
	\end{lstlisting}
	\end{column}
	\end{columns}

\end{frame}

\begin{frame}[fragile]
	\frametitle{Mapping-based Infobox}

	Solution : create "artificial" properties (ontology) not linked to Wikipedia, and manually specify which infobox property is linked to which ontological property.

	\begin{itemize}
	\item done with more than 350 most common infoboxes
	\item provides data for 843000 entities, against 1462000 for the generic approach
	\end{itemize}


\end{frame}

\begin{frame}[fragile]
	\frametitle{Mapping-based Infobox}
	
	Four different classifications :
	\begin{itemize}
	\item Wikipedia categories
	\item YAGO : leaf Wikipedia categories
	\item UMBEL : deriving from Cyc
	\item DBpedia ontology
	\end{itemize}
	
\end{frame}

\begin{frame}[fragile]
	\frametitle{Mapping-based Infobox}
	
	Some statistics on the graph structure : 
	\begin{tabular}{l!{\vrule}cccc} 
	  & nodes & indegree max & average & clustering \\\hline 
	Generic  & 102,9712 & 105,840 & 8.76 & .1336 \\\hline 
	Mapping based  & 62,7941 & 65,387 & 11.03 & .1037 \\\hline 
	\end{tabular}
	
\end{frame}

\section{Linking}
\begin{frame}[fragile]
	\frametitle{Linking with other data}
	
	The goal is not only to create links between URI, but also to create outgoing links. 
	\\Currently, there are more than 4.9 million outgoing links using the RDF triples.
	\\This allows : 
	\begin{itemize}
	\item Browsing the items
	\item Use a web crawler on them
	\item Link different pages about the same item
	\item Simplify categorisation on other pages (annotation)
	\end{itemize}
	
\end{frame}

\begin{frame}[fragile]
	\frametitle{Linking with other data}
	
	\begin{itemize}
	\item 2,400,000 links to Freebase
	\item 1,950,000 to Flickr wrappr
	\item 330,000 to Wordnet
	\item 85,000 to GeoNames
	\item 2,500 to Gutenberg Project
	\item 200 to CIA World Factbook and EuroStat
	\end{itemize}
	
\end{frame}

\begin{frame}[fragile]
	\frametitle{Examples}
		
	Linking to other pages :
	\begin{footnotesize}
	\begin{lstlisting}[frame=single]
	<http://dbpedia.org/resource/Spain> owl:sameAs
http://rdf.freebase.com/ns/guid.9202a8c04000634e30;
http://fu-berlin.de/factbook/resource/Spain;
http://[...].de/eurostat/resource/countries/Espa%C3%B1a;
http://sw.opencyc.org/2008/06/10/concept/Mx4rcN5Y29ycA.
	\end{lstlisting}
	\end{footnotesize}
	
	Categorisation on other pages :
	\begin{footnotesize}
	\begin{lstlisting}[frame=single]
<http://[...].org/conference/eswc/2008/paper/356>
swc:hasTopic 
       <http://dbpedia.org/resource/Data_integration>
	\end{lstlisting}
	\end{footnotesize}
\end{frame}

\begin{frame}[fragile]
	\frametitle{Other databases linking to DBPedia}

	\begin{itemize}
	\item BBC Music
	\item Bio2RDFDiseasome
	\item Flickr wrappr
	\item GeoNames
	\item GeoSpecies
	\item and many others ...
	\end{itemize}
	
\end{frame}

\begin{frame}[fragile]
	\frametitle{Some applications}

	\begin{itemize}
	\item A touristic gps, based on a Data browser.
	\item Improve geolocation on mobile apps, linking it to an URI.
	\item Query builder
	\item Relationship finder
	\item Annotation
	\item Other extraction projets : FreeBase (RIP), Yahoo, Wikidata, YAGO.
	\end{itemize}
	
\end{frame}

\begin{frame}[fragile]
	\frametitle{Query builder}
	\includegraphics[height=7.5cm]{Querybuilder.png}
\end{frame}

\begin{frame}[fragile]
	\frametitle{Relationship finder}
	\includegraphics[height=8cm]{Rel00.png}
\end{frame}

\begin{frame}[fragile]
	\frametitle{Relationship finder}
	\includegraphics[height=8.5cm]{Rel01.png}
\end{frame}


\begin{frame}[fragile]
	\frametitle{TODO List}

	\begin{itemize}
	\item Cross-language infobox extraction
	\item Improve automatically Wikipedia articles by collecting data from external links (CIA world factbook, ...)
	\item Checking for contradictions in Wikipedia articles, relative to other articles or to external data.
	\end{itemize}
	
\end{frame}

\begin{frame}
  \frametitle{Fin}
  Des questions ?
\end{frame}

\end{document}
