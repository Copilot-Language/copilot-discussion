  \documentclass{beamer}
  \usepackage[utf8]{inputenc}
  \usepackage{tabularx}
  \usetheme{Frankfurt}
  \usepackage{listings}
  \usepackage{mathpartir}
  \usepackage{multicol}
  \usepackage{graphicx}
  
  
  
  \title{DBpedia - A Crystallization Point for the Web of Data}
  \author{C. Bizer, J. Lehmann, G. Kobilarov, S. Auer, C. Becker, R. Cyganiak, S. Hellmann}
  \date{August 24, 2015}
  
  \begin{document}
  	\begin{frame}
  		
  		\begin{center}
  			\includegraphics[height=2cm]{images/NIA-logo.jpg} \text{                     } \includegraphics[height=2cm]{images/NASA.png}
  			\includegraphics[height=2cm]{images/ENS-logo.jpg}
  		\end{center}
  		\titlepage
  	\end{frame}
  	
  	\begin{frame}
  		
  		\begin{center}
  			\includegraphics[height=2cm]{images/NIA-logo.jpg}
  			\includegraphics[height=2cm]{images/ENS logo.jpg} \includegraphics[height=2cm]{images/NASA.png}
  		\end{center}
  		\titlepage
  	\end{frame}
  	
  	\section{Introduction}
  	\begin{frame}
  		\frametitle{Introduction}
  		\text{DBPedia is a project that :}
  		\begin{itemize}
  			\item extracts automatically information from Wikipedia pages and turns it into a knowledge base
  			\item has more than 3.4 million entries (2010), each of them having an ID referring to a Wikipedia article
  			\item works on 30 languages, covering individuals, places, music, films, games, companies, species (and diseases). 
  			\item has a totaly french version developped by INRIA : fr.dbpedia.org
  		\end{itemize}
  	\end{frame}
  	
  	
  	
  \end{document}
