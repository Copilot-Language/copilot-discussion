\section{Intermediate Language}~\label{sec:il}

\alwyn{Jonathan. This is where we need a lot of work. Since there isn't much in the way of documentation of the code so this section should spell out the syntax of the two formats as well as describe the translation algorithms. You might want to keep the description high level but include two small examples, but I will leave that up to you}

\alwyn{Jonathan. If I remember correctly, the IL format is translated into SMTLIB s-expressions for YICES2 or something very close to that. I'm hoping add a connection to another back end can handle math even a lot of trig stuff, but that's all it does). This would allow us to combine the three provers in really cool ways I think. It will take some experimenting on my part, but will make a cool paper on how we can orchestate the three provers to verify small but rather complex monitors. This desire to do this is one of the reasons I want to get the two translations well documented so me or another student could make add the third prover.} 


Each prover first translates the Copilot specification into an
intermediate representation best suited for model checking. Two
representations are available :

\begin{itemize}
\item
  The \textbf{IL} format : a list
  of quantifier-free equations over integer sequences, implicitly
  universally quantified by a free variable \emph{n}. Each sequence
  roughly corresponds to a stream. This format is the one used in G.
  Hagen's thesis~\cite{HagenPhD}. The \emph{light prover} works with this
  format.
\item
  The \textbf{TransSys} format : a modular representation of a \emph{state transition system} \cite{} The \emph{Kind2 prover} uses this
  format, which is similar to Kind2's native format.
\end{itemize}

\textbf{Cnub} is a simplified representation of a Copilot program where
only the informations useful for SMT-based model checking are kept. For
now, it is not used by the two standard provers but it could be used in
the future as an intermediate step in the translation of a copilot specification to the \textbf{IL} or \textbf{TransSys} format.



%For each of these formats, there is a folder in
%\texttt{src/Copilot/Kind} which contains at least three files

%\begin{itemize}
%\item \texttt{Spec.hs} where the format is defined
%\item \texttt{PrettyPrint.hs} for pretty
%printing (useful for debugging) 
%\item \texttt{Translate.hs} where the
%translation process from \texttt{Core.Spec} is defined
%\end{itemize}

%These three formats share a simplified set of types and operators,
%defined respectively in \texttt{Misc.Type} and \texttt{Misc.Operator}.


\subsection{The \textbf{Cnub} format}

\subsection{The \textbf{IL} format}

\subsection{The \textbf{TransSys} format}




