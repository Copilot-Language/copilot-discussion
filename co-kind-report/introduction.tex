\section{Introduction}~\label{sec:intro}

%In this section we shall put an introduction 

%What are we trying to accomplish

%Why

%How  (by connecting copilot with model checking) 

Critical cyber-physical systems that require correct software behavior
such as aircraft, spacecraft, medical devices, and nuclear reactors
undergo extensive testing and may require certification by regulatory
bodies before being deployed. Despite being constructed with thorough
engineering processes there continue to be well documented incidents
that demonstrate how hard it is to build robust software systems.


Runtime verification (RV)~\cite{KimVBKLS99, monitors} is a family of
approaches that employ {\em monitors} to observe the behavior of an
executing system and to detect if it is consistent with a formal
specification. Copilot is a RV framework that has been developed as
part of a NASA project investigating applying RV to safety-critical
hard real-time systems as a ``last line of defense.''  Given that RV
acts in the role last defense  the natural question arises
``who watches the watchmen?''  Nobody. For this reason, monitors for
safety-critical systems cannot fail.  Consequently, the Copilot team
has been investigating the question of how to produce monitors in
which we can be assured of a high-degree of confidence that they will
not fail when needed most. Earlier work focused on the light-weight
verification of the monitor compilation
process~\cite{pike-icfp-12}. The current report focuses on applying
SMT-based model checking to ensure that the monitor itself is correct. 


  