\section{Bounded Model Checking,  k-induction, and IC3}~\label{sec:mck}

\alwyn{Jonathan: please proof read this section especially the IC3 part and add in any text you think may make it easier to read or further the explanation.}

\alwyn{Jonathan: When I wrote this I tried to keep the notation as consistent as possible when expaining different ideas, but I hope I didn't over simplify things. When you read it make sure I didn't change some notation in a way that makes either makes it harder to read or just makes it false}

Highly automated proof techniques are a necessary step for the
widespread adoption of formal methods in the software industry.
Probably the most widely used automated technique is {\em model
  checking}, where a design is modeled as finite state machine, and
specification of some property is formalized in a suitable logic.  The
model checker traverses all of the 
reachable states of the design in order to verify that the property
holds.  If the property fails, then a sequence of states leading to
the failure is produced as a {\em counterexample.} Our work applies
the technique of {\em bounded model checking with satisfiability
  solving}~\cite{ClarkeBounded01,ClarkeBounded03} to the problem of
verifying Copilot monitors. Bounded model checking searchers for
counterexamples of length bounded by some user provided integer $k.$


\paragraph{Background and Definitions} Recall that a state transition system is a triple $(S,I,T),$
where $S$ is a set of states, $I \subseteq S$ is the set of initial
states and $T \subseteq S \times S $ is a transition relation over $S.$ A {\em kripke
  structure} is a state transition system $(S,I,T,L)$  with a labeling function
$L: S \rightarrow  \mathcal{P}(A)$.  A state $s 
\in S$ is said to be {\em k-reachable} if there exists a path from an
initial state $i\in I$ to $s$ of length at most $k.$ A state is said
to be {\em reachable} if it is $k$-reachable for some $k\in \mathbb{N}.$
Likewise, a set of states $Q \subset S$ is said to be reachable if
there exists a reachable state in $S.$  Given a transition system, a
set of states containing all the reachable states of the system is
said to be {\em invariant.} In the classic procedure for model checking of Linear
Temporal Logic (LTL) specifications, one computes the product of
the kripke structure modeling the system and an transition system that
represents the negation of the specification~\cite{ClarkeMC}. This product captures
the counterexamples of executions violating the property. 

\paragraph{Bounded Model Checking} Bounded model checking was developed to exploit the success of modern
satisfiability (SAT) solvers and more recently satisfiability modulo
theories (SMT) solvers. Bounded model checking only considers finite
prefixes of length $k$  of a path that may be a witness acting as a
counterexample. In practice, the value of $k$ is systematically
increased to look for counterexamples in longer traces.  

  
Suppose we wish to apply bounded model checking to show that property
$P$ holds in  the system modeled by the transition
system $(S,I,T,L),$ where $P: S \rightarrow \mathbb{B}$ represents
the predicate $P$ defined over the states of $S$ such that $P(s)$
holds in states $s.$
Formulating this as the following entailment:
$$ I(s_0) \tland T(s_0,s_1) \tland \cdots \tland T(s_{k-1},s_k)
\models P(s_k)$$
for increasing values of $k.$ In the case of $\tlnot P$ is reachable,
the SMT solver will provide an assignment for 
$$I(s_0) \tland T(s_0,s_1) \tland T(s_{k-1},s_k) \tland \tlnot
P(s_k)$$
that can be used a counterexample. 

\paragraph{$k$-induction}  In proving $P$ manually, a common technique
is mathematical induction.  Applying classical induction to prove a
property $P$ holds on  for a model one would first show the base case
by proving that the property is satisfied by the initial state:
$$I(s_0) \models P(s_0).$$
If the base case holds, then show the inductive step that $P$ holds
for some state $n$ and prove it holds for the next state $n+1:$
$$ T(s_{n},s_{n+1}) \tland P(s_n) \models P(s_{n+1}).$$
Unfortunately, it is often the case that the induction step cannot be
done automatically.  Sheeran~\etal~\cite{Sheeran00, EenS03} introduced
$k-$induction that has the advantage that it is well suited for SAT
based bounded model checking.  One begins as with the standard
induction, but if one cannot show the property to be true, the
property is strengthened by extending the formula and
progressively increasing the length of the reachable states
considered.  To show $P$ holds in the transition system one must first
show the base case holds, that is $P$ holds in all states reachable
from an initial state in $k$ steps, and then show the induction step,
that if $P$ holds in states $s_0,\ldots,s_{k-1}$ then it holds in
state $s_k.$ The $k$-induction principle is formally expressed in the
following two entailments:
\begin{eqnarray*}
I(s_0) \tland T(s_0,s_1) \tland \cdots \tland T(s_{k-1},s_k) &\models&
P(s_k) \\
P(s_0) \tland \cdots \tland P(s_{k-1}) \tland T(s_0,s_1) \tland \cdots \tland T(s_{k-1},s_k) &\models&
P(s_k) 
\end{eqnarray*} 

Property $P$ said to be a $k$-inductive property with respect to
$(S,I,T)$ if there exists some $k \in \mathbb{N}^{0<}$ such that $P$
satisfies the $k$-induction principle. As $k$ increases, weaker
invariants may be proved. The trick is to find an invariant that is
tractable by the SMT solver yet weak enough to satisfy the desired
property.  

\paragraph{Path Compression}  A major enhancement to the basic
$k$-induction 
algorithm eliminates redundant searches via a procedure
known as path compression~\cite{dMRS03} that strengthens the left-hand
side of the entailments to eliminate paths that contain repeated configurations  (due to
a cycle) or configurations that are equivalent.  

If $P$ is not invariant, a
counterexample exists such that:
\begin{itemize}
\item It is cycle-free
\item Only its first state belongs to $I$ 
\end{itemize}
formalizing these constraints:
$$C_k(s_0,\ldots,s_k) = \bigwedge_{i \neq j} s_i \neq s_j  \tland
\bigwedge_{i > 0} \tlnot I(s_i),$$
where equality is defined  pointwise over the state vectors.
We can now strengthen the left hand side of the induction step
entailment:
$$\bigwedge^{k-1}_{i=0} P(s_i) \tland \bigwedge ^{k-1}_{i=0}
T(s_i,s_{i+1}) \tland C_k(s_0, \ldots, s_k) \models P(s_k). $$
Now, we can make the algorithm complete for bounded state spaces by
checking the entailment 
$$I(s_0) \tland T(s_0,s_1) \tland \cdots \tland T(s_{k-1}, s_{k})
 \models \tlnot C_k(s_0,\ldots, s_{k-1})$$
after the $k$-th bounded model checking iteration.  If this entailment
holds, and $P$ is $k$-invariant, then $P$ is invariant. 

\paragraph{Structural Abstraction}
A limiting factor when it comes to applying $k-$induction in practice
is the computing resources required by the SMT solver. Abstraction is
often applied to the model the system being analyzed in order to
reduce the problem space to a more manageable problem.  {\em
  Structural abstraction}~\cite{ bh07structural} is a technique that
constructs a abstract model that simulates the concrete one and
accepts at least the behaviors accepted by the concrete model. The
abstraction can reduce the size of the problem
space and render $k-$induction tractable. This
approach is  
related to counterexample guided refinement
(CEGAR)~\cite{Clarke2003CAR}, where if a counterexample is produced in
the analysis of the abstract model that is inconsistent with the
concrete model, the abstraction is refined to eliminate the
counterexample.The idea is to replace $T$
by a weaker approximation $T'$ by removing some clauses. Then, 
\begin{itemize}
\item If $P$ is invariant for $T',$ then $P$ is also invariant for $T.$
\item Otherwise, we use the counterexample produced by the SMT solver
  to refine $T'$ by restoring some well-chosen clauses.
\end{itemize}  
Suppose a counterexample $(v_0,\ldots, v_k)$ is found for $T',$ we can
check  whether or not
$$\bigwedge_{i=0}^k  \bigwedge_{j=1}^n v_{ij} = s_{ij} \tland I(x)
\tland \bigwedge_{i=0}^{k-1} T(s_i,s_{i+1})$$ 
is satisfiable.  If it is, then the counterexample is valid for $T.$
Otherwise, the counterexample is spurious.   It is possible to query
the SMT solver to obtain an unsatisfiable core of state variables that
candidates for refinement.


\paragraph{IC3 Algorithm} 

Bounded model checking approaches such as $k$-induction suffer from
the drawback of searching for a single strengthening of the invariant
to be proved. This monolithic approach often  overwhelms the SAT
solver due to its many iterations.  

Recall that a property $P$ is inductive if 
$$ P(s_{k-1}) \tland T(s_{k-1},s_k) \models P(s_k). $$ Otherwise there
exists states $s_{k-1}$ and $s_k$ such that  
$$P(s_{k-1}) \tland T(s_{k-1},s_k) \tland \tlnot P(s_k),$$
where $s_{k-1}$ is said to be the {\em counterexample to
  inductiveness} (CTI). The property $P$ is said to be inductive
relative to $A$ if and only if
$$A(s_{k-1}) \tland  P(s_{k-1}) \tland T(s_{k-1},s_k) \models P(s_k). $$
Note that if $P$ is inductive relative to $Q$ and $Q$ is inductive
relative to $P,$ then $P \tland Q$ is inductive.  The idea of using
relative inductiveness to prove inductiveness lead to the development
of incremental approaches such as that used in the FSIS model
checker~\cite{Bradley06, Bradley2011}.  The basic incremental
algorithm used in FSIS is as follows:
\begin{itemize}
\item Try to prove $P$ is inductive.
\item If fails,  if $s$ is CTI, then find an inductive invariant
  $\phi_0$ such that $\phi_0(s)$ does not hold. 
\item Continue finding such lemmas $\phi_0, \phi_1, \ldots$ until
  there are no more CTIs. 
\item As $P$ is inductive to $\bigwedge_{i} \phi_i$ it s proven
  invariant if it holds for the initial states.
\end{itemize}  
Note that each $\phi_i$ has only to be proven inductive relative to
the previously discovered CTIs $\phi_0, \phi_1, \ldots, \phi_{i-1}$
or even inductive to $\{ \phi_j \}_{j<i} \tland P.$ 


As it is difficult to find inductive lemmas, we can limit our search
to lemmas that are inductive relative to a formula $R_k$ that
over-approximates the set of $k$-reachable states. We then try to
propagate these lemmas as $k$ increases, stopping only when reaching a
fixed point. The IC3 algorithm~\cite{Somenzi-FMCAD11,
  BradleyCAV12,bradley2012understanding} searches for an
over approximation $R$ of the reachable sets such that:
\begin{eqnarray*} 
I(s_{k-1})  & \models & R(s_{k-1}) \\
R(s_{k-1})  \tland  T(s_{k-1},s_k) &\models & R(s_k)  \\
R(s_{k-1})  & \models & P(s_{k-1})
\end{eqnarray*}
where $R$  is constructed as a fixed point of an increasing sequence
of $(R_k),$ where each frame $(R_k)$ can be viewed as a conjunction of
lemmas. The sequence exhibits the following properties: 
\begin{eqnarray*}
R_0 & = & \{ I \} \\
R_{i+1} & \subset & R_i \\
R_i(s_{k-1}) \tland  T(s_{k-1},s_k) &\models & R_{i+1}(s_k)   \\
P &\in& R_i \ \ \mbox{for} \ i > 0. 
\end{eqnarray*}
 If the sequence satisfies these four constraints and admits a fixed
 point, then $P$ is invariant.   In constructing the sequence, $R_0$
 is initialized to $\{ I \}$ and  $R_1 = P.$  We continue in similar
 fashion, at  each step, $k,$ is initialized to $P$ then propagate as
 many lemmas from the previous frame as possible while ensuring that
 the an error is not reached within one transition step. 

Just as $R$ represents the reachable good states, the IC3 algorithm
maintains a set of bad states $B$ such that each state in $B$ leads to
an error in one step.  In order to show $B$ is unreachable in $R_k$
the predicate $\tlnot B$ is conjoined to $R_k$ as a lemma.  Although
$B$ may not be inductive, it is sufficient to show that it is
inductive relative to the previous frame and consequently $B$ is not
$k-$reachable.  This can be accomplished by checking the following two
entailments:
\begin{eqnarray*}
I(s_{i-1}) & \models & \tlnot B(s_{i-1}) \\ 
R_{k-1} (s_{i-1}) \tland \tlnot  B(s_{i-1}) \tland T(s_{i-1}, s_i) 
& \models & \tlnot B(s_i).
\end{eqnarray*}
If the first entailment fails, then $P$ is not an invariant. If the
second fails,  then $B$ is reachable from some state of the previous
frame. We generalize this state to a set of states $B',$ consisting of
states belonging to the previous frame that lead to $B$ in a single
transition.  Next, the algorithm recursively refines $R_{k-1}$ to
remove the states in $B'.$ This backward reachability is performed
until one of the following conditions are met:
\begin{itemize}
\item The algorithm reaches $R_0 = I$ whence $P$ is not invariant. 
\item All the bad states that are recursively generated have been
  shown to be $k$-unreachable. 
\end{itemize} 

Once the strengthening process completes, meaning that no error state
is reachable from the last frame, the IC3 algorithm propagates the new
lemmas we discovered from each frame to its successor. This is
accomplished by verifying the entailment

$$ R_j(s_{i-1}) \tland  T(s_{i-1}, s_i)  \models C(s_i)$$
for all $C \in R_j - R_{j+1},$ where $j$ increases from $0$ to $k.$
If all the lemmas of $R_{k-1}$ are successfully propagated to $R_k$,
the sequence $(R_j)$ reaches a fixed point and $P$ is proven to be
invariant.  Otherwise,  a new frame $R_{k+1}$ is added that is  initialized to $P$
and the algorithm continues.  Numerous improvements to the basic
algorithm such as {\em lemma tightening} can be found in the
literature that can produce an invariant that is more tractable to
modern SMT solvers. 



