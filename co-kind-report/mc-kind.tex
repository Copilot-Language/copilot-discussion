\section{Bounded Model Checking,  k-induction, and IC3}~\label{sec:mck}

Highly automated proof techniques are a necessary step for the
widespread adoption of formal methods in the software industry.
Probably the most widely used automated technique is {\em model
  checking}, where a design is modeled as finite state machine, and
specification of some property is formalized in a suitable logic.  The
model checker traverses all of the 
reachable states of the design in order to verify that the property
holds.  If the property fails, then a sequence of states leading to
the failure is produced as a {\em counterexample.} Our work applies
the technique of {\em bounded model checking with satisfiability
  solving}~\cite{ClarkeBounded01,ClarkeBounded03} to the problem of
verifying Copilot monitors. Bounded model checking searchers for
counterexamples of length bounded by some user provided integer $k.$


\paragraph{Background and Definitions} Recall that a state transition system is a triple $(S,I,T),$
where $S$ is a set of states, $I \subseteq S$ is the set of initial
states and $T \subseteq S \times S $ is a transition relation over $S.$ A {\em kripke
  structure} is a state transition system $(S,I,T,L)$  with a labeling function
$L: S \rightarrow  \mathcal{P}(A)$.  A state $s 
\in S$ is said to be {\em k-reachable} if there exists a path from an
initial state $i\in I$ to $s$ of length at most $k.$ A state is said
to be {\em reachable} if it is $k$-reachable for some $k\in \mathbb{N}.$
Likewise, a set of states $Q \subset S$ is said to be reachable if
there exists a reachable state in $S.$  Given a transition system, a
set of states containing all the reachable states of the system is
said to be {\em invariant.} In the classic procedure for model checking of Linear
Temporal Logic (LTL) specifications, one computes the product of
the kripke structure modeling the system and an transition system that
represents the negation of the specification~\cite{ClarkeMC}. This product captures
the counterexamples of executions violating the property. 

\paragraph{Bounded Model Checking} Bounded model checking was developed to exploit the success of modern
satisfiability (SAT) solvers and more recently satisfiability modulo
theories (SMT) solvers. Bounded model checking only considers finite
prefixes of length $k$  of a path that may be a witness acting as a
counterexample. In practice, the value of $k$ is systematically
increased to look for counterexamples in longer traces.  

  
Suppose we wish to apply bounded model checking to show that property
$P$ holds in  the system modeled by the transition
system $(S,I,T,L),$ where $P: S \rightarrow \mathbb{B}$ represents
the predicate $P$ defined over the states of $S$ such that $P(s)$
holds in states $s.$
Formulating this as the following entailment:
$$ I(s_0) \tland T(s_0,s_1) \tland \cdots \tland T(s_{k-1},s_k)
\models P(s_k)$$
for increasing values of $k.$ In the case of $\tlnot P$ is reachable,
the SMT solver will provide an assignment for 
$$I(s_0) \tland T(s_0,s_1) \tland T(s_{k-1},s_k) \tland \tlnot
P(s_k)$$
that can be used a counterexample. 

\paragraph{$k$-induction}  In proving $P$ manually, a common technique
is mathematical induction.  Sheeran~\etal~\cite{Sheeran00, EenS03} introduced 
$k-$induction that generalizes the notion of induction to transition
systems. $k$-induction has the advantage that it is well suited for SAT based bounded model checking.  To
show $P$ holds in the transition system one must first show the base case holds, that
is $P$ holds in all states
reachable from an initial state in $k$ steps, and then show the
induction step, that if $P$ holds in states $s_0,\ldots,s_{k-1}$ then
it holds in state $s_k.$ The $k$-induction principle  is formally expressed in the following
two entailments:
\begin{eqnarray*}
I(s_0) \tland T(s_0,s_1) \tland \cdots \tland T(s_{k-1},s_k) &\models&
P(s_k) \\
P(s_0) \tland \cdots \tland P(s_{k-1}) \tland T(s_0,s_1) \tland \cdots \tland T(s_{k-1},s_k) &\models&
P(s_k) 
\end{eqnarray*} 

Property $P$ said to be a $k$-inductive property with respect to
$(S,I,T)$ if there exists some $k \in \mathbb{N}^{0<}$ such that $P$
satisfies the $k$-induction principle. As $k$ increases, weaker
invariants may be proved. The trick is to find an invariant that is
tractable by the SMT solver yet weak enough to satisfy the desired
property.  

\paragraph{Path Compression}  A major enhancement to the basic
algorithm eliminates redundant searches via a procedure
known as path compression~\cite{dMRS03} that strengthens the left-hand
side of the entailments to eliminate paths that contain repeated configurations  (due to
a cycle) or configurations that are equivalent.  

If $P$ is not invariant, a
counterexample exists such that:
\begin{itemize}
\item It is cycle-free
\item Only its first state belongs to $I$ 
\end{itemize}
formalizing these constraints:
$$C_k(s_0,\ldots,s_k) = \bigwedge_{i \neq j} s_i \neq s_j  \tland
\bigwedge_{i > 0} \tlnot I(s_i),$$
where equality is defined  pointwise over the state vectors.
We can now strengthen the left hand side of the induction step
entailment:
$$\bigwedge^{k-1}_{i=0} P(s_i) \tland \bigwedge ^{k-1}_{i=0}
T(s_i,s_{i+1}) \tland C_k(s_0, \ldots, s_k) \models P(s_k). $$
Now, we can make the algorithm complete for bounded state spaces by
checking the entailment 
$$I(s_0) \tland T(s_0,s_1) \tland \cdots \tland T(s_{k-1}, s_{k})
 \models \tlnot C_k(s_0,\ldots, s_{k-1})$$
after the $k$-th bounded model checking iteration.  If this entailment
holds, and $P$ is $k$-invariant, then $P$ is invariant. 

\paragraph{Structural Abstraction}
A limiting factor when it comes to applying $k-$induction in practice
is the computing resources required by the SMT solver.  {\em
  Structural abstraction}~\cite{ bh07structural} is a technique
related to counterexample guided refinement (CEGAR)~\cite{Clarke2003CAR} that can reduce the size of the problem
space and render $k-$induction tractable. The idea is to replace $T$
by a weaker approximation $T'$ by removing some clauses. Then, 
\begin{itemize}
\item If $P$ is invariant for $T',$ then $P$ is also invariant for $T.$
\item Otherwise, we use the counterexample produced by the SMT solver
  to refine $T'$ by restoring some well-chosen clauses.
\end{itemize}  
Suppose a counterexample $(v_0,\ldots, v_k)$ is found for $T',$ we can
check  whether or not
$$\bigwedge_{i=0}^k  \bigwedge_{j=1}^n v_{ij} = s_{ij} \tland I(x)
\tland \bigwedge_{i=0}^{k-1} T(s_i,s_{i+1})$$ 
is satisfiable.  If it is, then the counterexample is valid for $T.$
Otherwise, the counterexample is spurious.   It is possible to query
the SMT solver to obtain an unsatisfiable core of state variables that
candidates for refinement.


\paragraph{IC3 Algorithm} 

\alwyn{TBD} 