\section{Implementation Overview}~\label{sec:overview}


\jonathan{I find the typography and layout of this part not very good. The indentation should be handled better and it would be great to display the code into boxes. Syntax highlighting would be a bonus. Could you do it Alwyn ? }

\jonathan{Warning : the code environment removes the curly braces}


%\jonathan{Is this really an implementation overview ? Shouldn't we add a section "Presentation of Copilot Kind"}
\jonathan{Maybe we should move the definition of a safety property in the previous part}

The library \emph{copilot-kind} is aimed at checking \textbf{safety properties} on
Copilot programs. Intuitively, a safety property is a property which
express the idea that \emph{nothing bad can happen}. In particular, any
invalid safety property can be disproved by a finite execution trace of
the program called a \textbf{counterexample}. Safety properties are
often opposed to \textbf{liveness} properties, which express the idea
that \emph{something good will eventually happen}. The latters are out
of the scope of \emph{copilot-kind}.

Safety properties are simply expressed with standard boolean streams. In
addition to triggers and observers declarations, it is possible to bind
a boolean stream to a property name with the \texttt{prop} construct in
the specification.

For instance, here is a straightforward specification declaring one
property :
\begin{code}
spec :: Spec
spec = do
  prop "gt0" (x > 0)
  where
    x = [1] ++ (1 + x)
\end{code}


Let's say we want to check that \texttt{gt0} holds. For this, we use
the function

\begin{code}
    prove :: Prover -> ProofScheme -> Spec -> IO ()
\end{code}

exported by \texttt{Copilot.Kind}. This function takes three
arguments :

\begin{itemize}
\itemsep1pt\parskip0pt\parsep0pt
\item
  The prover we want to use. For now, two provers are available,
  exported by the \texttt{Copilot.Kind.Light} and
  \texttt{Copilot.Kind.Kind2} module.
\item
  A \emph{proof scheme}, which is a sequence of instructions like
  \emph{check}, \emph{assume}, \emph{assert}\ldots{}
\item
  The Copilot specification
\end{itemize}

Here, we can just write
\begin{code}
prove (lightProver def) (check "gt0") spec
\end{code}

where \texttt{lightProver def} stands for the \emph{light prover} with
default configuration.

\subsubsection{The Prover interface}\label{the-prover-interface}

The \texttt{Copilot.Kind.Prover} defines a general interface for
provers. Therefore, it is really easy to add a new prover by creating a
new object of type \texttt{Prover}. The latter is defined like this :

\begin{code}
data Cex = Cex

type Infos = [String]

data Output = Output Status Infos

data Status
  = Valid
  | Invalid (Maybe Cex)
  | Unknown
  | Error
  
data Feature = GiveCex | HandleAssumptions
  
data Prover = forall r . Prover 
  { proverName     :: String
  , hasFeature     :: Feature -> Bool
  , startProver    :: Core.Spec -> IO r
  , askProver      :: r -> [PropId] -> PropId -> IO Output 
  , closeProver    :: r -> IO () 
  }

\end{code}

Each prover mostly has to provide a \texttt{askProver} function which
takes as an argument * The prover descriptor * A list of assumptions * A
conclusion

and checks if the assumptions logically entail the conclusion.

Two provers are provided by default : \texttt{Light} and \texttt{Kind2}.

\paragraph{The light prover}\label{the-light-prover}

The \emph{light prover} is a really simple prover which uses the Yices
SMT solver with the \texttt{QF\_UFLIA} theory and is limited to prove
\emph{k-inductive} properties.

\jonathan{Removed a redundant explanation of what k-induction is}
For instance, in this example

\begin{code}
spec :: Spec
spec = do
  prop "gt0"  (x > 0)
  prop "neq0" (x /= 0)
  where
    x = [1] ++ (1 + x)
\end{code}

the property \texttt{"gt0"} is inductive (1-inductive) but the property
\texttt{"neq0"} is not.

The \emph{light prover} is defined in \texttt{Copilot.Kind.Light}. This
module exports the 
$$\texttt{lightProver :: Options -\textgreater{} Prover}$$ function which
builds a prover from a record of type \texttt{Options} :

\begin{code}
data Options = Options 
  { kTimeout  :: Integer
  , onlyBmc   :: Bool
  , debugMode :: Bool } 

\end{code}
Here,

\begin{itemize}
\itemsep1pt\parskip0pt\parsep0pt
\item
  \texttt{kTimeout} is the maximum number of steps of the k-induction
  algorithm the prover executes before giving up.
\item
  If \texttt{onlyBmc} is set to \texttt{True}, the prover will only
  search for counterexamples and won't try to prove the properties
  discharged to it.
\item
  If \texttt{debugMode} is set to \texttt{True}, the SMTLib queries
  produced by the prover are displayed in the standard output.
\end{itemize}

\texttt{Options} is an instance of the \texttt{Data.Default} typeclass :

\begin{code}
instance Default Options where
  def = Options 
    { kTimeout  = 100
    , debugMode = False 
    , onlyBmc   = False }

\end{code}

Therefore, \texttt{def} stands for the default configuration.

\paragraph{The Kind2 prover}\label{the-kind2-prover}

The \emph{Kind2} prover uses the model checker with the same name, from
Iowa university. It translates the Copilot specification into a
\emph{modular transition system} (the Kind2 native format) and then
calls the \texttt{kind2} executable.

It is provided by the \texttt{Copilot.Kind.Kind2} module, which exports
a $$\texttt{kind2Prover :: Options -\textgreater{} Prover}$$ where the
\texttt{Options} type is defined as

\begin{code}
data Options = Options { bmcMax :: Int }
\end{code}

and where \texttt{bmcMax} corresponds to the \texttt{-\/-bmc\_max}
option of \emph{kind2} and is equivalent to the \texttt{kTimeout} option
of the light prover. Its default value is 0, which stands for infinity.

\paragraph{Combining provers}\label{combining-provers}

The
$$\texttt{combine :: Prover -\textgreater{} Prover -\textgreater{} Prover}$$
function lets you merge two provers A and B into a prover C which
launches both A and B and returns the \emph{most precise} output. It
would be interesting to implement other merging behaviours in the
future. For instance, a \emph{lazy} one such that C launches B only if A
has returns \emph{unknown} or \emph{error}.

As an example, the following prover is used in \texttt{Driver.hs} :

\begin{code}
prover =
  lightProver def {onlyBmc = True, kTimeout = 5} 
  `combine` kind2Prover def
\end{code}

We will discuss the internals and the experimental results of these
provers later.

\subsubsection{Proof schemes}\label{proof-schemes}

Let's consider again this example :

\begin{code}
spec :: Spec
spec = do
  prop "gt0"  (x > 0)
  prop "neq0" (x /= 0)
  where
    x = [1] ++ (1 + x)

\end{code}
and let's say we want to prove \texttt{"neq0"}. Currently, the two
available solvers fail at showing this non-inductive property (we will
discuss this limitation later). Therefore, we can prove the more general
inductive lemma \texttt{"gt0"} and deduce our main goal from this. For
this, we use the proof scheme

\begin{code}
assert "gt0" >> check "neq0"
\end{code}


instead of just \texttt{check "neq0"}. A proof scheme is chain of
primitives schemes glued by the $\texttt{\textgreater{}\textgreater{}}$
operator. For now, the available primitives are :

\begin{itemize}
\itemsep1pt\parskip0pt\parsep0pt
\item
  \texttt{check "prop"} checks whether or not a given property is true
  in the current context.
\item
  \texttt{assume "prop"} adds an assumption in the current context.
\item
  \texttt{assert "prop"} is a shortcut for
  \texttt{check "prop" \textgreater{}\textgreater{} assume "prop"}.
\item
  \texttt{assuming :: {[}PropId{]} -\textgreater{} ProofScheme -\textgreater{} ProofScheme}
  is such that \texttt{assuming props scheme} assumes the list of
  properties \emph{props}, executes the proof scheme \emph{scheme} in
  this context, and forgets the assumptions.
\item
  \texttt{msg "..."} displays a string in the standard output
\end{itemize}

We will discuss the limitations of this tool and a way to use it in
practice later.

\subsubsection{Some examples}\label{some-examples}

\jonathan{I will add a deeper analysis of these examples.}

Some examples are included in the \emph{examples} folder of the implementation. The \texttt{Driver.hs}
contains the \texttt{main} function to run any example. Each other
example file exports a specification \texttt{spec} and a proof scheme
\texttt{scheme}. You can change the example being run just by changing
one \emph{import} directive in \texttt{Driver.hs}.

These examples include :

\begin{itemize}
\itemsep1pt\parskip0pt\parsep0pt
\item
  \texttt{Incr.hs} : a straightforward example in the style of the
  previous one.
\item
  \texttt{Grey.hs} : an example where two different implementations of a
  periodical counter are shown to be equivalent.
\item
  \texttt{BoyerMoore.hs} : a certified version of the majority vote
  algorithm introduced in the Copilot tutorial.
\item
  \texttt{SerialBoyerMoore.hs} : a \emph{serial} version of the first
  step of the \emph{Boyer Moore algorithm}, where a new element is added
  to the list and the majority candidate is updated at each clock tick.
  See the section \emph{Limitations related to the SMT solvers} for an
  analysis of this example.
\end{itemize}

