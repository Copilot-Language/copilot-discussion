\documentclass{beamer}
\usetheme{Boadilla}
\useinnertheme{rectangles}
\usepackage{listings}
\usepackage{xcolor}

\lstset {
  basewidth = {0.5em,0.45em},
  basicstyle = \footnotesize\ttfamily,
  flexiblecolumns = false,
  keywordstyle = \color{beamer@blendedblue},
  language = Haskell,
  literate = {++}{{\ttfamily{\color{beamer@blendedblue}++}}}2,
  morekeywords = {extern, Spec, Stream, Word64, trigger},
  stringstyle = \color{beamer@blendedblue!60}
}


\title{Monitoring Real-Time Properties in Copilot}
\author{Lauren Pick}
\date{\today}

\begin{document}
\maketitle

\begin{frame}[fragile]
\frametitle{Copilot Streams and Iterations}
\begin{itemize}
\item Copilot manipulates and receives data in the form of streams
  \begin{itemize}
  \item The $n$th value of a stream $s$ is denoted as $s(n)$
  \end{itemize}
\item Copilot programs have a cyclic behavior, where each iteration consists of
  \begin{enumerate}
  \item sampling externs
  \item updating internal variables
  \item firing external triggers
  \end{enumerate}
  \begin{itemize}
  \item So for a stream $s$, the $n$th iteration provides the value
        $s(n)$, where the first iteration is the 0th iteration.
  \item For a Copilot program that has been compiled into C, an iteration
        occurs each time the \verb,step(), function is called.
  \end{itemize}
\end{itemize}
\end{frame}

\begin{frame}[fragile]
\frametitle{Monitoring Properties Involving Real-Time with Copilot}
\begin{itemize}
\item Given knowledge of a constant sampling rate for externs of a monitor,
real-time properties can be monitored, since each iteration would
correspond to a known, constant amount of time elapsing.
  \begin{itemize}
  \item e.g. if the sampling period is known to be 1 second:
\begin{lstlisting}
bools = extern "bools" Nothing
-- 1 second delay:
bools' :: Stream Bool
bools' = [False] ++ bools
\end{lstlisting}
  \end{itemize}
\item Monitoring properties involves checking if the current and past values
in streams match the values that would be expected if the property held
  \begin{itemize}
  \item As a result, it may be necessary to delay streams or express
  the property in terms of past values if the original property involves future
  values
  \end{itemize}
\end{itemize}
\end{frame}

\begin{frame}[fragile]
\frametitle{A Property Involving Real-Time}
A property that is to be monitored assuming that the global sampling
period is 0.1 second:

\begin{center}
A sampled temperature value should not exceed the threshold temperature $thresh$ for 0.3
second or longer.
\end{center}

\begin{itemize}
\item If $over$ is the stream of booleans indicating whether or not the temperature
$temp$ exceeds $thresh$ ($temp > thresh$), then this means that for all
$n$, it is not the case that \vspace{0.5em}

\begin{tabular}{lcl c lcl c}
$over (n + 3)$ & = & \verb,true, & $\wedge$ & $over(n + 2)$ & = & \verb,true, & $\wedge$\\
$over (n + 1)$ & = & \verb,true, & $\wedge$ & $over(n)$ & = & \verb,true,
\end{tabular}

\item or, if expressed in terms of past values, then for all $n > 2$,
it is not the case that \vspace{0.5em}

\begin{tabular}{lcl c lcl c}
$over (n)$ & = & \verb,true, & $\wedge$ & $over(n - 1)$ & = & \verb,true, & $\wedge$\\
$over (n - 2)$ & = & \verb,true, & $\wedge$ & $over(n - 3)$ & = & \verb,true,
\end{tabular}
\end{itemize}
\end{frame}

\begin{frame}[fragile]
\frametitle{Monitoring Properties Involving Real-Time with Copilot}
\begin{lstlisting}
propTempExceedThreshold :: Spec
propTempExceedThreshold =
  trigger "over_threshold" (overThresh) []

  where
  thresh = 500

  over :: Stream Bool
  over = [False, False, False, False] ++ (temp > thresh) -- delay

  temp :: Stream Float
  temp = extern "temp" Nothing

  overThresh :: Stream Bool
  overThresh = drop 3 over &&
               drop 2 over &&
               drop 1 over &&
               over

\end{lstlisting}
\end{frame}

\begin{frame}[fragile]
\frametitle{Monitoring Properties Involving Real-Time with Copilot}
\begin{lstlisting}
propTempExceedThreshold :: Spec
propTempExceedThreshold =
  trigger "over_threshold" (overThresh) []

  where
  thresh = 500

  over :: Stream Bool
  over = temp > thresh

  temp :: Stream Float
  temp = extern "temp" Nothing

  overThresh :: Stream Bool
  overThresh = over &&
               ([False] ++ over) &&
               ([False, False] ++ over) &&
               ([False, False, False] ++ over)

\end{lstlisting}
\end{frame}

\begin{frame}[fragile]
\frametitle{The LTL and PTLTL libraries}
\begin{itemize}
\item The LTL library implements bounded Linear Temporal Logic
  \begin{itemize}
  \item E.g., the property that $\lozenge^n p$ can be
  expressed as \verb,eventually n b,, where \verb,b, is a boolean
  stream indicating whether $p$ currently holds
  \item When using the library's functions, it is necessary that the streams
  have sufficient history to ``drop'' from
  \end{itemize}
\item The PTLTL library implements past time Linear Temporal Logic
  \begin{itemize}
  \item Other than \verb,previous,, all functions provided in the PTLTL
  library range over the entirety of a streams' history
    \begin{itemize}
    \item E.g., the property $\overset{\leftarrow}{\lozenge}p$ can be expressed as
    \verb,eventuallyPrev b,
    \end{itemize}
  \end{itemize}
\item Using these libraries, properties such as the one mentioned previously
become easier to express within a monitor
  \begin{itemize}
  \item Using bounded LTL and the assumed 0.1 second sampling period, the property can be
  written as
  $$\neg\square^{3}({\tt temp} > {\tt thresh})$$
  \end{itemize}
\end{itemize}
\end{frame}

\begin{frame}[fragile]
\frametitle{Monitoring Real-Time Properties with Bounded LTL}
\begin{lstlisting}
propTempExceedThreshold :: Spec
propTempExceedThreshold =
  trigger "over_threshold" (overThresh) []

  where
  thresh = 500

  over :: Stream Bool
  over = [False, False, False, False] ++ (temp > thresh) -- delay

  temp :: Stream Float
  temp = extern "temp" Nothing

  overThresh :: Stream Bool
  overThresh = always 3 over
\end{lstlisting}
\end{frame}

\begin{frame}
\frametitle{Drawbacks of Non-Explicit Real-Time}
In the monitors for the \texttt{propTempExceedThreshold} specification so far,
any connection to real-time values  was implicit;
the monitors were stated to assume a constant sampling period of 0.1 second.
\begin{itemize}
\item If the sampling period is changed, then it is then necessary to alter the
monitors each time this occurs
\item If samples do not occur with exactly 0.1 second in between (e.g. if one sample
is, for some reason, 0.1 second late), then the properties are no longer monitored
correctly
\end{itemize}
\end{frame}

\begin{frame}[fragile]
\frametitle{Metric Temporal Logic}
Metric Temporal Logic is an extension to Linear Temporal Logic where operators
are augmented with \emph{intervals} over which the properties should hold.
\begin{itemize}
\item An implementation of bounded MTL in Copilot that uses iterations
as time units rather than real-time units would resemble the bounded
LTL library but with each function accepting a lower bound in addition
to the upper bound
\item If the lower bound were set to zero, then the result would be the same as the LTL
library could provide; e.g., given an integer value \texttt{n} and streams \texttt{s0}
and \texttt{s1}, the following would result in the same streams:
 \begin{itemize}
 \item \verb,until n s0 s1,
 \item \verb,until 0 n s0 s1,
 \end{itemize}
\end{itemize}
\end{frame}

\begin{frame}[fragile]
\frametitle{Making Time More Explicit}
\begin{itemize}
\item The most straightforward way to incorporate real-time values into monitors
is by sampling the clock, i.e., introducing a \texttt{clk} stream
\item By adding \texttt{clk} and \texttt{dist} (the minimum sampling period) parameters
to MTL functions, intervals can refer to real-time values instead of iterations
\item So MTL properties such as $\mathbf{R}_{[l,u]} (p_1, p_2)$ can be expressed as
\verb,release l u clk dist b1 b2,
\end{itemize}
\end{frame}

\begin{frame}[fragile]
\frametitle{A Property Involving Real-Time}
In MTL, the property that the temperature should not exceed the
threshold temperature for 0.3 second (300 ms) or longer can be written
as:

$$\neg~\square_{[0,300]} (\mathtt{temps} > \mathtt{threshold})$$

For the purpose of monitoring, this property should instead be
expressed in terms of a past-time operator:

$$\neg~\overset{\leftarrow}{\square}_{[0,300]} (\mathtt{temps} > \mathtt{threshold})$$
\end{frame}

\begin{frame}[fragile]
\frametitle{Monitoring Real-Time Properties with MTL}
\begin{lstlisting}
propTempExceedThreshold :: Spec
propTempExceedThreshold =
  trigger "over_threshold" (overThresh) []

  where
  thresh = 500

  temp :: Stream Float
  temp = extern "temp" Nothing

  clk :: Stream Word64
  clk = extern "clk_ms" Nothing

  overThresh :: Stream Bool
  overThresh = alwaysBeen 0 300 clk 100 (temp > thresh)
\end{lstlisting}
\end{frame}
\begin{frame}[fragile]
\frametitle{Monitoring Real-Time Properties with MTL}
\begin{itemize}
\item No longer the requirement that the global sampling period is exactly 0.1 second
  \begin{itemize}
  \item The monitor instead assumes that the sampling period for all the streams used
  within the specification is \emph{at least} 0.1 second
  \item Increasing the global sampling period to e.g. 0.15 second requires no change
  to the monitor
  \end{itemize}
\item May set off ``false alarms'' until 300 ms pass
  \begin{itemize}
  \item Can prevent this at the expense of complicating the monitor slightly, e.g.
  changing the \verb,trigger, to
\begin{lstlisting}
trigger "over_threshold" (overThresh && canTrigger) []
\end{lstlisting}
  where
    \begin{itemize}
    \item \verb,canTrigger = (clk - hold(clk)) >= 300,
    \item
\begin{lstlisting}
hold s = s'
  where s' = mux ([True] ++ false) s ([0] ++ s')
\end{lstlisting}
    \end{itemize}   
  \end{itemize}
\end{itemize}
\end{frame}

\begin{frame}
\frametitle{Summary of Logic Libraries in Copilot}
\begin{itemize}
\item LTL
  \begin{itemize}
  \item an implementation of bounded Linear Temporal Logic
  \item the parameter \texttt{n} provides the number of \emph{iterations} over which
  the property should hold
  \end{itemize}
\item PTLTL
  \begin{itemize}
  \item an implementation of past-time Linear Temporal Logic
  \item properties should hold over the current and all previous sampled values
  \end{itemize}
\item MTL
  \begin{itemize}
  \item an implementation of bounded Metric Temporal Logic with
    \begin{itemize}
    \item bounds $l,u$ that must be nonnegative integers
    \item a time domain consisting of all time values sampled
    \end{itemize}
  \item the parameters \texttt{l} and \texttt{u} provide the \emph{real-time values}
  for which the property should hold
  \end{itemize}
\end{itemize}
\end{frame}
\end{document}
