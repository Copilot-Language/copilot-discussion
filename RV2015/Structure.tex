\section{Implementation}~\label{sec:structure} 

\begin{figure}[ht!]
  \begin{tikzpicture}[->, node distance=2.5cm, auto, shorten >=1pt, bend angle=45,
   thick]
    \tikzstyle{every state}=[rectangle, rounded corners]

    \node[state]         (Lang) 
         {
           \begin{tabular}[b]{l}
           Copilot Libraries\\ \hline Copilot Language
           \end{tabular}};
   \node[state]         (Core) [below of =Lang]          {Copilot Core};
   %\node[state]        (Cnub) [below of = Core]       {Cnub}; 
   \node[state]        (TransSys) [left of=Core] {$\mathrm{Transition  Sys}$};
   \node[state]        (IL) [right of =Core] {IL}; 
    \node[state]       (SmtLib) [right of=IL] {SMT LIB Format};
 %  \node[state]        (NMTransSys)[below left of = TransSys]   {$\mathrm{Transition  Sys}^{nm}$} ;
  % \node[state]        (MprimeTransSys)[below right of = TransSys] {$\mathrm{Transition  Sys}^{m}$} ;
    \node[state]       (Kind2Native) [left of = TransSys]{Kind 2 Format} ;
    \tikzstyle{every node}=[]

    \path 
    (Lang) edge   [ anchor=west, text width=7.00cm] node {Reification and DSL-specific type-checking} (Core)
  % (Core) edge              node {} (Cnub)
      (Core)   edge              node {} (TransSys)
                  edge              node {} (IL)
  %   (TransSys)     edge             node[swap] {Inline} (NMTransSys)
  %                      edge              node {Remove Cycles}  (MprimeTransSys)
  %  (NMTransSys) edge              node {} (Kind2Native)
    (TransSys)  edge  node  {}   (Kind2Native)   
   (IL) edge             node{} (SmtLib) ;
  \end{tikzpicture}
  \caption{The Copilot verification  toolchain.}
  \label{fig:toolchain}
\end{figure}

In this section, we shall outline the structure of the implementation
or our Copilot verification system as illustrated in
Figure~\ref{fig:toolchain}.  Copilot is deeply embedded in
Haskell.  A Copilot program is \emph{reified} (i.e. transformed from
a recursive structure into explicit graphs) and then domain specific
type checking is done. Our current implementation employs two internal
representations into the 
on whether one uses the light prover or the Kind2 proving engine. 

When using the \emph{light prover}, we translate the Copilot Core
language into \textbf{IL} format: a list of quantifier-free equations
over integer sequences, implicitly universally quantified by a free
variable \emph{n}. Each sequence roughly corresponds to a stream. This
format is similar to the one used in G.  Hagen's
thesis~\cite{HagenPhD}.  
In this format, a stream of type $a$ is modeled by a function of type $\mathbb{N} \to a$. Each stream definition is translated into a list of constraints on such functions. For instance, the stream definition :
\begin{lstlisting}
fib = [1, 1] ++ (fib + drop 1 fib)
\end{lstlisting}
is translated into a function $f : \mathbb{N} \to \mathbb{N}$ with the constraints :
$$
\begin{array}{c}
f(0) = 1 \\
f(1) = 1 \\
\forall n . \; f(n + 2) = f(n + 1) + f(n)
\end{array}
$$
Specifications in the \textbf{IL} format can be printed out in the
SMTLib format. 


When using the Kind2 prover, we the Copilot Core
language into   textbf{TransitionSys} format : a modular
representation of a \emph{state transition system}. This translation process is pretty straightforward : for each stream definition
\begin{center}\texttt{ s = [$s_0$,...,$s_p$] ++ e}\end{center}
we add a node with the same name as $s$ and the local variables $out_0, out_1, \cdots, \, out_p$ with the descriptors :

\begin{itemize}
\item $out_{i} = $  \texttt{Pre} $s_i$  $out_{i + 1}$
\item $out_{p} = $  \texttt{Expr} $\tilde e$  where $\tilde e$ is a translation of $e$ in the \textbf{TransSys} expression format, which is quite similar to the core expression format. Moreover :

\begin{itemize}
\item Local variables in the core format can be expressed by adding variables in the current node.
\item The core expression \texttt{Drop} $g$ $i$ is translated by $out_{p - i}$ if $g = s$. Otherwise, a local alias is defined for the variable $out_{p - i}$ of the node $g$ and is returned.
\end{itemize}

\end{itemize}
This format  is similar
to Kind2's native format thus interfacing with prover was
straightforward. 

