\section{Implementation}~\label{sec:structure} 

\begin{figure}[ht!]
  \begin{tikzpicture}[->, node distance=2.5cm, auto, shorten >=1pt, bend angle=45,
   thick]
    \tikzstyle{every state}=[rectangle, rounded corners]

    \node[state]         (Lang) 
         {
           \begin{tabular}[b]{l}
           Copilot Libraries\\ \hline Copilot Language
           \end{tabular}};
   \node[state]         (Core) [below of =Lang]          {Copilot Core};
   %\node[state]        (Cnub) [below of = Core]       {Cnub}; 
   \node[state]        (TransSys) [left of=Core] {$\mathrm{Transition  Sys}$};
   \node[state]        (IL) [right of =Core] {IL}; 
    \node[state]       (SmtLib) [right of=IL] {SMT LIB Format};
 %  \node[state]        (NMTransSys)[below left of = TransSys]   {$\mathrm{Transition  Sys}^{nm}$} ;
  % \node[state]        (MprimeTransSys)[below right of = TransSys] {$\mathrm{Transition  Sys}^{m}$} ;
    \node[state]       (Kind2Native) [left of = TransSys]{Kind 2 Format} ;
    \tikzstyle{every node}=[]

    \path 
    (Lang) edge   [ anchor=west, text width=7.00cm] node {Reification and DSL-specific type-checking} (Core)
  % (Core) edge              node {} (Cnub)
      (Core)   edge              node {} (TransSys)
                  edge              node {} (IL)
  %   (TransSys)     edge             node[swap] {Inline} (NMTransSys)
  %                      edge              node {Remove Cycles}  (MprimeTransSys)
  %  (NMTransSys) edge              node {} (Kind2Native)
    (TransSys)  edge  node  {}   (Kind2Native)   
   (IL) edge             node{} (SmtLib) ;
  \end{tikzpicture}
  \caption{The Copilot verification  toolchain.}
  \label{fig:toolchain}
\end{figure}

In this section, we shall outline the structure of the implementation
or our Copilot verification system as illustrated in
Figure~\ref{fig:toolchain}.  Copilot is deeply embedded in
Haskell.  A Copilot program is \emph{reified} (i.e. transformed from
a recursive structure into explicit graphs) and then domain specific
type checking is done. Our current implementation employs two internal
representations 
\begin{itemize}
\item
  The \textbf{IL} format : a list
  of quantifier-free equations over integer sequences, implicitly
  universally quantified by a free variable \emph{n}. Each sequence
  roughly corresponds to a stream.   This
format is similar to the one used in G.  Hagen's
dissertation~\cite{HagenPhD}, but somewhat customized for Copilot.   The \emph{light prover} works with this
  format.
\item The \textbf{TransSys} format : a modular representation of a
  \emph{state transition system}. The \emph{Kind2 prover} uses this
  format, which is similar to Kind2's native format.
\end{itemize}

\subsection{IL Format} \textbf{IL} format: a list of quantifier-free equations
over integer sequences, implicitly universally quantified by a free
variable \emph{n}. Each sequence roughly corresponds to a stream. This
format is similar to the one used in G.  Hagen's
thesis~\cite{HagenPhD}.  
In this format, a stream of type $a$ is modeled by a function of type $\mathbb{N} \to a$. Each stream definition is translated into a list of constraints on such functions. For instance, the stream definition :
\begin{lstlisting}
fib = [1, 1] ++ (fib + drop 1 fib)
\end{lstlisting}
is translated into a function $f : \mathbb{N} \to \mathbb{N}$ with the constraints :
$$
\begin{array}{c}
f(0) = 1 \\
f(1) = 1 \\
\forall n . \; f(n + 2) = f(n + 1) + f(n)
\end{array}
$$
Specifications in the \textbf{IL} format can be printed out in the
SMTLib format. 

paragraph{The translation process} 

The translation process is straightforward as the reification process~\cite{gill,pike-icfp-12} has previously transformed the copilot program such that the \texttt{(++)} operator only occurs at the top of a stream definition. Indeed, all stream definitions are written with the pattern
\begin{center}\texttt{ s = [$s_0$,...,$s_p$] ++ e}\end{center}
where \texttt{e} is an expression in which \texttt{(++)} does not occur. In \texttt{Core.Spec}, a stream is defined as :
\begin{lstlisting}[frame=single]
data Stream = forall a.Typed a => Stream
  { streamId         :: Id
  , streamBuffer     :: [a]
  , streamExpr       :: Expr a
  , streamExprType   :: Type a }
\end{lstlisting}
where the field \texttt{streamBuffer} corresponds to \texttt{[$s_0$,...,$s_p$]} and the field \texttt{streamExpr} to \texttt{e}. Moreover, a simplified definition of the type \texttt{Expr} defined in \texttt{Core.Spec} is :
\begin{lstlisting}[frame=single]
data Expr a where
  Const  :: Type a -> a -> Expr a
  Drop   :: Type a -> DropIdx -> Id -> Expr a
  Op2    :: Op2 a b c -> Expr a -> Expr b -> Expr c
\end{lstlisting}Then, the translation algorithm is the following. For each stream equation \begin{center}\texttt{ s = [$s_0$,...,$s_p$] ++ e},\end{center}we introduce a new fresh function $f_s$ and the following constraints :

\begin{itemize}
\item $f_s(i) = s_i$ \quad for $1 \leq i \leq p$
\item $\forall n . \; f_s(n + p) = \texttt{expr}_p \; e$
\end{itemize}
where $\texttt{expr}_p$ is defined in this way on expressions :
\begin{itemize}
\item $\texttt{expr}_p \; (\texttt{Const } v) = v$
\item $\texttt{expr}_p \; (\texttt{Drop } g \; i) = f_g(n + p - i)$
\item $\texttt{expr}_p \; (\texttt{Op2 } \oplus \; x_1 \; x_2) =  {expr}_p \; x_1 \; \oplus \; {expr}_p \; x_2 $
\end{itemize}





When using the Kind2 prover, we the Copilot Core
language into   textbf{TransitionSys} format : a modular
representation of a \emph{state transition system}. This translation process is pretty straightforward : for each stream definition
\begin{center}\texttt{ s = [$s_0$,...,$s_p$] ++ e}\end{center}
we add a node with the same name as $s$ and the local variables $out_0, out_1, \cdots, \, out_p$ with the descriptors :

\begin{itemize}
\item $out_{i} = $  \texttt{Pre} $s_i$  $out_{i + 1}$
\item $out_{p} = $  \texttt{Expr} $\tilde e$  where $\tilde e$ is a translation of $e$ in the \textbf{TransSys} expression format, which is quite similar to the core expression format. Moreover :

\begin{itemize}
\item Local variables in the core format can be expressed by adding variables in the current node.
\item The core expression \texttt{Drop} $g$ $i$ is translated by $out_{p - i}$ if $g = s$. Otherwise, a local alias is defined for the variable $out_{p - i}$ of the node $g$ and is returned.
\end{itemize}

\end{itemize}
This format  is similar
to Kind2's native format thus interfacing with prover was
straightforward. 



\subsection{The \textbf{TransSys} format}

Recall, a state transition system is a triple $(S,I,T),$
where $S$ is a set of states, $I \subseteq S$ is the set of initial
states and $T \subseteq S \times S $ is a transition relation over $S$.
Here, a state consists of  the values of a finite set of variables, with types belong to $\{ \texttt{Int}, \texttt{Real},  \texttt{Bool}\}$. $I$ is encoded by a logical formula whose free variables correspond to the state variables and that holds for a state $q$ if and only if $q$ is an initial state. Similarly, the transition relation is given by a formula $T$ such that $T\left[\, q, \, q' \,\right]$ holds if and only if $q \rightarrow q'$. 

The \textbf{TransSys} format is a modular encoding of such a state transition system. Related variables are grouped into \textit{nodes}, each node providing a distinct namespace and expressing some constraints between its variables. 



\paragraph{Formal definition}

Inside a node, a variable is referenced by a string identifier. Indeed, the corresponding \texttt{Var} type is defined as :
\begin{lstlisting}[frame=single]
data Var = Var {varName :: String}
\end{lstlisting}
Outside a node, a variable is referenced by a node identifier and an instance of \texttt{Var} :
\begin{lstlisting}[frame=single]
data ExtVar = ExtVar {extVarNode :: NodeId, extVarLocalPart :: Var} 
\end{lstlisting}
where \texttt{NodeId = String}. Then, the \texttt{Node} type is defined by :
\begin{lstlisting}[frame=single]
data Node = Node
  { nodeId            :: NodeId
  , nodeDependencies  :: [NodeId]
  , nodeLocalVars     :: Map Var VarDescr
  , nodeImportedVars  :: Bimap Var ExtVar 
  , nodeConstrs       :: [Expr Bool] }
\end{lstlisting} 
where :
\begin{itemize}
\item \texttt{nodeId} is the identifier of the node
\item \texttt{nodeImportedVars} contains the bijection between the external variables used in the node and their local aliases. Indeed, inside a node we cannot refer to an external variable (a variable belonging to another node). Therefore, a mean to bind it to a local name is provided.
\item \texttt{nodeDependencies} is the list of the nodes from which a variable is given a local alias. This information is redundant.
\item \texttt{nodeLocalVars} is a dictionnary which binds each local variable to its descriptor. The type of descriptors \texttt{VarDescr} will be described later.
\item \texttt{nodeConstrs} is a list of constraints on local variables.
\end{itemize}
A variable descriptor is comprised  of a type and a definition :
\begin{lstlisting}[frame=single]
data VarDescr = forall t.VarDescr
  { varType :: Type t
  , varDef  :: VarDef t }
\end{lstlisting}
Note that in the same way as in \texttt{Copilot.Core}, {\sc
  gadt}s~\cite{Xi2003,CheneyHinze03, Johann2008} are used to achieve some additional type safety. The \texttt{VarDef} type is defined as :
\begin{lstlisting}[frame=single]
data VarDef t =
    Pre t Var
  | Expr (Expr t)
  | Constrs [Expr Bool]
\end{lstlisting}
As we can see, a local variable can be defined
\begin{itemize}
\item as the previous value of another variable, an initial value being given too (\texttt{Pre} constructor)
\item by an expression involving other variables (\texttt{Expr} constructor)
\item implicitly by a serie of constraints (\texttt{Constrs} constructor)
\end{itemize}
Note that the last constructor can be used to achieve some non-determinism. For instance, a variable whose \texttt{varDef} field is \texttt{Constrs []} is left totally unconstrained~\footnote{In fact, this is not exact as each constraint which is put in a \texttt{Constrs} constructor could have been put in the \texttt{nodeConstrs} field of the current node instead. However, this is to be avoided for readability and the \texttt{nodeConstrs} field should be left empty as often as possible.
}. The expression type associated with the second constructor is defined as :
\begin{lstlisting}[frame=single]
data Expr t where
  Const  :: Type t -> t -> Expr t
  Ite    :: Type t -> Expr Bool -> Expr t -> Expr t -> Expr t
  Op1    :: Type t -> Op1 x t -> Expr x -> Expr t
  Op2    :: Type t -> Op2 x y t -> Expr x -> Expr y -> Expr t
  VarE   :: Type t -> Var -> Expr t
\end{lstlisting}
Therefore, an expression is a combination of constants, operators, and local variable names.


Finally, a specification in the \textbf{TransSys} format is comprised
of  a list of node and a dictionary binding property names to boolean variables which have to be shown constant equals to \texttt{true}.
\begin{lstlisting}[frame=single]
data Spec = Spec
  { specNodes :: [Node]
  , specProps :: Map PropId ExtVar }
\end{lstlisting}


\paragraph{Semantics of a transition system}
We give a semantics to \textbf{TransSys} specifications by extracting from them a state transition system in the form of a list of variables and two boolean formulas $I$ and $T$. 

The list of variables is simply a concatenation of the local variables of all nodes. Then, we define the formula $C$ as the concatenation of the following clauses, for each node :


\begin{itemize}
\item A series of equalities $a_i = b_i$ for each entry $(a_i, b_i)$ in \texttt{nodeImportedVars}
\item Each constraint defined in \texttt{nodeConstrs}
\item For each local variable $v$ defined in \texttt{nodeLocalVars} :

\begin{itemize}
\item If the \texttt{Expr} $e$ constructor is used, add the constraint $v = e$
\item If the \texttt{Constrs cs} constructor is used, add the constraints \texttt{cs} 
\item If the \texttt{Pre} constructor is used, add no constraint
\end{itemize}

\end{itemize}


Then, we define a formula $T_0[\vec q, \vec q \,']$ where $\vec q$ is the vector of variables representing the current state. Moreover, $\vec q \,'$ is a vector containing the same variables, primed, and represents the \textit{next} state. $T_0$ is defined as the conjunction of the following clauses, for each node and each local variable $u$ which descriptor is \texttt{Pre \_} $v$ :
\begin{center} $u' = v$ \end{center}
After this, we define a formula $I_0[\vec q]$ that  is the conjunction of the following clauses, for each node and each local variable $u$ which descriptor is \texttt{Pre} $a$ \_ :
\begin{center} $u = a$ \end{center}
Finally, we define $I$ and $T$ as :
\[  I[\vec q] =  I_0[\vec q] \;\wedge\; C[\vec q] \qquad T[\vec q, \; \vec q \, '] = T_0[\vec q, \; \vec q \,'] \, \wedge \, C[\vec q \, ']  \]




\paragraph{Translation from the \texttt{Core} format to the \texttt{TransSys} format}

This translation process is pretty straightforward : for each stream definition
\begin{center}\texttt{ s = [$s_0$,...,$s_p$] ++ e}\end{center}
we add a node with the same name as $s$ and the local variables $out_0, out_1, \cdots, \, out_p$ with the descriptors :

\begin{itemize}
\item $out_{i} = $  \texttt{Pre} $s_i$  $out_{i + 1}$
\item $out_{p} = $  \texttt{Expr} $\tilde e$  where $\tilde e$ is a translation of $e$ in the \textbf{TransSys} expression format, which is quite similar to the core expression format. Moreover :

\begin{itemize}
\item Local variables in the core format can be expressed by adding variables in the current node.
\item The core expression \texttt{Drop} $g$ $i$ is translated by $out_{p - i}$ if $g = s$. Otherwise, a local alias is defined for the variable $out_{p - i}$ of the node $g$ and is returned.
\end{itemize}

\end{itemize}
