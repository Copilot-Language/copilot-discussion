\section{Background on SMT-based $k-$induction}~\label{sec:background} 
The focus of our  investigation has  been on applying model checking to  prove
invariant properties of our monitors.   We  
employ a powerful technique known as $k-$induction~\cite{Sheeran00,
  EenS03} that has been demonstrated to be successful for verifying inductive
properties of infinite state systems.   $k-$induction  has the
advantage that it is well suited  to  SMT 
based bounded model checking. This section profiles the
basic concepts of the  $k-$induction proof technique needed in the
remainder of the paper. In practice, we
use tools that implement  enhancements of the basic procedure such as path
compression~\cite{dMRS03} that help the process scale, but are beyond
the focus of the paper. 

Consider  a state transition system  $(S,I,T),$
where $S$ is a set of states, $I \subseteq S$ is the set of initial
states and $T \subseteq S \times S $ is a transition relation over
$S.$ To show $P$ holds in the transition system one must first
show the base case holds, that is $P$ holds in all states reachable
from an initial state in $k$ steps, and then show the induction step,
that if $P$ holds in states $s_0,\ldots,s_{k-1}$ then it holds in
state $s_k.$ The $k$-induction principle is formally expressed in the
following two entailments:
\begin{eqnarray*}
I(s_0) \tland T(s_0,s_1) \tland \cdots \tland T(s_{k-1},s_k) &\models&
P(s_k) \\
P(s_0) \tland \cdots \tland P(s_{k-1}) \tland T(s_0,s_1) \tland \cdots \tland T(s_{k-1},s_k) &\models&
P(s_k) 
\end{eqnarray*} 
If one cannot show the property to be true, the
property is strengthened by extending the formula and
progressively increasing the length of the reachable states
considered.  

Property $P$ said to be a $k$-inductive property with respect to
$(S,I,T)$ if there exists some $k \in \mathbb{N}^{0<}$ such that $P$
satisfies the $k$-induction principle. As $k$ increases, weaker
invariants may be proved. If $P$ is a safety property that doesn't hold, then the first entailment will break for a finite $k$ and a counterexample will be provided. The trick is to find an invariant that is
tractable by the SMT solver yet weak enough to satisfy the desired
property.  

\jonathan{Maybe I'll add a few words about IC3 here}
