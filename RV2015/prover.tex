\section{Copilot Prover Interface}~\label{sec:prover} 
We have modified  the Copilot framework with an extensible interface  to multiple
provers, called \texttt{copilot-kind}, aimed at checking  safety properties on Copilot
programs. Currently Copilot interacts with two provers: the first is a
homegrown prover we call ``the light prover'' built on top of
Yices~\cite{Dutertre:cav2014} and the second is the Kind2 model checker~\cite{kind2} being developed at
the University of Iowa.  The 

Let us illustrate the use of the verification functionality with a
very simple example.  Safety properties are simply expressed with
standard boolean streams.  We bind a boolean stream to a property name
with the \texttt{prop} construct in the specification, where the
specification has type \texttt{Spec}.

For instance, here is a straightforward specification declaring one
property :

\begin{lstlisting}[frame=single]
spec :: Spec
spec = do
  prop "gt0" (x > 0)
  where
    x = [1] ++ (1 + x)
\end{lstlisting}

Let's say we want to check that \texttt{gt0} holds. For this, we use
the function

\begin{lstlisting}[frame=single]
    prove :: Prover -> ProofScheme -> Spec -> IO ()
\end{lstlisting}

Here, we can discharge the proof by  using the light prover using the command:
\begin{code}
prove (lightProver def) (check "gt0") spec
\end{code}
where \texttt{lightProver def} stands for the \emph{light prover} with
default configuration.  

\subsubsection{The Prover Interface.}\label{the-prover-interface}

The \texttt{Copilot.Kind.Prover} defines a general interface for
provers. Therefore, it is really easy to add a new prover by creating a
new object of type \texttt{Prover}. The latter is defined as follows:

\begin{lstlisting}[frame=single]
data Cex = Cex

type Infos = [String]

data Output = Output Status Infos

data Status
  = Valid
  | Invalid (Maybe Cex)
  | Unknown
  | Error
  
data Feature = GiveCex | HandleAssumptions
  
data Prover = forall r.Prover 
  { proverName     :: String
  , hasFeature     :: Feature -> Bool
  , startProver    :: Core.Spec -> IO r
  , askProver      :: r -> [PropId] -> PropId -> IO Output 
  , closeProver    :: r -> IO () 
  }

\end{lstlisting}

To interface with a  particular prover, one must  provide a \texttt{askProver} function that
takes as an argument  the following three parameters: 
\begin{itemize} 
\item  The prover descriptor 
\item  A list of assumptions 
\item  A conclusion
\end{itemize} 
and uses the prover to check if the assumptions logically entail the conclusion.

Two provers are currently supported by default : \texttt{Light} and \texttt{Kind2}.

\subsubsection{The Light Prover.}\label{the-light-prover}

The \emph{light prover} is a really simple prover which uses the Yices
SMT solver with the \texttt{QF\_UFLIA} theory and is limited to prove
\emph{k-inductive} properties.

\subsubsection{The Kind2 Prover.}\label{the-kind2-prover}

The \emph{Kind2} prover uses the model checker with the same name, from
the University of Iowa. It translates the Copilot specification into a
\emph{modular transition system} (the Kind2 native format) and then
calls the \texttt{kind2} executable.

%It is provided by the \texttt{Copilot.Kind.Kind2} module, which exports
%a $$\texttt{kind2Prover :: Options -\textgreater{} Prover}$$ where the
%\texttt{Options} type is defined as
%
%\begin{lstlisting}[frame=single]
%data Options = Options { bmcMax :: Int }
%\end{lstlisting}

%and where \texttt{bmcMax} corresponds to the \texttt{-\/-bmc\_max}
%option of \emph{kind2} and is equivalent to the \texttt{kTimeout} option
%of the light prover. Its default value is $0,$ which stands for infinity.

\subsubsection{Combining Provers.}\label{combining-provers}

The
$$\texttt{combine :: Prover -\textgreater{} Prover -\textgreater{} Prover}$$
function lets you merge two provers A and B into a prover C which
launches both A and B and returns the \emph{most precise} output. It
would be interesting to implement other merging behaviours in the
future. For instance, a \emph{lazy} one such that C launches B only if A
has returns \emph{unknown} or \emph{error}.

As an example, the following prover is used in \texttt{Driver.hs} :

\begin{lstlisting}[frame=single]
prover =
  lightProver def {onlyBmc = True, kTimeout = 5} 
  `combine` kind2Prover def
\end{lstlisting}

We will discuss the internals and the experimental results of these
provers later.

\subsubsection{Proof Schemes.}\label{proof-schemes}

Let's consider again this example :

\begin{lstlisting}[frame=single]
spec :: Spec
spec = do
  prop "gt0"  (x > 0)
  prop "neq0" (x /= 0)
  where
    x = [1] ++ (1 + x)

\end{lstlisting}
and let us say we want to prove \texttt{"neq0"}. Currently, the two
available solvers fail at showing this non-inductive property (we will
discuss this limitation later). Therefore, we can prove the more general
inductive lemma \texttt{"gt0"} and deduce our main goal from this. For
this, we use the proof scheme

\begin{lstlisting}[frame=single]
assert "gt0" >> check "neq0"
\end{lstlisting}


instead of just 
\begin{lstlisting}[frame=single] 
  check "neq0" 
\end{lstlisting}
A proof scheme is chain of
primitives schemes glued by the $\texttt{\textgreater{}\textgreater{}}$
operator. For now, the available primitives are :

\begin{itemize}
\itemsep1pt\parskip0pt\parsep0pt
\item
  \texttt{check "prop"} checks whether or not a given property is true
  in the current context.
\item
  \texttt{assume "prop"} adds an assumption in the current context.
\item
  \texttt{assert "prop"} is a shortcut for
  \texttt{check "prop" \textgreater{}\textgreater{} assume "prop"}.
\item
  \texttt{assuming :: {[}PropId{]} -\textgreater{} ProofScheme -\textgreater{} ProofScheme}
  is such that \texttt{assuming props scheme} assumes the list of
  properties \emph{props}, executes the proof scheme \emph{scheme} in
  this context, and forgets the assumptions.
\item
  \texttt{msg "..."} displays a string in the standard output
\end{itemize}
