\section{Copilot Prover Interface}~\label{sec:prover} 
The \texttt{copilot-kind} model-checker is an extensible interface to multiple
provers used to verify safety properties of Copilot
programs. Currently, two backends for \texttt{copiot-kind} have been implemented: the first is a
homegrown prover we call ``the light prover'' built on top of
Yices~\cite{Dutertre:cav2014} and the second is the Kind2 model checker being developed at
the University of Iowa~\cite{kind}.

To begin, we describe how safety properties are specified in Copilot. Using the
``synchronous observer'' approach~\cite{amast93}, properties \emph{about}
Copilot programs are specified \emph{within} Copilot itself. In particular,
properties are encoded with standard boolean streams; and Copilot streams are
sufficient to encode past-time linear temporal logic~\cite{ptltl}. We bind a boolean stream
to a property name with the \texttt{prop} construct in the specification, where
the specification has type \texttt{Spec}. 

For instance, here is a straightforward specification declaring one
property:

\begin{lstlisting}[frame=single]
spec :: Spec
spec = do
  prop "gt0" (x > 0)
  where
    x = [1] ++ (1 + x)
\end{lstlisting}

In order  to check that property \texttt{gt0} holds, we use a \texttt{prove}
function implemented as part of \texttt{copilot-kind}.
%% \begin{lstlisting}[frame=single]
%%     prove :: Prover -> ProofScheme -> Spec -> IO ()
%% \end{lstlisting}
Here, we can discharge the proof-obligation for the program above with the light prover using the command:
\begin{code}
prove (lightProver def) (check "gt0") spec
\end{code}
where \texttt{lightProver def} stands for the \emph{light prover} with
default configuration.

%% \subsubsection{The Prover Interface.}\label{the-prover-interface}

%
%The latter is defined as follows:
%
%\begin{lstlisting}[frame=single]
%data Output = Output Status Infos
%type Infos = [String]
%data Status
%  = Valid
%  | Invalid (Maybe Cex)   -- 'Cex' is the type for counterexamples
%  | Unknown
%  | Error
%  
%data Prover = forall r. Prover 
%  { proverName     :: String
%  , hasFeature     :: Feature -> Bool
%  , startProver    :: Core.Spec -> IO r
%  , closeProver    :: r -> IO () 
%  , askProver      :: r          -- ^ The prover internal state
%                   -> [PropId]   -- ^ A list of assumptions A
%                   -> PropId     -- ^ The property to prove P
%                   -> IO Output  -- ^ Whether or not A entails P
%  }
%
%\end{lstlisting}
%

%%  \texttt{Copilot-Kind} provides a general interface for
%% provers. A new prover is created by implementing a
%% new prover object.
%% To interface with a particular prover, one must mostly provide an \texttt{askProver} function that
%% takes as its arguments the internal state of the prover, a list of assumptions $(A_i)_i$ and a property $P$ and decides whether or not : \[ \bigwedge_i A_i \models P  \] In the case the answer is \emph{no}, a counterexample trace leading to a state where $\bigwedge_i A_i \,\wedge\, \neg P$ is true is given. Moreover, even if the prover was given no time limit, we might still get the answer \emph{unknown} as the request might not be expressible in a decidable subset of a \textsc{smt-lib} theory\footnote{For instance, even Peano arithmetic is undecidable in general.}.
%% %\begin{itemize}
%% %\item \texttt{Valid} if the entailment holds.
%% %\item \texttt{Invalid} if it does not, in which case.
%% %\item \texttt{Unknown} if the prover can't find out. This type of answer is necessary even if the %prover has no time limit, as 
%% %\item \texttt{Error} when something wrong happens.
%% %\end{itemize}
%% Two provers are currently supported by default :
%% \begin{itemize}
%% \item The homegrown \emph{light prover} which uses the Yices \textsc{smt}-solver with the \textsc{qf\_uflia} theory and which implements the basic $k$-induction algorithm.
%% \item A prover based on the \emph{Kind2} model checker, which implements both $k$-induction and the more subtle IC3 algorithm.
%% \end{itemize}


%It is provided by the \texttt{Copilot.Kind.Kind2} module, which exports
%a $$\texttt{kind2Prover :: Options -\textgreater{} Prover}$$ where the
%\texttt{Options} type is defined as
%
%\begin{lstlisting}[frame=single]
%data Options = Options { bmcMax :: Int }
%\end{lstlisting}

%and where \texttt{bmcMax} corresponds to the \texttt{-\/-bmc\_max}
%option of \emph{kind2} and is equivalent to the \texttt{kTimeout} option
%of the light prover. Its default value is $0,$ which stands for infinity.

\subsubsection{Combining Provers.}\label{combining-provers}
\texttt{Copilot-Kind} allows provers to be combined. Given provers $A$ and $B$,
the \texttt{combine} function returns a prover $C$ which launches both $A$ and
$B$ and returns the most \emph{precise} output of the two upon
termination. ``Precise'' in this case means returning the result least element
in the following partial order: for a given execution, classify prover outputs
as valid ($V$), unknown ($U$), invalid with countexample ($C$), and invalid with
no counterexample ($N$); the partial order is the least relation such that

$$V \leq U, \; C \leq U, \; N \leq U, \; C \leq N$$

\noindent
hold as well as reflexivity, antisymmetry, and transitivity. (Merging provers
that handles non-termination within a bound is future work.)

In practice, we used the following prover in the examples of Section~\ref{sec:example}:
\begin{lstlisting}[frame=single]
prover = lightProver def {kTimeout = 5} `combine` kind2Prover def
\end{lstlisting}
which uses both the light and Kind2 provers, the first being limited to 5 steps of the $k$-induction 

\subsubsection{Proof Schemes.}\label{proof-schemes}

Consider the example :
\begin{lstlisting}[frame=single]
spec :: Spec
spec = do
  prop "gt0"  (x > 0)
  prop "neq0" (x /= 0)
  where
    x = [1] ++ (1 + x)

\end{lstlisting}
and suppose we want to prove \texttt{"neq0"}. Currently, the two available solvers fail at showing this non-inductive property. Yet, we can prove the more general
inductive lemma \texttt{"gt0"} and deduce our main goal from this. For
this, we use the proof scheme

\begin{lstlisting}[frame=single]
assert "gt0" >> check "neq0"
\end{lstlisting}

\noindent
rather than
\begin{lstlisting}[frame=single] 
  check "neq0" 
\end{lstlisting}
A \emph{proof scheme} is a chain of
primitives schemes glued by the $\texttt{\textgreater{}\textgreater{}}$
operator to combine proofs, and in particular, provide lemmas. The available primitives are:

\begin{itemize}
\itemsep1pt\parskip0pt\parsep0pt
\item
  \texttt{check "prop"} checks whether or not a given property is true
  in the current context.
\item
  \texttt{assume "prop"} adds an assumption in the current context.
\item
  \texttt{assert "prop"} is a shortcut for
  \texttt{check "prop" \textgreater{}\textgreater{} assume "prop"}.
\item

\texttt{assuming props scheme} assumes the list of
  properties \emph{props}, executes the proof scheme \emph{scheme} in
  this context, and forgets the assumptions.
%% The signature of \texttt{assuming} is  
%%  {\texttt{{[}PropId{]} -\textgreater{} ProofScheme -\textgreater{} ProofScheme}}.

\item
  \texttt{msg "..."} displays a string in the standard output.
\end{itemize}

%% Although very simple, such a mechanism is necessary to make our model-checking approach scale up by splitting a proof in small lemmas. Some possible enhancements of it will be discussed in Section~\ref{sec:conclusion}.
