\section{Copilot Prover Interface}~\label{sec:prover} 
We have modified  the Copilot framework with an extensible interface to multiple
provers aimed at checking  safety properties on Copilot
programs. Currently Copilot interacts with two provers: the first is a
homegrown prover we call ``the light prover'' built on top of
Yices~\cite{Dutertre:cav2014} and the second is the Kind 2 prover~\cite{ZZZ} being developed at
the University of Iowa.  

Let us illustrate the use of the verification functionality with a
very simple example.  Safety properties are simply expressed with
standard boolean streams.  We bind a boolean stream to a property name
with the \texttt{prop} construct in the specification, where the
specification has type \texttt{Spec}.

For instance, here is a straightforward specification declaring one
property :

\begin{lstlisting}[frame=single]
spec :: Spec
spec = do
  prop "gt0" (x > 0)
  where
    x = [1] ++ (1 + x)
\end{lstlisting}

Let's say we want to check that \texttt{gt0} holds. For this, we use
the function

\begin{lstlisting}[frame=single]
    prove :: Prover -> ProofScheme -> Spec -> IO ()
\end{lstlisting}

Here, we can discharge the proof by  using the light prover using the command:
\begin{code}
prove (lightProver def) (check "gt0") spec
\end{code}
where \texttt{lightProver def} stands for the \emph{light prover} with
default configuration.  

