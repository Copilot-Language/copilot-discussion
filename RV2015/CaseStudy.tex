\subsection{Example}\label{sec:example}
The Copliot research team has previously investigated fault-tolerant
runtime verification~\cite{pike-isse-13}. In this work we employed the Boyer-Moore
Majority Vote Algorithm~\cite{MooreBoyer82,Hesselink2005} to determine
if in a given list of votes there is a candidate that has more than
half of the votes, and if so, of finding this candidate. The algorithm
operates in two passes, first it chooses a candidate and the second
pass verifies that the candidate is indeed a majority.  The algorithm
is subtle and the desire to apply formal verification to our Copilot
implementation helped motivate the effort described here. 


\begin{figure*}[ht]
\begin{lstlisting}[frame=single]
majority :: (Eq a, Typed a) => [Stream a] -> Stream a
majority []     = error "majority: empty list!"
majority (x:xs) = majority' xs x 1

majority' []     can _   = can
majority' (x:xs) can cnt =
  local
    (if cnt == 0 then x else can) $
      \ can' ->
        local (if cnt == 0 || x == can then cnt+1 else cnt-1) $
          \ cnt' ->
            majority' xs can' cnt'
\end{lstlisting}
\caption{The first pass of the majority vote algorithm in Copilot.}
\label{fig:majority}
\end{figure*}

%\begin{figure*}[ht]
%\begin{lstlisting}[frame=single]
%aMajority :: (Eq a, Typed a) => [Stream a] -> Stream a -> Stream Bool
%aMajority xs can = aMajority' 0 xs can > (fromIntegral (length xs) `div` 2)
%
%aMajority' cnt []     _   = cnt
%aMajority' cnt (x:xs) can =
%  local
%    (if x == can then cnt+1 else cnt) $
%      \ cnt' ->
%        aMajority' cnt' xs can
%\end{lstlisting}
%\caption{The second pass of the majority vote algorithm in Copilot.}
%\label{fig:amajority}
%\end{figure*}

The Copilot implementations of the first a
pass is  given in Figure \ref{fig:majority} and Figure \ref{fig:amajority}
respectively. 
