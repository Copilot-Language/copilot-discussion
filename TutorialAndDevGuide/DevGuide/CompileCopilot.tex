\section{Compiling Copilot}~\label{sec:compile}

This section describes hot to compile the Copilot code in a Cabal sandbox.

What you need to do : 
\begin{enumerate}
\item First install all the useful stuff : CompCert (2.4 or later, coq needed, opam recommanded), Frama-c (version sodium or later), Z3 (at least 4.3.2), CVC4 (at least 1.4), ghc 7.10 or later (and its whole toolchain, like the haskell platform).
\subitem Optionnaly, you could install splint, an up to date gcc (5.1 or later).
\item Make sure that cabal is installed, and update it to the version 1.22 or later (run \texttt{cabal install cabal cabal-install } for that). 
\item create a directory named Copilot-Language and cd into it. Then clone the following :
\subitem https://github.com/Copilot-Language/examplesForACSL
\subitem https://github.com/LeventErkok/sbv
\subitem https://github.com/Copilot-Language/Copilot
\item cd into the directory named Copilot and then do the following
\subitem \texttt{git submodule update --init}
\subitem go into the submodules and change the branches you want (for example lib/copilot-sbv has a branch named acsl) with the command \texttt{git checkout BRANCHNAME}. 
\subitem In the Copilot folder (the one cded in 3) \texttt{make test}
\item Normally, it should have failed ! Because it has installed all dependecies on their official release in a cabal sandbox located in this folder Copilot (if not, you should probably do a \texttt{cabal install --only-dependencies}). But you need at least a sbv 5.0 (which is not released yet). You have now to do the following :
\subitem Go into Copilot-Language/sbv and do \texttt{cabal sandbox init --sandbox ../Copilot/.cabal-sandbox/} and then \texttt{cabal install}.
\item Now go in Copilot and redo make test : it should compile everything, but the Copilot-regression should fail (which is normal because it is deprecated).
\item It's done ! You now have the latest version of Copilot in your sandbox ! To run some test, go into the Copilot-Language/examplesForACSL
\item choose one example randomly (choose the 29 if you have a good computer). Cd into it
\item do \texttt{cabal sandbox init --sandbox ../../Copilot/.cabal-sandbox/} and then \texttt{make sandbox} to build it from the sandbox.
\item It should have created you a folder named copilot-sbv-codegen. Cd into it :
\subitem do \texttt{make fwp} to use the wp plugin of frama-c to check, \texttt{make all} to compile it with compcert (into a internal.a library), \texttt{make splint} to check it with splint, \texttt{make fval} to check it with value analysis plugin (it will unroll the infinite loop, so the analysis may never terminate).
\item For the examples 29 and bigger, there is also a m4 preprocessing done before compilation. Do not hesitate to use it if you need it !
\end{enumerate}


\subsection{Updating A Build}
When you need to update your build here are the steps.

\begin{enumerate}
\item First you have to pull all the following (according to the branch you want to execute) :
\subitem sbv
\subitem examplesForACSL
\subitem Copilot
\subitem Copilot/lib/*
\item Then, go in Copilot, and \texttt{make veryclean}, then \texttt{make test}. When it fails, go to sbv and \texttt{cabal sandbox init --sandbox ../Copilot/.cabal-sandbox/} and \texttt{cabal install}. Then go again in Copilot and \texttt{make test}. Then you can go in the examples and make them again (if they do not compile, do a \texttt{cabal sandbox init --sandbox ../../Copilot/.cabal-sandbox/})
\end{enumerate}a



