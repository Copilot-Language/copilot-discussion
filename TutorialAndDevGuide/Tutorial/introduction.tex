
\section{Introduction} \label{sec:introduction}

Copilot is embedded into the functional programming language Haskell
\cite{PeytonJones02}.  A working knowledge of Haskell is necessary to use
Copilot effectively; a variety of books and free web resources introduce Haskell.
Copilot uses Haskell language extensions
specific to the Glasgow Haskell Compiler (GHC).

\subsection{Uses} \label{uses}
\textbf{Here we will discuss sampling, runtime verification, and practical uses of the copilot language.}

\subsection{Domain} \label{domain}

Copilot is a domain-specific language tailored to programming \emph{runtime
monitors} for \emph{hard real-time}, \emph{distributed}, \emph{reactive systems}.
Briefly, a runtime monitor is program that runs concurrently with a target program
with the sole purpose of assuring that the target program behaves in accordance with a
pre-established specification. Copilot is a language for writing such specifications.

A reactive system is a system that responds continuously to its environment.
All data to and from a reactive system are communicated progressively during
execution. Reactive systems differ from transformational systems which transform
data in a single pass and then terminate, as for example compilers and numerical
computation software.

A hard real-time system is a system that has a statically bounded execution time
and memory usage.  Typically, hard real-time systems are used in
mission-critical software, such as avionics, medical equipment, and nuclear power
plants; hence, occasional dropouts in the response time or crashes are not
tolerated.

A distributed system is a system which is layered out on multiple pieces of hardware.
The distributed systems we consider are all synchronized, i.e., all components agree on
a shared global clock.


\subsection{Installation} \label{sec:install}

Before downloading the copilot source code, you must first install an up-to-date version of GHC.
(The minimal required version is 7.10.1)
The easiest way to do this is to download and install the Haskell Platform,
which is freely distributed from here:

\begin{center}
\url{http://hackage.haskell.org/platform}
\end{center}

\noindent Because Copilot compiles to C code, you must also install a C compiler such as gcc (\url{https://gcc.gnu.org/install/}).  \textbf{XCODE?} After having installed the Haskell Platform and C compiler, Copilot can be downloaded and
installed in the following two ways: 

\begin{itemize}
\item \textbf{Using Cabal: } Copilot can be downloaded and installed by executing the following command:

\begin{code}
> cabal install copilot
\end{code}

\noindent This should, if everything goes well, install Copilot on your system.

\item \textbf{Using Git: } 

\begin{code}
     git clone https://github.com/Copilot-Language/Copilot.git
     git submodule update --init
     make test
\end{code}
\end{itemize}

\noindent To use the language, your Haskell module should contain the following import:
%
\begin{code}
import Language.Copilot  
\end{code}
%
To use the back-ends, import them, respectively:
%
\begin{code}
import Copilot.Compile.C99
import Copilot.Compile.SBV
\end{code}
%
If you need to use functions defined in the Prelude that are redefined by
Copilot (e.g., arithmetic operators), import the Prelude as qualified:
%
\begin{code}
import qualified Prelude as P  
\end{code}

\subsection{Structure} \label{structure}

\noindent Copilot is distributed through a series of packages at Hackage:

\begin{itemize}
\item copilot-language: Contains the language front-end.
\item copilot-theorem: Contains some extensions to the language for proving
properties about copilot programs using various SMT solvers and model checkers.
\item copilot-core: Contains an intermediate representation for Copilot programs (shared by all back-ends).
\item copilot-c99: A back-end for Copilot targeting C99 (based on Atom, \url{http://hackage.haskell.org/package/atom}). \textbf{Not updated anymore, might be deprecated soon.}
\item copilot-sbv: A back-end for Copilot targeting C99 (based on SBV, \url{http://hackage.haskell.org/package/sbv}).
\item copilot-libraries: A set of utility functions for Copilot, including a clock-library, a linear temporal logic framework,
a voting library, and a regular expression framework.
\item copilot-cbmc: A driver for proving the correspondence between code
  generated by the copilot-c99 and copilot-sbv back-ends.
\end{itemize}

Many of the examples in this paper can be found at
\url{https://github.com/Copilot-Language/Copilot/tree/copilot2.0/Examples}.


