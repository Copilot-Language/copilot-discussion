
\section{Introduction} \label{sec:introduction}

 
Neither formal verification nor testing can ensure system reliability.
Over 25 years ago, Butler and Finelli showed that testing alone cannot verify
the reliability of ultra-critical software~\cite{butler}.
\emph{Runtime verification} (RV)~\cite{monitors}, where monitors
detect and respond to property violations at runtime, has the
potential to enable the safe operation of safety-critical systems that
are too complex to formally verify or fully test.  Technically
speaking, a RV monitor takes a logical specification $\phi$ and
execution trace $\tau$ of state information of the system under
observation (SUO) and decides whether $\tau$ satisfies $\phi$. The
\emph{Simplex Architecture}~\cite{simplex} provides a model
architectural pattern for RV, where a monitor checks that the
executing SUO satisfies a specification and, if the property is
violated, the RV system will switch control to a more conservative
component that can be assured using conventional means that
\emph{steers} the system into a safe state.  \emph{High-assurance} RV
provides an assured level of safety even when the SUO itself cannot be
verified by conventional means.


Copilot is a RV framework for specifying and generating monitors for C programs that is
embedded into the functional programming language Haskell
\cite{PeytonJones02}.  A working knowledge of Haskell is not necessary to use
Copilot at a basic level. However, knowledge of Haskell will significantly aid you in writing Copilot specifications;  a variety of books and free web resources introduce Haskell.
Copilot uses Haskell language extensions
specific to the Glasgow Haskell Compiler (GHC).

\subsection{Target Application Domain} \label{domain}


Copilot is a domain-specific language tailored to programming \emph{runtime
monitors} for \emph{hard real-time}, \emph{distributed}, \emph{reactive systems}.
Briefly, a runtime monitor is program that runs concurrently with a target program
with the sole purpose of assuring that the target program behaves in accordance with a
pre-established specification. Copilot is a language for writing such specifications.

A reactive system is a system that responds continuously to its environment.
All data to and from a reactive system are communicated progressively during
execution. Reactive systems differ from transformational systems which transform
data in a single pass and then terminate, as for example compilers and numerical
computation software.

A hard real-time system is a system that has a statically bounded execution time
and memory usage.  Typically, hard real-time systems are used in
mission-critical software, such as avionics, medical equipment, and nuclear power
plants; hence, occasional dropouts in the response time or crashes are not
tolerated.

A distributed system is a system which is layered out on multiple pieces of hardware.
The distributed systems we consider are all synchronized, i.e., all components agree on
a shared global clock.


\subsection{Installation} \label{sec:install}

Before downloading the copilot framework, you must first install an
up-to-date version of GHC (the minimal required version is 8.0).
The easiest way to do this is to download and install the Haskell Platform,
which is freely distributed from here:

\begin{center}
\url{http://hackage.haskell.org/platform}
\end{center}

\noindent Because Copilot compiles to C code, you must also install a C compiler such as GCC (\url{https://gcc.gnu.org/install/}). After having installed the Haskell Platform and C compiler, Copilot can be downloaded and
installed in the following two ways:

\begin{itemize}
\item \textbf{Hackage (Prefered Method): } Copilot is available from Hackage,
and the latest version can be installed easily:
\begin{code}
cabal install copilot
\end{code}

\item \textbf{From source: } Alternatively one can download and install it from
the repositories as well. Typically one would only go this route to develop
Copilot. For regular users the hackage method above is highly recommended.
\begin{code}
git clone https://github.com/Copilot-Language/copilot.git
cd copilot
git submodule update --init --remote
\end{code}
Compiling can either be done in a Nix-style build, or a traditional one:

\noindent\emph{Nix-Style build (Cabal $\ge$ 2.x):}
\begin{code}
cabal build       # For Cabal 3.x
cabal v2-build    # For Cabal 2.x
\end{code}

\noindent\emph{Traditional build (Cabal 1.x):}
\begin{code}
cabal install --dependencies-only
cabal build
\end{code}
\end{itemize}


\subsection{Structure} \label{structure}

\noindent Copilot is distributed through a series of packages at Hackage:

\begin{itemize}
\item copilot-language: Contains the language front-end.
\item copilot-theorem: Contains extensions to the language for proving
properties about Copilot programs using various SMT solvers and model checkers.
\item copilot-core: Contains an intermediate representation for Copilot programs.
\item copilot-c99: A back-end for Copilot targeting C99.
\item copilot-libraries: A set of utility functions for Copilot, including a
clock-library, a linear temporal logic framework, a voting library, and a regular
expression framework.
\end{itemize}

Many of the examples in this paper can be found at
\url{https://github.com/Copilot-Language/Copilot/tree/master/Examples}.

\subsection{Sampling} \label{sampling}
The idea of sampling representative data from a large set of data  is well
established in data science and engineering. For instance, in digital
signal processing, a signal such as music is sampled at a high enough
rate to obtain enough discrete points to represent the physical sound
wave. The fidelity of the recording is dependant on the sampling
rate. Sampling a state variable of an executing program is similar,
but variables are rarely continuous signals so they lack the nice
properties of continuity.
Monitoring based on sampling state-variables has historically been disregarded
as a runtime monitoring approach, for good reason: without the assumption of
synchrony between the monitor and observed software, monitoring via sampling may
lead to false positives and false negatives~\cite{DwyerDE08}.  For example,
consider the property $(0;1;1)^*$, written as a regular expression, denoting the
sequence of values a monitored variable may take.  If the monitor samples the
variable at the inappropriate time, then both false negatives (the monitor
erroneously rejects the sequence of values) and false positives (the monitor
erroneously accepts the sequence) are possible.  For example, if the actual
sequence of values is $0,1,1,0,1,1$, then an observation of $0,1,1,1,1$ is a
false negative by skipping a value, and if the actual sequence is $0,1,0,1,1$,
then an observation of $0,1,1,0,1,1$ is a false positive by sampling a value
twice.



However, in a hard real-time context, sampling is a suitable strategy.  Often,
the purpose of real-time programs is to deliver output signals at a predicable
rate and properties of interest are generally data-flow oriented.  In this
context, and under the assumption that the monitor and the observed program
share a global clock and a static periodic schedule, while false positives are
possible, false negatives are not.  A false positive is possible, for example,
if the program does not execute according to its schedule but just happens to
have the expected values when sampled.  If a monitor samples an unacceptable
sequence of values, then either the program is in error, the monitor is in
error, or they are not synchronized, all of which are faults to be reported.

Most of the popular runtime monitoring frameworks inline monitors in
the observed program to avoid the aforementioned problems with
sampling.  However, inlining monitors changes the real-time behavior
of the observed program, perhaps in unpredicable ways.
Solutions that
introduce such unpredictability are not a viable solution for
ultra-critical hard real-time systems.  In a sampling-based approach,
the monitor can be integrated as a separate scheduled process during
available time slices (this is made possible by generating efficient
constant-time monitors).  Indeed, sampling-based monitors may even be
scheduled on a separate processor (albeit doing so requires additional
synchronization mechanisms), ensuring time and space partitioning from
the observed programs.  Such an architecture may even be necessary if
the monitored program is physically distributed.

When deciding whether to use Copilot to monitor systems that are not hard real-time, the user must determine if
sampling is suitable to capture the errors and faults of interest in the  SUO. In many cyber-physical systems, the trace being monitored is
coming from sensors measuring physical data such as GPS coordinates, air speed, and actuation signals. These continuous signals do not
change abruptly so as long as it is sampled at a suitable rate, that usually must be determined experimentally, sampling is sufficient.



