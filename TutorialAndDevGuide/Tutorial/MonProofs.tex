\section{Proofs of Monitors}~\label{sec:proof}

In addition to proofs on the soundness of the Copilot compilation process, we
also support (in the {\tt Copilot.Theorem} module) some automated proving of
safety properties about Copilot specifications themselves. A proposition is a
Copilot value of type \lstinline{Prop Existential} or \lstinline{Prop Universal},
 which can be introduced using \lstinline{exists} and
\lstinline{forall}, respectively. These are functions taking as an argument a
normal Copilot stream of type \lstinline{Stream Bool}. Propositions can be added
to a specification using the \lstinline{prop} and \lstinline{theorem} functions,
where \lstinline{theorem} must also be passed a tactic for autmatically proving
the proposition. Consider this Copilot monitor specification for a version of
the fibonacci sequence:

\begin{lstlisting}[language = Copilot]
module Fib where

import Prelude ()
import Copilot.Language
import Copilot.Language.Reify
import Copilot.Theorem
import Copilot.Theorem.Prover.SMT

fib = do
  theorem "fibn_gre_n" (forall $ fibn >= n) $ kInduction def cvc4

  observer "fibn" fibn
  observer "n"    n
  where
    fibn :: Stream Word32
    fibn = [1, 1] ++ (fibn + drop 1 fibn)
    n = [0] ++ (n + 1)
\end{lstlisting}

In the specification above, in addition to observing the value of streams in
the specification, \lstinline{fibn_gre_n} names a theorem provable
automatically with induction using the Z3 SMT solver. This theorem can be
checked during reification:

\begin{code}
*Fib> reify fib
fibn_gre_n: valid (proved with k-induction (k = 3))
Finished: fibn_gre_n: proof checked successfully
\end{code}

The {\tt Copilot.Theorem} module provides two main mechanisms for interacting
with SMT solver backends: a generic backend at {\tt Copilot.Theorem.Prover.SMT}
that nominally supports a wide range of SMT solvers and {\tt
Copilot.Theorem.Prover.Z3}, a backend specialized to Z3. The example above uses
the generic {\tt SMT} backend with the CVC4 SMT solver.

See Figure~\ref{fig:solvers} for an overview of SMT solvers that have at least
been tested to work with our tool. The second column contains the value of type
{\tt Backend} from the {\tt Copilot.Theorem.Prover.SMT} module which must be
passed to the tactics in this module for executing the tactic using that SMT
solver (e.g., as is done in the second argument to {\tt kInduction} in the
example above). To use a certain SMT solver, the solver must be installed and
the executable should reside in one of the directories listed in the {\tt
\$PATH} environment variable.

\newcommand{\Yes}{\checkmark}
\newcommand{\No}{\textsf{X}}
\newcommand{\Some}{\textsf{Some}}

\begin{figure}
\begin{center}
\begin{tabular}{llllcccc}
Tool       & {\tt Backend} & Version & Interface & NL Real Arith.\ & Trig.\ funs.\ & Quantif. & Bitvec. \\
\toprule
Alt-Ergo   & {\tt altErgo} & 0.99.1  & SMTLib2   & \Yes{}          & \No{}         & \Yes{}      & \No{}      \\
CVC4       & {\tt cvc4}    & 1.4     & SMTLib2   & \Some{}         & \No{}         & \Yes{}      & \Yes{}     \\
DReal      & {\tt dReal}   & 2.15.01 & SMTLib2   & \Yes{}          & \Yes{}        & \No{}       & \No{}      \\
MathSAT    & {\tt mathSat} & 5.3.7   & SMTLib2   & \Some{}         & \No{}         & \No{}       & \Yes{}     \\
MetiTarski & {\tt metit}   & 2.5     & TPTP      & \Yes{}          & \Yes{}        & \Yes{}      & \No{}      \\
Yices      & {\tt yices}   & 2.4.0   & SMTLib2   & \Some{}         & \No{}         & \No{}       & \Yes{}     \\
Z3         & {\tt z3}      & 4.4.0   & SMTLib2   & \Yes{}          & \No{}         & \Yes{}      & \Yes{}     \\
\end{tabular}
\end{center}
\caption{An overview of some supported SMT solvers.}
\label{fig:solvers}
\end{figure}

As seen in the figure mentioned above, different SMT solvers have different
feature sets and the methods for using the various features are generally not
well standardized. As a result, we also have backend specialized to Z3, one of
the more feature-rich SMT solvers we've tested. The example above using the Z3
backend looks very similar to the version above:

\begin{lstlisting}[language = Copilot]
module Fib where

import Prelude ()
import Copilot.Language
import Copilot.Language.Reify
import Copilot.Theorem
import Copilot.Theorem.Prover.Z3

fib = do
  theorem "fibn_gre_n" (forall $ fibn >= n) $ kInduction def

  observer "fibn" fibn
  observer "n"    n
  where
    fibn :: Stream Word32
    fibn = [1, 1] ++ (fibn + drop 1 fibn)
    n = [0] ++ (n + 1)
\end{lstlisting}

Instead of importing {\tt Copilot.Theorem.Prover.SMT}, we import {\tt
Copilot.Theorem.Prover.Z3}, and we no longer need to pass the {\tt cvc4}
argument to {\tt kInduction} (because the SMT solver backend will be Z3). But
now when we go to reify {\tt fib}:

\begin{code}
*Fib> reify fib
fibn_gre_n: unknown (proof by k-induction failed)
Finished: fibn_gre_n: proof failed
Warning: failed to check some proofs.
\end{code}

This demonstrates an important limitation of the {\tt
Copilot.Theorem.Prover.SMT} backend: it encodes fixed-width integers using the
SMTLib {\tt Int} type (with some extra propositions about the bounds of {\tt
Int} variables)---it {\em does not model overflow}. The {\tt
Copilot.Theorem.Prover.Z3} backend, however, encodes Copilot's fixed-width
integer types using Z3's bitvectors. Now if we allow {\tt fibn} to overflow, the
theorem from the above example is clearly not true.

The above example contains a single {\tt theorem}. A {\tt theorem} takes three
arguments: the name of the theorem (which is only used to identify the theorem
in output), a proposition, and a proof ``tactic'' to use when trying to
automatically prove the proposition using SMT solvers. The {\tt
Copilot.Theorem.Prover.SMT} and {\tt Copilot.Theorem.Prover.Z3} modules both
export the same set of tactics\footnote{In fact, these modules share a lot a
duplicated code and desparately need to be refactored.}:
\begin{itemize}
\item {\tt onlySat}: check that the proposition is satisfiable.
\item {\tt onlyValidity}: check that the negation of the proposition is
unsatisfiable.
\item {\tt induction}: special case of $k$-induction, with $k = 0$.
\item {\tt kInduction}: version of induction where the induction hypothesis is
strengthened by including $k$ previous states.
\end{itemize}

Additionally, a few other tactics live in {\tt Copilot.Theorem.Tactics}:
\begin{itemize}
\item {\tt instantiate}: turn a proof for a universally quantified proposition
into a proof for an existentially quantified one.
\item {\tt assume}: prove a proposition true under an assumption.
\item {\tt admit}: prove anything.
\end{itemize}

Also, we support some older versions of the Kind2 model checker.
