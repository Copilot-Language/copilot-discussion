\section{Complete example}
\label{sec:complete_example}
This section describes a complete use case example in Copilot. We will be using
one of the provided examples as our code, and focus more on using Copilot
within a project.

\subsection{C Code}
Lets start of with the C program our monitor to connect to.
\begin{lstlisting}[language=c, numbers=left]
#include <stdlib.h>
#include <stdint.h>

#include "heater.h" /* Generated by our specification */

int8_t temperature;

void heaton (float temp) {
  /* Low-level code to turn heating on */
}

void heatoff (float temp) {
  /* Low-level code to turn heating off */
}

int main (int argc, char *argv[]) {
  for (;;) {
    temperature = readbyte(); /* Dummy function to read a byte from a sensor. */

    step();
  }

  return 0;
}
\end{lstlisting}

For this code we left out the low-level details for interfacing with our
hardware. Let us look at a couple of interesting lines:

\begin{description}
  \item[Line 4] Here we include the header file generated by our Copilot
  specification (see next section).
  \item[Line 8] Global variable that stores the raw output of the temperature
  sensor. This variable should be global, so it can be read from the code
  generate from our monitor.
  \item[Line 8-14] Functions that turn on and turn off the
  heater, low-level details are provided.
  \item[Line 17-21] Our infinite main-loop:
    \begin{description}
      \item[Line 18] Update our global temperature variable by reading it from
      the sensor.
      \item[Line 20] Execute a single evaluation step of Copilot.
      \texttt{step()} is imported from the \texttt{heater.h}, and is the only
      publicly available function from the specification.
    \end{description}
\end{description}

As the code shows, the rate at which Copilot is updated is entirely up to the
programmer of the main C program. In this case it is updated as quick as
possible, but we could have opted to slow it down with a delay or a scheduler.
Theoretically there could be multiple calls to \texttt{step()} throughout the
program, but this complicated things and is highly discouraged.


