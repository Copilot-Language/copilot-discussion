\newpage 
\section{Interpreting and Compiling}
\label{interpcompile}
The Copilot RV framework comes with both an interpreter and a
compiler. 
\subsection{Interpreting Copilot}
Assume we are currently in a directory containing a \texttt{.hs} file with our
specification (\texttt{Spec.hs} in this case), and that Copilot is installed
globally. If we want to interpret the specification, we need to start the GHC
Interpreter with the file as an argument:
%
\begin{lstlisting}
$ ghci Spec.hs
GHCi, version 8.4.3: http://www.haskell.org/ghc/  :? for help
Loaded GHCi configuration from /home/user/.ghc/ghci.conf
[1 of 1] Compiling Spec             ( Spec.hs, interpreted )
Ok, one module loaded.
*Spec > 
\end{lstlisting}
%
This launches \texttt{ghci}, the Haskell interpreter, with \texttt{Spec.hs}
loaded. It provides us with a prompt, recognisable by the \texttt{>} sign. Lets
assume that our file contains one specification, called \texttt{spec}. We can
interpret this using the \texttt{interpret}-function:
\begin{lstlisting}[language = Copilot]
*Spec > interpret 10 spec
\end{lstlisting}
%
The first argument to the function \emph{interpret} is the number of iterations
that we want to evaluate. The second argument is the specification (of type
{\tt Spec}) that we wish to interpret.

The interpreter outputs the values of the arguments passed to the trigger, if
its guard is true, and {\tt --} otherwise. We will discuss triggers in more
detail later, but for now, just know that they produce an output only when the
guard function is true. For example, consider the following Copilot program:
%
\begin{lstlisting}[language = Copilot]
spec = do
  trigger "trigger1" (even nats) [arg nats, arg $ odd nats]
  trigger "trigger2" (odd nats) [arg nats]
\end{lstlisting}
% $
where {\tt nats} is the stream of natural numbers, and {\tt even} and {\tt odd}
are the guard functions that take a stream and return whether the point-wise
values are even or odd, respectively. The lists at the end of the trigger
represent the values the trigger will output when the guard is true. The output
of
%
\begin{lstlisting}[language = Copilot]
interpret 10 spec
\end{lstlisting}
% $
is as follows:
%
\begin{code}
trigger1:   trigger2:
(0,false)  --
--         (1)
(2,false)  --
--         (3)
(4,false)  --
--         (5)
(6,false)  --
--         (7)
(8,false)  --
--         (9)
\end{code}
%

Note that trigger1 outputs both the number and whether that number is odd,
while trigger2 only outputs the number. This output reflects the arguments
	passed to them. 

\subsection{Compiling Copilot} \label{sec:compiling}

Compiling a Copilot specification is straightforward. Currently Copilot
supports one back-end, \texttt{copilot-c99} that creates constant-space C99
code. Using the back-end is rather easy, as it just requires one to import it in
their Copilot specification file:

\begin{lstlisting}[language = Copilot]
import Copilot.Compile.C99
\end{lstlisting}

Importing the back-end provides us with the \texttt{compile}-function, which
takes a prefix as its first parameter and a \textit{reified} specification as
its second. When inside \texttt{ghci}, with our file loaded, we can generate
output code by executing:
\footnote{Two explanations are in order: (1) {\tt reify} allows sharing in the
expressions to be compiled~\cite{DSLExtract}, and {\tt >>=} is a higher-order
operator that takes the result of reification and ``feeds'' it to the compile
function.}
\begin{lstlisting}[language = Copilot]
reify spec >>= compile "monitor"
\end{lstlisting}

This generates two output files:
\begin{itemize}
  \item \texttt{<prefix>.c}: C99 file containing the generated code and the
  \texttt{step()} function. This should be compiled by the C compiler, and
  included in the final binary.
  \item \texttt{<prefix>.h}: Header providing the public interface to the
  monitor. This file should be included from your main project.
\end{itemize}

Please refer to the complete example \ref{sec:complete_example} for more on
detail to use the monitor in your C program.
