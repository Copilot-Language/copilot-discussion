\newpage 
\section{Interpreting}
\label{interpcompile}
The Copilot RV framework comes with both an interpreter and a
compiler. We will address compiling in section 4 with a complete example. 
\noindent To use the language, your Haskell module should contain the following import:
%
\begin{code}
import Language.Copilot
\end{code}
%
If you need to use functions defined in the Prelude that are redefined by
Copilot (e.g., arithmetic operators), import the Prelude qualified:
%
\begin{code}
import qualified Prelude as P
\end{code}

\subsection{Interpreting Copilot}
In the ./Examples directory we have provided you with an example
for you to follow allong. \texttt{Spec.hs} is the following Copilot program:
%
\lstinputlisting[language = Copilot, numbers = left]{Examples/Spec.hs}
%import Language.Copilot hiding (even, odd) 
%
%import qualified Prelude as P hiding ((++),(==), mod,  even, odd)
%
%nats :: Stream Int32
%nats = [0] ++ (1 + nats)
%
%even :: Stream Int32 -> Stream Bool
%even n = n `mod` 2 == 0
%
%odd :: Stream Int32 -> Stream Bool
%odd n2 = n2 `mod` 2 == 1
%
%spec = do
%  trigger "trigger1" (even nats) [arg nats, arg $ odd nats]
%  trigger "trigger2" (odd nats) [arg nats]
%\end{lstlisting}
% $
\begin{description}
  \item[Line 1] Here we inclued the Copilot Language so that we gain access to the
  front end language.
  \item[Line 3] Here we include the Prelude so that we can hide base Haskell syntax
  that we have redefined. If this is not included you will get an \texttt{Ambiguious
  Occurance} error. 
  \item[Line 5-12] Here we define data streams as input and output data streams. We
  go over defining functions as streams in Section 3 of this tutorial. 
  \item[Line 14-16] Here {\tt nats} is the stream of natural numbers, and {\tt even} and {\tt odd}
  are the guard functions that take a stream and return whether the point-wise
  values are even or odd, respectively. The lists at the end of the trigger
  represent the values the trigger will output when the guard is true.

\end{description}

If we want to interpret the specification, we need to start the GHC Interpreter with the file as an argument:
%
\begin{lstlisting}
$ ghci Spec.hs
[1 of 1] Compiling Spec             ( Spec.hs, interpreted )
Ok, one module loaded.
ghci > 
\end{lstlisting}
%
This launches \texttt{ghci}, the Haskell interpreter, with \texttt{Spec.hs}
loaded. It provides us with a prompt, recognisable by the \texttt{>} sign. Lets
assume that our file contains one specification, called \texttt{spec}. We can
interpret this using the \texttt{interpret}-function:
\begin{lstlisting}[language = Copilot]
ghci > interpret 10 spec
\end{lstlisting}
%
The first argument to the function \emph{interpret} is the number of iterations
that we want to evaluate. The second argument is the specification (of type
{\tt Spec}) that we wish to interpret.

The interpreter outputs the values of the arguments passed to the trigger, if
its guard is true, and {\tt --} otherwise. We will discuss triggers in more
detail later, but for now, just know that they produce an output only when the
guard function is true. The output is as follows:
%
\begin{code}
trigger1:   trigger2:
(0,false)  --
--         (1)
(2,false)  --
--         (3)
(4,false)  --
--         (5)
(6,false)  --
--         (7)
(8,false)  --
--         (9)
\end{code}
%

Note that trigger1 outputs both the number and whether that number is odd,
while trigger2 only outputs the number. This output reflects the arguments
	passed to them. 


