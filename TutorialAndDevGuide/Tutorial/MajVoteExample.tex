\subsection{The Boyer-Moore Majority-Vote Algorithm}
\label{subsec:boyer_moore}

In this section we demonstrate how to use Haskell as an advanced macro language
on top of Copilot by implementing an algorithm for solving the voting problem
in Copilot.

Reliability in mission critical software is often improved by replicating
the same computations on separate hardware and by doing a vote in the end
based on the output of each system. The majority vote problem consists of
determining if in a given list of votes there is a candidate that has more
than half of the votes, and if so, of finding this candidate.

The Boyer-Moore Majority Vote Algorithm \cite{MooreBoyer82,Hesselink2005} solves
the problem in linear time and constant memory. It does so in two passes: The
first pass chooses a candidate; and the second pass asserts that the
found candidate indeed holds a majority.

The algorithm for the first pass involves the sequence of elements we are interpreting,
a single element that represents the current majority, and a counter,
which is initially set to zero. The algorithm is as follows:
\begin{itemize}
\item Initialize an element $m$ and a counter $i$ where $i=0$
\item For each element x of the input sequence: 
	\begin{itemize} 
	\item If $i=0$ then let $m=x$ and $i=1$
	\item else if $m=x$ then increment $i$
	\item else let $i = i-1$
	\end{itemize}
\item Return $m$
\end{itemize}

This algorithm will produce an output even if there is no majority, which is why 
the second pass is needed to verify that the output of the first pass is valid. 


\lstinputlisting[language = Copilot, numbers = left]{Examples/MajVoteExample.hs}

The first pass can be implemented
in Haskell as shown in lines 2-13. The second pass, which
simply checks that a candidate has more than half of the votes, is
straightforward to implement and is shown in lines 15-23.
E.g. applying {\tt majorityPure} on the string {\tt AAACCBBCCCBCC} yields {\tt
  C}, which {\tt aMajorityPure} can confirm is in fact a majority.

\begin{figure*}[!htb]
\begin{lstlisting}[language = Copilot, frame = none]
majority :: (Eq a, Typed a) => [Stream a] -> Stream a
majority []     = error "majority: empty list!"
majority (x:xs) = majority' xs x 1

majority' []     can _   = can
majority' (x:xs) can cnt =
  local
    (if cnt == 0 then x else can) $
      \ can' ->
        local (if cnt == 0 || x == can then cnt+1 else cnt-1) $
          \ cnt' ->
            majority' xs can' cnt'
\end{lstlisting}
\caption{The first pass of the majority vote algorithm in Copilot.}
\label{fig:majority}
\end{figure*}

\begin{figure*}[!htb]
\begin{lstlisting}[language = Copilot, frame = none]
aMajority :: (Eq a, Typed a) => [Stream a] -> Stream a -> Stream Bool
aMajority xs can = aMajority' 0 xs can > (fromIntegral (length xs) `div` 2)

aMajority' cnt []     _   = cnt
aMajority' cnt (x:xs) can =
  local
    (if x == can then cnt+1 else cnt) $
      \ cnt' ->
        aMajority' cnt' xs can
\end{lstlisting}
\caption{The second pass of the majority vote algorithm in Copilot.}
\label{fig:amajority}
\end{figure*}
% $

When implementing the majority vote algorithm for Copilot, we can use reuse
almost all of the code from the Haskell implementation. However, as functions
in Copilot are macros that are expanded at compile time, care must
be taken in order to avoid an explosion in the code size. Hence, instead of
using Haskell's built-in \emph{let}-blocks, we use explicit sharing, as
described in Section~\ref{sec:explicit_sharing}. The Copilot implementations
of the first and the second pass are given in Figure \ref{fig:majority} and
Figure \ref{fig:amajority} respectively. Comparing the Haskell implementation
with the Copilot implementation, we see that the code is almost identical,
except for the type signatures and the explicit sharing annotations.



