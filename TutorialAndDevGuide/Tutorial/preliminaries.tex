

\section{Preliminaries}
\label{sec:preliminaries}

Copilot is embedded into the functional programming language Haskell
\cite{PeytonJones02}.  A working knowledge of Haskell is necessary to use
Copilot effectively; a variety of books and free web resources introduce Haskell.
Copilot uses Haskell language extensions
specific to the Glasgow Haskell Compiler (GHC); hence in order to start
using Copilot, you must first install an up-to-date version of GHC.
(The minimal required version is 7.10.1)
The easiest way to do this is to download and install the Haskell Platform,
which is freely distributed from here:

\begin{center}
\url{http://hackage.haskell.org/platform}
\end{center}

\noindent After having installed the Haskell Platform, Copilot is downloaded and
installed by executing the following command:

\begin{code}
> cabal install copilot
\end{code}

\noindent This should, if everything goes well, install Copilot on your system.

Copilot is distributed throughout a series of packages at Hackage:

\begin{itemize}
\item copilot-language: Contains the language front-end.
\item copilot-core: Contains an intermediate representation for Copilot programs (shared by all back-ends).
\item copilot-c99: A back-end for Copilot targeting C99 (based on Atom, \url{http://hackage.haskell.org/package/atom}). \textbf{Not updated anymore, might be deprecated soon.}
\item copilot-sbv: A back-end for Copilot targeting C99 (based on SBV, \url{http://hackage.haskell.org/package/sbv}).
\item copilot-libraries: A set of utility functions for Copilot, including a clock-library, a linear temporal logic framework,
a voting library, and a regular expression framework.
\item copilot-cbmc: A driver for proving the correspondence between code
  generated by the copilot-c99 and copilot-sbv back-ends.
\end{itemize}

Many of the examples in this paper can be found at
\url{https://github.com/Copilot-Language/Copilot/tree/copilot2.0/Examples}.
\todo[inline]{Update the version of Copilot when released (3.0 ?), and do not forget to put some examplesForACSL not already in the directory (WCV, TCAS, Self-Separation).}

To use the language, your Haskell module should contain the following import:
%
\begin{code}
import Language.Copilot  
\end{code}
%
To use the back-ends, import, them, respectively:
%
\begin{code}
import Copilot.Compile.C99
import Copilot.Compile.SBV
\end{code}
%
If you need to use functions defined in the Prelude that are redefined by
Copilot (e.g., arithmetic operators), import the Prelude as qualified:
%
\begin{code}
import qualified Prelude as P  
\end{code}

