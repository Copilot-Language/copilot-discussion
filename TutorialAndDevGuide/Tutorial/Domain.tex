\section{Domain}

Copilot is a domain-specific language tailored to programming \emph{runtime
monitors} for \emph{hard real-time}, \emph{distributed}, \emph{reactive systems}.
Briefly, a runtime monitor is program that runs concurrently with a target program
with the sole purpose of assuring that the target program behaves in accordance with a
pre-established specification. Copilot is a language for writing such specifications.

A reactive system is a system that responds continuously to its environment.
All data to and from a reactive system is communicated progressively during
execution. Reactive systems differ from transformational systems which transforms
data in a single pass and then terminate, as for example compilers and numerical
computation software.

A hard real-time system is a system that has a statically bounded execution time
and memmory usage.  Typically, hard real-time systems are used in
mission-critical software, such as avionics, medical equipment, and nuclear power
plants; hence, occasional dropouts in the response time or crashes are not
tolerated.

A distributed system is a system which is layered out on multiple pieces of hardware.
The distributed systems we consider are all synchronized, i.e., each component agree on
a shared global clock.
