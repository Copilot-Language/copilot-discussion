\documentclass{beamer}
\usepackage[english]{babel}
\usepackage{listings}
\usepackage{tikz}
\usetikzlibrary{arrows,automata}

\lstset {
    language=haskell
  , frame=tb
  , moredelim=**[is][\only<-+>{\color{white}}%
                     \only<.(1)>{\color{red}}%
                     \only<.(2)>{\color{black}}]{\#}{\#}%
}

\title{A new C backend for Copilot}

\author{Frank Dedden\inst{1,2} \and Alwyn Goodloe\inst{3} \and Wouter Swierstra\inst{1}}

\institute
{
  \inst{1}
  Utrecht University\\
  Department of Information and Computing Sciences\\
  \emph{Utrecht, The Netherlands}
  \and
  \inst{2}
  National Institute of Aerospace\\
  \emph{Hampton, Virginia, United States of America}
  \and
  \inst{3}
  NASA Langley Research Center\\
  Safety Critical Avionics Systems Branch\\
  \emph{Hampton, Virginia, United States of America}
}

\begin{document}

\begin{frame}
  \titlepage
\end{frame}

\begin{frame}{Outline}
  \tableofcontents
\end{frame}



\section{Introduction}
\begin{frame}{Introduction}
\begin{itemize}
  \item Copilot is a Haskell library for monitoring C programs.
  \item Implements a Domain Specific Language to write specifications.
  \item Currently comes with two backends for translation to C:
  \begin{itemize}
    \item \emph{Copilot-C99}: Based on Atom, unmaintained.
    \item \emph{Copilot-SBV}: SBV based code generator.
  \end{itemize}
  \item Task: to create a new backend, replacing both current ones.
\end{itemize}
\end{frame}


\section{Copilot overview}
\begin{frame}{Copilot overview}
\begin{figure}[ht!]
  \centering
  \input{toolchain.tex}
  \caption{The Copilot toolchain}
  \label{fig:copilot_toolchain}
\end{figure}
\end{frame}

\begin{frame}[fragile]{Small example}
\begin{lstlisting}
module Main where

import Language.Copilot
import Copilot.Compile.SBV
import qualified Prelude as P

a :: Stream Int8
a = [1,2,3] ++ a

b :: Stream Int8
b = [5,6] ++ a

spec = do
  trigger "func" (a > 2) [arg b]

main = do reify spec >>= compile defaultParams
\end{lstlisting}
\end{frame}


\begin{frame}{Limitations}
\begin{itemize}
  \item \emph{Copilot-C99} is unmaintained, and produces ugly code.
  \item \emph{Copilot-SBV} produces better code, but can be hard to read due to
  formatting.
  \item Neither support structs or arrays.
  \item Both could be more efficient.
  \item \emph{Copilot-SBV} produces ACSL specifications as well,
  \emph{Copilot-C99} does not.
\end{itemize}
\end{frame}



\section{Code generator}
\begin{frame}{Code generator}
\begin{itemize}
  \item Outputs code nearly identical to \emph{Copilot-SBV}.
  \item Every stream has a buffer, current value and current index in the buffer.
  \item Every stream gets its own generator function.
  \item Guards and arguments of trigger have their own function as well.
  \item These functions are called in the \texttt{step()} function.
\end{itemize}
\end{frame}

\begin{frame}[fragile]
\begin{lstlisting}
module Main where

import Language.Copilot
import Copilot.Compile.C

import Prelude hiding ((++), drop)

s0 :: Stream Int8
s0 = [1,2,3] ++ s0

spec = do
  trigger "fbasic" true [arg s0]

main = do reify spec >>= compile
\end{lstlisting}
\end{frame}


\begin{frame}[fragile]
\begin{lstlisting}[basicstyle=\small, language=c]
static int8_t s0_buff[3] = {1, 2, 3};
static int8_t s0 = 1;
static size_t s0_idx = 0;
static int8_t s0_gen () {
  int8_t s0_loc;
  {
    size_t idx = s0_idx;
    s0_loc = s0_buff[idx];
  };
  return s0_loc;
}
static bool fbasic_guard () {
  return true;
}
/* fbasic_arg0 should be here */
static void step () {
  if (fbasic_guard()) fbasic(fbasic_arg0());
  s0 = s0_gen();
  s0_buff[s0_idx] = s0;
  ++(s0_idx);
  s0_idx = s0_idx % 3;
}
\end{lstlisting}
\end{frame}


\section{Structs}
\begin{frame}{Structs}
  \begin{itemize}
    \item Bundled values in a single datatype.
    \item Possibility to read and modify fields.
    \item We need to be able to write functions on arbitrary structs.
  \end{itemize}

  Multiple ways to model structs:
    \begin{enumerate}
      \item Separate streams for every field.
      \item List of tuples: \texttt{[(String, Type)]}.
      \item Using Haskell's record syntax.
    \end{enumerate}
\end{frame}



\begin{frame}[fragile]{Proposed solution: Separate streams}
  \begin{itemize}
    \item Current way to use structs in Copilot.
    \item Very easy to implement, requiring no changes to Copilot.
    \item Tedious to use.
  \end{itemize}
\begin{lstlisting}[language=c]
struct vec {
  float x;
  float y;
};
vec v = {1, 2};
\end{lstlisting}
\begin{lstlisting}
vx, vy :: Float
vx = extern "v.x" Nothing
vy = extern "v.y" Nothing
\end{lstlisting}
\end{frame}

\begin{frame}[fragile]
  However, there is no syntactical connection between fields:
\begin{lstlisting}
length :: Float -> Float -> Float
length x y = sqrt (x*x + y*y)
\end{lstlisting}

  We could solve this with a tuple:
\begin{lstlisting}
type Vec = (Float, Float)
length :: Vec -> Float
length (x, y) = sqrt (x*x + y*y)
\end{lstlisting}

  But we need to pack and unpack the tuple by hand every time...
\end{frame}



\begin{frame}[fragile]{Proposed solution: List of tuples}
  \begin{itemize}
    \item Easy to implement, fields can be accessed by their names.
    \item Needs a \texttt{lookup} function to access fields.
    \item Does create a syntactical connection between fields.
    \item We need a heterogeneous list to store the values.
  \end{itemize}
\begin{lstlisting}
type VarName = String
data Type    = Bool | Int8 | Int16 | ...

data Value   = forall a. V Type a
type Struct  = [(VarName, Value)]

v :: Struct
v = extern "v" Nothing
\end{lstlisting}
\end{frame}

\begin{frame}[fragile]
  The definition of \texttt{Value} does allow incorrect values:
\begin{lstlisting}
v :: Value
v = V Int8 True
\end{lstlisting}
  Adding a parameter to \texttt{Type} can solve the problem. Copilot already
  provides such a type:
\begin{lstlisting}
data Type :: * -> * where
  Bool    :: Type Bool
  Int8    :: Type Int8
  Int16   :: Type Int16
  ...

data Value  = forall a. V (Type a) a
type Struct = [(VarName, Value)]
\end{lstlisting}
\end{frame}

\begin{frame}[fragile]
Note that the \texttt{Type a} and \texttt{a} are bound by the same
quantifier, which disallows incorrect values:
\begin{lstlisting}
v :: Value
v = V Int8 True
\end{lstlisting}

\begin{lstlisting}[basicstyle=\scriptsize,language={}]
* Couldn't match expected type `Int8' with actual type `Bool'
* In the second argument of `V', namely `True'
  In the expression: V Int8 True
  In an equation for `v': v = V Int8 True
\end{lstlisting}
\end{frame}


\begin{frame}[fragile]
  We are unable to write incorrect \texttt{Value}'s, but the type of
  \texttt{Struct} provides us with little to no type-safety:

\begin{lstlisting}
v1 :: Struct
v1 = [ ("x", V Float 1.0) ]

v2 :: Struct
v2 = [ ("foo", V Bool True) ]
\end{lstlisting}
Here \texttt{v1} and \texttt{v2} share the same type, but share no semantic
meaning. Even worse, we can modify the fields of a struct:
\begin{lstlisting}
faulty :: Struct -> Struct
faulty v = ("bar", V Bool False) : v

v3 :: Struct
v3 = faulty v1 -- v3 is not a vector anymore
\end{lstlisting}
\end{frame}



\begin{frame}[fragile]{Final solution: Haskell datatypes}
  \begin{itemize}
    \item Haskell's record syntax emulates C structs.
    \item Gives us accessor functions for free.
    \item We can rely on Haskell to force type correctness.
  \end{itemize}
\begin{lstlisting}
data Vec = Vec
  { x :: Float
  , y :: Float
  }

vec :: Vec
vec = Vec 1 2

x' :: Float
x' = x vec
\end{lstlisting}
\end{frame}

\begin{frame}[fragile]
  Obviously, having distinct types disallows writing functions that work on all
  possible structs $\Rightarrow$ Use a \texttt{Struct} class:

\begin{lstlisting}
class #Typed a =># Struct a #where
  typename    :: a -> Typename a
  toValues    :: a -> Values a
  fromValues  :: Values a -> a#

#data Typename a = TyTypedef String
                | TyStruct  String#

#type Values a = [Value]
data Value    = forall a.
  (Typed a, Show a) => V (Type a) String a#

instance Struct Vec #where ...

instance Typed Vec where ...#
\end{lstlisting}
\end{frame}

\begin{frame}[fragile]
Both instances of \texttt{Struct} and \texttt{Typed} need to be defined by the
user.
\begin{lstlisting}
instance Struct Vec where
  typename _  = TyTypedef "vec"
  #toValues v  = [ V Int8  "x" (x v)
                , V Int16 "y" (y v)
                ]#
  #fromValues  ( V Int8  "x" x
              : V Int16 "x" y
              : []
              ) = Vec x y#
\end{lstlisting}
\pause
\begin{itemize}
  \item User must ensure result of \texttt{toValues} is of the correct length.
  \item We pattern match in \texttt{fromValues}, types are matched runtime.
\end{itemize}
\end{frame}

\begin{frame}[fragile]
\begin{lstlisting}
instance Typed Vec where
  -- typeOf :: Type a
  typeOf = Struct (Vec 0 0)
\end{lstlisting}
\begin{itemize}
  \item \texttt{Vec 0 0} is used to infer a correct type, the values itself are
  ignored.
\end{itemize}
\end{frame}


\begin{frame}[fragile]{Struct example}
With structs being first-class in Copilot, we can use the regular
\texttt{extern} function on them:
\begin{lstlisting}
exvec :: Stream Vec
exvec = extern "exvec" Nothing

s :: Stream Vec
s = [Vec 1 2, Vec 3 4] ++ exvec
\end{lstlisting}
\end{frame}



\section{Arrays}
\begin{frame}{Arrays}
A few requirements make arrays harder to implement then structs:
\begin{itemize}
  \item Only unsigned integers should be allowed as index, to aid the
  connection with C.
  \item We want to disallow nested arrays, but allow multi-dimensional arrays.
  \item We prefer arrays to be fixed length.
\end{itemize}

We have to create a custom type:
\begin{itemize}
  \item Haskell lists are not suitable $\Rightarrow$ No fixed length;
  \item \texttt{Data.Array} has fixed lengths, but allows all kinds of indices
  and type inside the array.
\end{itemize}

Later on we will see that creating a custom type has more benefits.
\end{frame}


\begin{frame}[fragile]{Proposed solution: Type-dependent arrays}
\begin{itemize}
  \item Type-dependency gives us type-safe fixed length arrays.
  \item Sadly GHC is not usable (yet) for this level of type-dependency.
\end{itemize}
\begin{lstlisting}
data Array (n :: Nat) a where
  Nil  :: Array 0 a
  (:>) :: a -> Array n a -> Array (n+1) a

infixr :>

arr :: Array 3 Int
arr = 1 :> 2 :> 3 :> Nil
\end{lstlisting}
\end{frame}

\begin{frame}[fragile]
Some functions can be implemented:
\begin{lstlisting}[basicstyle=\small]
foldr :: (a -> b -> b) -> b -> Array n a -> b
foldr f b Nil       = b
foldr f b (x :> xs) = f x (foldr f b xs)
\end{lstlisting}
For others GHC is not able to unify types:
\begin{lstlisting}[basicstyle=\small]
append :: Array n a -> Array m a -> Array (n + m) a
append Nil       ys = ys
append (x :> xs) ys = x :> append xs ys
\end{lstlisting}
\begin{lstlisting}[basicstyle=\small,language={}]
* Could not deduce: ((m + n1) + 1) ~ (m + n)
  from the context: n ~ (n1 + 1)
\end{lstlisting}
GHC does not know about basic mathematical rules.
\end{frame}



\begin{frame}[fragile]{Proposed solution: Simpler type}
Mimicking \texttt{Data.Array}:
\begin{lstlisting}
data Array i a = Index i => Array i a

array :: [(i, a)] -> Array i a
\end{lstlisting}
\begin{itemize}
  \item Nearly no type safety, except for the dimension of the array and its
  inner type.
  \item For example no difference between array of length 2 or 200.
  \item It turned out to be really convenient to have the length inside the
  type.
\end{itemize}
\end{frame}



\begin{frame}[fragile]{Final solution: Type literals}
We still want store the size of the array in the type $\Rightarrow$ type
literals.
\begin{lstlisting}[basicstyle=\small]
data Array i a where
  Array :: Index i n => i -> [(n, a)] -> Array i a

array :: forall n i a.
  Index i n => [(n, a)] -> Array i a
array xs | valid     = Array idx xs
         | otherwise = error "..."
  where
    valid = ...
    idx   = index
\end{lstlisting}
\begin{itemize}
  \item Smart constructor checks if input is valid.
  \item \texttt{Index}-class is used to define dimension of array.
  \item To be used in Copilot, we need to create an instance for \texttt{Typed}.
\end{itemize}
\end{frame}


\begin{frame}[fragile]{Indexing}
\begin{itemize}
  \item We disallow nested arrays for code generation purposes.
  \item Natural numbers would not suffice for multidimensional arrays.
  \item Instead we use \texttt{Index} to support this.
\end{itemize}
\begin{lstlisting}[basicstyle=\small]
class Index i n | i -> n where
  index     :: i
  fromIndex :: i -> [n]
  size      :: i -> Int
  size i = length $ fromIndex i
\end{lstlisting}
\end{frame}

\begin{frame}[fragile]{Indexing}
\begin{lstlisting}[basicstyle=\small]
#data Len a = Len#

instance KnownNat n => Index (Len n) Int where
  index     = Len
  fromIndex = idxRange

#instance (KnownNat m, KnownNat n) =>
    Index (Len m, Len n) (Int, Int) where
  index = (Len, Len)
  fromIndex (m, n) = [ (m', n') | m' <- idxRange m
                                , n' <- idxRange n ]#
\end{lstlisting}
\begin{itemize}
  \item The instance binds \texttt{Len n} to \texttt{Int}.
  \item<2-> \texttt{Len} is a phantom type used to promote kinds to types.
  \item<3-> Correct instances keeps users from writing invalid indices i.e.:\\
    \texttt{Index (Len n) (Int, Int)}
\end{itemize}
\end{frame}

\begin{frame}[fragile]{Array example}
As with structs, arrays have become first-class members in Copilot. They
require no specific functions to work with:
\begin{lstlisting}
-- Little helper function
array' as = array (zip [0..] as)

exarr :: Stream (Array (Len 3) Int8)
exarr = extern "exarr" Nothing

arr :: Stream (Array (Len 3) Int8)
arr = [ array' [4,5,6],
        array' [1,2,3] ] ++ arr
\end{lstlisting}
Note that the length of the external array is specified in its type.
\end{frame}


\section{ACSL}
\begin{frame}{ACSL}
\begin{itemize}
  \item ANSI C Specification Language.
  \item A language to specify what C program should do.
  \item Hoare-triples inside comments to cover a function.
  \item Tools like Frama-C can be used to check program to specification.
\end{itemize}
\end{frame}

\begin{frame}[fragile]
\begin{itemize}
  \item Generate ACSL code from our Copilot specification.
  \item We can check the monitor and specification using Frama-C.
  \item This gives us insight in possible bugs, especially regarding arrays and
  indices.
\end{itemize}

\begin{lstlisting}[language=c, basicstyle=\small]
static int8_t s0_buff[3] = {1, 2, 3};
static int8_t s0 = 1;
static size_t s0_idx = 0;

/*@ requires \valid (s0_buff+(0..2));
    requires 0 <= s0_idx < 3;
    assigns  \nothing;
    ensures  \result == s0_buff[s0_idx]; */
static int8_t s0_gen () {
  int8_t s0_loc;
  { size_t idx = s0_idx;
    s0_loc = s0_buff[idx];
  };
  return s0_loc;
}
\end{lstlisting}
\end{frame}

\begin{frame}[fragile]
In some cases, it is a lot easier:
\begin{lstlisting}[language=c]
/*@ assigns  \nothing;
    ensures  \result == true;
*/
static bool fbasic_guard () {
  return true;
}
\end{lstlisting}
\end{frame}

\begin{frame}[fragile]
The \texttt{step()} differs from the other functions:
\begin{lstlisting}[language=c, basicstyle=\small]
/*@ requires 0 <= s0_idx < 3;
    assigns  s0;
    assigns  s0_buff[s0_idx];
    assigns  s0_idx;
    ensures  \forall int i; 0 <= i < 3
        && i != \old(s0_idx) ==>
        s0_buff[i] == \old(s0_buff[i]);
*/
static void step () {
  if (fbasic_guard()) fbasic(fbasic_arg0());
  s0 = s0_gen();
  s0_buff[s0_idx] = s0;
  ++(s0_idx);
  s0_idx = s0_idx % 3;
}
\end{lstlisting}
\end{frame}



\section{Summary}
\begin{frame}{Summary}
\begin{itemize}
  \item Structs and arrays are now possible in Copilot.
  \item Type literals saved us from a lot of work, and kept the implementation
  of Copilot clean.
  \item Dependent types could improve the implementation of arrays, but GHC is
  not yet up for the task.
  \item ACSL can be usefull in both describing our program, as well as
  verifiying it using Frama-C.
\end{itemize}
\end{frame}

\section{Future work}
\begin{frame}{Future work}
\begin{itemize}
  \item Improve efficiency of generated code by implementing implicit sharing.
  \item Possibly improve array implementation to get rid of the \texttt{Len} in
  the type.
  \item Improve arrays to make them more type-safe.
  \item Implement a few functions to make working with arrays and structs easier.
\end{itemize}
\end{frame}

\end{document}
