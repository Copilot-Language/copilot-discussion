\section{Related Work}\label{sec:related}

%% Using RV to implement fault-tolerance can be considered to be a
%% ``one-out-of-two'' (1oo2) architecture~\cite{rushby-possibly}, similar in spirit
%% to the idea of the Simplex architecture~\cite{simplex}.  In a 1oo2 architecture,
%% one component may be an arbitrarily complex control system, and the other
%% component is a monitor.

%% One motivation for a 1oo2 architecture is that if the
%% monitor is developed using formal techniques, then it may be \emph{possibly
%%   perfect}~\cite{rushby-possibly}, meaning that the probability of failure on
%% demand of each component is ``mostly'' independent of each other; more
%% precisely, the dependency can be bounded.  %% The upshot of the work is that it
%% %% provides a probabilistic basis for how much additional reliability one obtains
%% %% by having a formally verified monitor.
%% Our work can be viewed as an approach to
%% generate possibly-perfect monitors.

%% Given our focus on ultra-critical systems, our work presents a possible approach
%% for \emph{runtime certification}, in which evidence of system assurance is given
%% at runtime~\cite{rvRushby}.  Traditionally, evidence is provided at design-time
%% and is process-oriented (e.g., the DO-178B standard for software development in
%% commercial aviation~\cite{DO178B}).

Copilot shares similarities with other RV systems that emphasize real-time or
distributed systems.  Kr\"{u}ger, Meisinger, and Menarini describe their work in
synthesizing monitors for a automobile door-locking system~\cite{msc}.  While
the system is distributed, it is not ultra-reliable and is not hard real-time or
fault-tolerant via hardware replication.  The implementation is in Java and
focuses on the aspect-oriented monitor synthesis, similar in spirit to
JavaMOP~\cite{javamop}.  \textsc{Syncraft} is a tool that takes a distributed
program (specified in a high-level modeling language) that is fault-intolerant
and given some invariant and fault model, transforms the program into one that
is fault-tolerant (in the same modeling language).~\cite{syncraft}.  %% Although
%% \textsc{Syncraft} does not handle real-time programming constraints, the problem
%% being solved by that tool is at the algorithmic level, rather than generating
%% code.

There are few instances of RV focused on C code.  One exception is {\sc
  Rmor}, which generates constant-memory C monitors~\cite{havelundc}.  {\sc
  Rmor} does not address real-time behavior or distributed system RV, though.

%% \textsc{Salt}~\cite{bauer} is a timed and
%% untimed linear temporal logic specification language and compiler.  Unlike
%% Copilot, Salt is not focused on generating hard real-time code, nor does it
%% address scheduling or monitor communication.  Nevertheless, \textsc{Salt} goes
%% beyond our work in the sense that it also address the diagnosis problem:
%% inferring a fault from its symptoms.

Research at the University of Waterloo also investigates the use of
time-triggered RV (i.e., periodic sampling).  Unlike with Copilot, the authors
do not make the assumptions that the target programs are hard real-time
themselves, so a significant portion of the work is devoted to developing the
theory of efficiently monitoring for state changes using time-triggered RV for
arbitrary programs, particularly for testing~\cite{sampling,borzoo}.  On the
other hand, the work does not address issues such as distributed systems,
fault-tolerance, or monitor integration.

With respect to work outside of RV, other research also addresses the use of
eDSLs for generating embedded code.  Besides Atom~\cite{atom}, which we use as a
back-end, Feldspar is an eDSL for digitial signal processing~\cite{feldspar}.
Copilot is similar in spirit to other languages with stream-based semantics,
notably represented by the Lustre family of languages~\cite{lustre-intro}.
Copilot is a simpler language, particularly with respect to Lustre's clock
calculus, focused on monitoring (as opposed to developing control systems).
Copilot can be seen as an generalization of the idea of Lustre's ``synchronous
observers''~\cite{lustre}, which are Boolean-valued streams used to track
properties about Lustre programs.  Whereas Lustre uses synchronous observers to
monitor Lustre programs, we apply the idea to monitoring arbitrary
periodically-scheduled real-time systems.  The main advantages of Copilot over
Lustre is that Copilot is implemented as an eDSL, with the associated benefits;
namely Haskell compiler and library reuse the ability to define polymorphic
functions, like the {\tt majority} macro in Section~\ref{subsec:boyer_moore}, that get
monomorphised at compile-time.


\section{Conclusions and Remaining Challenges}
\label{sec:conclusions}
Ultra-critical systems need RV.  Our primary goals in this paper are to (1)
motivate this need, (2) describe one approach for RV in the ultra-critical
domain, (3) and present evidence for its feasibility.

The approach we have described in this report is not without shortcomings, which
present opportunities for future research.

\paragraph{eDSL efficiency.}
First, we have demonstrated that the embedded DSL approach is powerful,
turning regular programming on its head: while Copilot is simple, its
macro language is a higher-order functional language!  One disadvantage of this
approach is that with a powerful macro language, it is easy to
build up large expressions---much larger than would be built in a
conventional programming language.  For example, the Boyer-Moore
voting algorithm described in Section~\ref{subsec:boyer_moore} is compiled into
a single Copilot expression.  The use of explicit sharing
(Section~\ref{sec:explicit_sharing}) reduces the cost of computation by ensuring
sub-expressions are not needlessly recomputed, but if the sub-expressions
themselves are expensive to compute, the entire expression becomes expensive.
This is analogous to a standard compiler inlining \emph{every} function, which
would result in infeasible code-size.  

Techniques to improve the efficiency of evaluating embedded domain-specific
languages (eDSLs) and borrowing sharing
from the host language to the DSL language are needed.  Fortunately in our
domain, monitoring code is terse, in general.

\paragraph{Assurance}
The lightweight approach to monitor assurance discussed in
Section~\ref{sec:tools} is described in more detail
in~\cite{PikeWNG2012}.  The current framework for front-end/back-end
testing is built on Quickcheck and model-checking, which does not provide coverage
testing capability.  Given the criticality of the monitors,
DO-178~\cite{DO178B} would require that MC/DC testing be performed on
them if they were to be employed in industrial avionics. Adding this
capability to the toolchain is a challenge for future work.

%% A more heavyweight approach to verification is exemplified by the
%% CompCert~\cite{leroy} C compiler.   Adapting a
%% similar philosophy, we could generate proofs or certificates that the C
%% monitors created by the back-end meet the specification given by the
%% Copilot operational semantics and then use the CompCert C compiler to
%% provide a certificate that the executable created during compilation
%% is correct. 

\paragraph{Scheduling monitors.}
In the experiments described in Section~\ref{sec:case-study}, we use hardware
interrupts to ensure monitors run at fixed intervals.  This technique works in
practice and obviates the need for an underlying operating system to handle
scheduling.  However, we must ensure that monitors execute quickly (so that the
monitored system does not miss other interrupts), and we need to ensure that the
monitor has been given sufficient time to execute.  With the current set of code
generators, worst-case execution time is easier to compute, as there is just one
control-path through the code (that is, worst-case execution time is equal to
nominal execution time).

    Safety-Critical hard real-time systems typically  employ real-time operating
  systems (RTOS) to manage the schedule.  Copilot has been ported to an
  ARINC 653~\cite{ARINC653-10,ARINC653-12} compliant RTOS and experiments are being planned to test
  applications being  monitored by Copilot programs, where they are both  scheduled by the OS
  using algorithms such as rate-monotonic scheduling.  

The only model of time in Copilot monitors, like other synchronous languages, is
the tick.  The tick is an abstract model of time that gets mapped to a real-time
duration by the underlying hardware.  The duration of a tick matters when
specifying monitors: the property
%
\begin{quote}
The value of {\tt x} must satisfy {\tt -0.5
  <= x - x' <= 0.5}, where {\tt x'} is the value of {\tt x} exactly one second ago.
\end{quote}
%
\noindent
requires building a stream of values.  If a tick is one second long, then the
specification
%
\begin{code}
prop = (x - x') <= 5 && (x - x') <= (-5)
  where
  x  = [0] ++ e0 
  x' = drop 1 x
  e0 = externI32 "x" Nothing
\end{code}
%
\noindent
If a tick is a half-second, we must use {\tt drop 2 ...}, and so on.  Thus,
monitors may be hardware/scheduler dependent.  It would help the specifier to
lift the abstraction level, so she can write properties in terms of real-time.

\paragraph{Other language features.}
In analyzing protocol streams, reconstructing values out of the payload of
a packet from incoming bytes is necessary. Copilot currently lacks casting
operations to do this. Adding a general set of casting functions that includes
different byte orders, bit orders and number representations would help on
monitoring protocols.

\paragraph{Steering.}
We have not addressed the \emph{steering} problem of how to address faults once
they are detected.  Steering is critical at the application level; for example,
if an RV monitor detects that a control system has violated its permissible
operational envelope.

\paragraph{Faults.}
We have built a system to detect both hardware and software (logical) faults.
Stochastic methods might be used to distinguish random hardware faults from
systematic faults, as the steering strategy for responding to each
differs~\cite{SammapunLS05}.

\paragraph{Conclusions.}
Research developments in RV have potential to improve the reliability of
ultra-critical systems.  Research into runtime monitoring for hard real-time
distributed systems has been under-represented in the community, but we hope a
growing number of RV researchers address this application domain.
