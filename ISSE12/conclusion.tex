\section{Conclusions and Remaining Challenges}
\label{sec:conclusions}
Ultra-critical systems need RV.  Our primary goals in this paper are to (1)
motivate this need, (2) describe one approach for RV in the ultra-critical
domain, (3) and present evidence for its feasibility.

The approach we have described in this report is not without shortcomings, which
present opportunities for future research.

\paragraph{eDSL efficiency.}
First, we have demonstrated that the embedded DSL approach is powerful,
turning regular programming on its head: while Copilot is simple, its
macro language is a higher-order functional language!  One disadvantage of this
approach is that with a powerful macro language, it is easy to
build up large expressions---much larger than would be built in a
conventional programming language.  For example, the Boyer-Moore
voting algorithm described in Section~\ref{subsec:boyer_moore} is compiled into
a single Copilot expression.  The use of explicit sharing
(Section~\ref{sec:explicit_sharing}) reduces the cost of computation by ensuring
sub-expressions are not needlessly recomputed, but if the sub-expressions
themselves are expensive to compute, the entire expression becomes expensive.

Techniques to improve the efficiency of evaluating eDSLs are needed.
Fortunately, monitoring code is terse, in general.  

\paragraph{Scheduling monitors.}
In the experiments described in Section~\ref{sec:case-study}, we use hardware
interrupts to ensure monitors run at fixed intervals.  This technique works in
practice and obviates the need for an underlying operating system to handle
scheduling.  However, we must ensure that monitors execute quickly (so that the
monitored system does not miss other interrupts), and we need to ensure that the
monitor has been given sufficient time to execute.  With the current set of code
generators, worst-case execution time is easy to compute, as there is just one
control-path through the code (that is, worst-case execution time is equal to
nominal execution time).  

The only model of time in Copilot monitors, like other synchronous languages, is
the tick.  The tick is an abstract model of time that gets mapped to a real-time
duration by the underlying hardware.  The duration of a tick matters when
specifying monitors: the property
%
\begin{quote}
The value of {\tt x} must satisfy {\tt -0.5
  <= x - x' <= 0.5}, where {\tt x'} is the value of {\tt x} exactly one second ago.
\end{quote}
%
\noindent
requires building a stream of values.  If a tick is one second long, then the
specification
%
\begin{code}
prop = (x - x') <= 5 && (x - x') <= (-5)
  where
  x  = [0] ++ e0 
  x' = drop 1 x
  e0 = externI32 "x" Nothing
\end{code}
%
\noindent
If a tick is a half-second, we must use {\tt drop 2 ...}, and so on.  Thus,
monitors may be hardware/scheduler dependent.  It would help the specifier to
lift the abstraction level, so she can write properties in terms of real-time.

\paragraph{Other language features.}
In analyzing protocol streams, reconstructing values out of the payload of
a packet from incoming bytes is necessary. Copilot currently lacks casting
operations to do this. Adding a general set of casting functions that includes
different byte orders, bit orders and number representations would help on
monitoring protocols.

\paragraph{Steering.}
We have not addressed the \emph{steering} problem of how to address faults once
they are detected.  Steering is critical at the application level; for example,
if an RV monitor detects that a control system has violated its permissible
operational envelope.

\paragraph{Faults.}
We have built a system to detect both hardware and software (logical) faults.
Stochastic methods might be used to distinguish random hardware faults from
systematic faults, as the steering strategy for responding to each
differs~\cite{SammapunLS05}.

\paragraph{Conclusions.}
Research developments in RV have potential to improve the reliability of
ultra-critical systems.  Research into runtime monitoring for hard real-time
distributed systems has been under-represented in the community, but we hope a
growing number of RV researchers address this application domain.
