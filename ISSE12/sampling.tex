\subsection{Timed-Triggered Monitoring} \label{sec:sampling}

Monitoring based on sampling state-variables has historically been disregarded as a
runtime monitoring approach, for good reason: without the assumption of
synchrony between the monitor and observed software, monitoring via
sampling may lead to false positives and false negatives~\cite{DwyerDE08}.  For
example, consider the property $(0;1;1)^*$, written as a regular expression,
denoting the sequence of values a monitored variable may take.  If the monitor
samples the variable at the inappropriate time, then both false negatives
(the monitor erroneously rejects the sequence of values) and false positives
(the monitor erroneously accepts the sequence) are possible.  For example, if
the actual sequence of values is $0,1,1,0,1,1$, then an observation of $0,1,1,1,1$ is a
false negative by skipping a value, and if the actual sequence is
$0,1,0,1,1$, then an observation of $0,1,1,0,1,1$ is a false positive by
sampling a value twice.


%Suppose the values $0$ and $1$ are the values a monitored variable $v$ may take.
%values or by sampling the same value twice.  For example, if $v$ contains the
%sequence of values $0,1,1,0,1,1$, then skipping the second `$0$' yields the
%observation $0,1,1,1,1$, a false negative.  If $v$ contains $0,1,1,1,0,1,1$,
%then skipping the third `$1$' yields the observation $0,1,1,0,1,1$, a false
%positive.

However, in a hard real-time context, sampling is a suitable strategy.  Under
the assumption that the monitor and the observed program share a global clock
and a static periodic schedule, while false positives are possible, false
negatives are not.  A false positive is possible, for example, if the program
does not execute according to its schedule but just happens to have the expected
values when sampled.  If a monitor samples an unacceptable sequence of values,
then either the program is in error, the monitor is in error, or they are not
synchronized, all of which are faults to be reported.

Most of the popular runtime monitoring frameworks inline monitors in
the observed program to avoid the aforementioned problems with
sampling.  However, inlining monitors changes the real-time behavior
of the observed program, perhaps in unpredicable ways. Recalling our
four criteria from Section~\ref{sec:constraints}, monitors that
introduce such unpredictability are not a viable solution for
ultra-critical hard real-time systems.  In a sampling-based approach,
the monitor can be integrated as a separate scheduled process during
available time slices (this is made possible by generating efficient
constant-time monitors).  Indeed, sampling-based monitors may even be
scheduled on a separate processor (albeit doing so requires additional
synchronization mechanisms), ensuring time and space partitioning from
the observed programs.  Such an architecture may even be necessary if
the monitored program is physically distributed. 

Recent work in RV investigates the use of time-triggered sampling for arbitrary
programs~\cite{sampling,borzoo,rv11stoller}, which is a harder problem than in our hard
real-time context.

% Another shortcoming of inlining monitors is
%that certified code (e.g., DO-178B for avionics~\cite{DO178B}) is common in this
%domain.  Inlining monitors could necessitate re-certifying the observed program.
%We cannot claim our approach would obviate the need for re-certification, but it
%is a more modular approach than one based on instrumenting the source code of
%the observed program, which may result in a less onerous re-certification
%process.
