%%%%%%%%%%%%%%%%%%%%%%%%%%%%%%%%%
% Springer Journal format for special issue of ISSE on SHM

\documentclass[twocolumn]            % options: twocolumn, smallcondinsed, smallextended 
{report-templates/svjour3}                       %   plus final, draft, referee

\usepackage[utf8]{inputenc}
\usepackage[hyphens]{url}
%% \usepackage{breakurl}
\usepackage{graphicx}
\usepackage{color}
\usepackage{amssymb}
\usepackage{float}
\usepackage{amsmath}
%%%\usepackage{amsthm}
\usepackage{setspace}
\usepackage{framed}

\usepackage{listings}
%\usepackage{natbib}
\usepackage{fancyvrb}
%\usepackage{microtype}
\usepackage{alltt}
% \usepackage{tabularx}
\usepackage[center]{caption}
\usepackage{subfigure}
\usepackage{fancyhdr}

\usepackage{tikz}
\usetikzlibrary{arrows,automata}

% table of contents depth
\setcounter{tocdepth}{2}


% boxes around figs
\usepackage{float}
\floatstyle{boxed} 
\restylefloat{figure}

\DefineVerbatimEnvironment{code}{Verbatim}{fontsize=\footnotesize}

\lstset{basicstyle=\tt\footnotesize}

\lstdefinelanguage{Haskell}
	{
		morekeywords={as, case, of, class, data, data, family, data, instance,
		default, deriving, do, forall, foreign, hiding, if, then, else, import,
		infix, infixl, infixr, let, in, mdo, module, newtype, proc, qualified,
		rec, type, where},
    	morecomment=[l]{--},
    	morecomment=[s]{{\{-}{-\}}},
    	morestring=[b]",
    	keywordstyle=\color{blue}\bf,
    	commentstyle=\it\color{ForestGreen},
    	stringstyle=\color{red},
    	literate= {<-}{{$\leftarrow$}}1 {->}{{$\rightarrow$}}1 {-<}{{$\prec$}}1
    	{=>}{{$\Rightarrow$}}1 {>>>}{{$\ggg$}}2 {<<<}{{$\lll$}}2 {***}{{$\times$}}1
    	{&&&}{{$\otimes$}}1  {'a}{{$\alpha$}}1 {'b}{{$\beta$}}1 {'c}{{$\gamma$}}1
    	{'d}{{$\delta$}}1 {'e}{{$\eta$}}1
    }

%\newtheoremstyle{example}{\topsep}{\topsep}
%    {\normalsize\sl} % Body font.
%     {}               % Indent amount (empty = no indent, \parindent = para indent).
%     {\small\it}      % Thm head font.
%     {:}              % Punctuation after thm head.
%     {\newline}       % Space after thm head (\newline = linebreak).
%     {\thmname{#1} \thmnumber{#2}\thmnote{#3}} % Thm head spec.
%\theoremstyle{example}
%\newtheorem{example}{Example}

%% \newcommand{\hlinepage}{\rule{\textwidth}{0.25pt}}
%% \newcommand{\HRule}{\rule{\linewidth}{0.25pt}}
\newcommand{\fixme}[1]{\emph{\color{Red}\{!~#1~!\}}}
% - FROM TUTORIAL --------------------------------------------------------------------

%\usepackage[colorlinks=true,linkcolor=blue,citecolor=red,pagebackref=true]{hyperref}


% took todo command over from the ICFP paper
\usepackage{ifthen}
\newboolean{submission}  %set to true for the submission version
\setboolean{submission}{false}
%\setboolean{submission}{true}
\ifthenelse
{\boolean{submission}}
{ \newcommand{\todo}[1]{ } } %hide todo
{ \newcommand{\todo}[1]{ {\color{red}$<<$#1$>>$}
}}

\newenvironment{myfig}{\begin{figure*}}{\end{figure*}}

%\usepackage{subfigure}



\usepackage{fancyhdr}

\title{Copilot: Monitoring Embedded Systems}

\author{Lee Pike and Nis Wegmann and Sebastian Niller and Alwyn Goodloe}

%% \\Galois, Inc.\\\url{leepike@galois.com}
%%   \and \\University of Copenhagen\\ \url{wegmann@diku.dk} 
%%   \\ National Institute of
%%   Aerospace\\ \url{sebastian.niller@gmail.com} \and Alwyn Goodloe\\National
%%   Institute of Aerospace\\\url{a.goodloe@nasa.gov}\footnote{The research contributed by
%%   Alwyn Goodloe was performed during his tenure as a staff scientist at the
%%   National Institute of Aerospace; his current affiliation is with the NASA
%%   Langley Research Center.}
%% }


%% \author{Lee Pike\\Galois, Inc.\\\url{leepike@galois.com}
%%   \and Nis Wegmann\\University of Copenhagen\\ \url{wegmann@diku.dk} 
%%   \and Sebastian Niller\\ National Institute of
%%   Aerospace\\ \url{sebastian.niller@gmail.com} \and Alwyn Goodloe\\National
%%   Institute of Aerospace\\\url{a.goodloe@nasa.gov}\footnote{The research contributed by
%%   Alwyn Goodloe was performed during his tenure as a staff scientist at the
%%   National Institute of Aerospace; his current affiliation is with the NASA
%%   Langley Research Center.}
%% }


 \journalname{Innovations in Systems and Software Engineering}

\institute{Lee Pike \at 
Galois, Inc., Portland, Oregon\\
\email{leepike@galois.com} 
\and
Nis Wegmann \at 
University of Copenhagen, Copenhagen, Denmark\\
\email{wegmann@diku.dk}
\and 
Sebastian Niller \at 
Unaffiliated\\
\email{sebastian.niller@gmail.com}
\and
Alwyn Goodloe \at 
NASA Langley Research Center, Hampton, Virginia  \\
\email{a.goodloe@nasa.gov}
}




\begin{document}\sloppy

\maketitle
\abstract{Runtime verification (RV) is a natural fit for
  ultra-critical systems that require correct software behavior.  Due
  to the low reliability of commodity hardware and the adversity of
  operational environments, it is common in ultra-critical
  systems to replicate processing units (and their hosted software)
  and incorporate fault-tolerant algorithms to compare the outputs,
  even if the software is considered to be fault-free.  In this paper
  we investigate the use of software monitoring in distributed
  fault-tolerant systems and the implementation of fault-tolerance
  mechanisms using RV techniques, we describe the Copilot language and
  compiler that generates monitors for distributed real-time systems,
  and we discuss two case-studies in which Copilot-generated monitors
  were used to detect onboard software and hardware faults and monitor
  air-ground data link messaging protocols.}
 

%% Runtime verification (RV) is a natural fit for ultra-critical systems, where
%%  correctness is imperative.  In ultra-critical systems, even if the software is
%%  fault-free, because of the inherent unreliability of commodity hardware and
%%  the adversity of operational environments, processing units (and their hosted
%%  software) are replicated, and fault-tolerant algorithms are used to compare
%%  the outputs.  We investigate both software monitoring in distributed
%%  fault-tolerant systems as well as implementing fault-tolerance mechanisms
%%  using RV techniques.  We describe the Copilot language and compiler,
%%  specifically designed for generating monitors for distributed, hard real-time
%%  systems.  We also describe two case-studies in which we generated Copilot
%%   monitors in avionics systems.}

\input intro

\input overview

\input sampling

\input copilot_tutorial


%\input watchmen

\input flight-tests

\input mavlink

%\input target
%\input speclang
%       \input creating
%       \newpage
%\input integration
%\input future

\input conclusion

\begin{acknowledgements}
This work is supported by NASA Contract NNL08AD13T.  Portions
  of this work have been published as conference papers in \emph{Runtime
    Verification}, 2010 and  \emph{Runtime
    Verification}, 2011.  We wish to especially thank the following individuals:
  Ben Di~Vito at the NASA Langley Research Center (NASA LaRC) monitored this
  contract, and Paul Miner and Eric Cooper, also at NASA LaRC, provided valuable
  input.  Robin Morisset developed an earlier version of Copilot.
\end{acknowledgements}


\addcontentsline{toc}{section}{References}
\bibliographystyle{report-templates/spmpsci}  %%%  options spbasic or  spmpsci  
                          
\bibliography{biblio,mybib}

\end{document}
