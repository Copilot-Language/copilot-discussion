\section{Runtime Monitoring for Embedded Systems: Constraints and Approaches}
\label{sec:constraintsApproaches}
In this section, we  present constraints to runtime monitoring for
real-time embedded systems. 

\subsection{RV Constraints}\label{sec:constraints}
%\section{RV Constraints in Ultra-Critical Systems}\label{sec:constraints}
%\subsection{Constraints}

Ideally, the RV approaches that have been developed in the literature could be
applied straightforwardly to ultra-critical systems.  Unfortunately,
typical RV approaches violate constraints imposed on ultra-critical
systems. We summarize these
constraints using the acronym ``FaCTS'':

\begin{itemize}
\item {\bf F}unctionality: the RV system cannot change the target's behavior (unless the
  target has violated a specification).
\item {\bf C}ertifiability: the RV system must not make re-certification (e.g.,
  DO-178B~\cite{DO178B}) of the target onerous.
\item {\bf T}iming: the RV system must not interfere with the target's timing.
\item {\bf S}WaP: The RV system must not exhaust size, weight, and power (SWaP)
  tolerances.
\end{itemize}

The functionality constraint is common to all RV systems, and we will not
discuss it further.  The certifiability constraint is at odds with
aspect-oriented programming techniques, in which source code instrumentation
occurs across the code base---an approach classically taken in RV (e.g., the
Monitor and Checking (MaC)~\cite{KimVBKLS99} and Monitor Oriented Programming
(MOP)~\cite{ChenR05} frameworks).  For codes that are certified, instrumentation
is not a feasible approach, since it requires costly reevaluation of the code.
Source code instrumentation can modify both the control flow of the instrumented
program as well as its timing properties.  Rather, an RV approach must isolate
monitors in the sense of minimizing or eliminating the effects of monitoring on
the observed program's control flow.

Timing isolation is also necessary for real-time systems to ensure that timing
constraints are not violated by the introduction of RV.  Assuming a fixed upper
bound on the execution time of RV, a worst-case execution-time analysis is used
to determine the exact timing effects of RV on the system---doing so is
imperative for hard real-time systems.

Code and timing isolation requirements cause the most significant deviations from
traditional RV approaches.  Section~\ref{sec:sampling}  argues 
that these requirements dictate a \emph{time-triggered} RV approach, in which a program's state is
periodically sampled based on the passage of time rather than occurrence of
events~\cite{copilot}.

The final constraint, SWaP, applies both to memory (embedded processors may have
just a kilobytes of available memory) as well as additional hardware (e.g.,
processors or interconnects).




%% \subsection{Architectures}\label{sec:archs}


%% Ideally, a monitoring architecture requires as little
%% additional hardware as possible; besides SWaP considerations, additional
%% hardware increases the likelihood of some component suffering a random fault.
%% On the other hand, a monitor should reduce common failures with the target
%% system.  Finally, we include as a design goal making few changes to a target's
%% architecture to enable RV.  We identify three broad architectures emphasizing
%% various aspects of these considerations.

%% Suppose the target is made up of a set of distributed processes $x_0$,\nobreak
%% $x_1$,\nobreak $\dots$,\nobreak $x_n$ together with an interconnect.  The
%% interconnect between processes could be a simple bus or form other network
%% topologies~\cite{Rushby01:buscompare}.  As depicted in
%% Figure~\ref{fig:singlebus}, we propose three architecture families: (1) A single
%% monitor watching a shared bus, (2) a single monitor watching on a dedicated bus,
%% and (3) distributed monitors.  We discuss each in turn.

%% \begin{figure}[ht]
%%   \centering
%% \begin{framed}
%%   \begin{tabular}[t]{c|c|c}
%%   \includegraphics[width=3cm]{Figs/BusMonArch}  &
%%   \includegraphics[width=3cm]{Figs/oneMonOnBus} &
%%   \includegraphics[width=3cm]{Figs/NewdistMonArch}\\
%% Shared bus. & Dedicated bus. & Distributed monitors.
%%   \end{tabular}


%% % \end{framed}
%% % \caption{A Monitor Watching a Shared Bus}
%% % \end{myfig}

%% % \begin{figure}[ht]
%% % \begin{framed}
%% %   \centering

%% % \end{framed}
%% % \caption{Single Monitor on a Dedicated Bus}
%% % \end{myfig}

%% % \begin{figure}[ht]
%% % \begin{framed}
%% %   \centering

%% \end{framed}
%% \caption{Three monitoring architectures.\label{fig:dist}\label{fig:onemonbus}\label{fig:singlebus}}
%% \end{myfig}

%% %\todo{Edit the distributed monitors figure}


%% \paragraph{A single monitor watching a shared bus.}
%% In this architecture, the monitor receives messages over the bus just like any
%% other process in the system.  In general, the monitor is a silent participant
%% that does additional error checking on in-bound messages or verification that
%% protocol communication patterns are followed.  Only if it detects a fault, it
%% would send messages to the other processes through the bus alerting them of the
%% fault.  This architecture requires only a single additional processing element,
%% making RV cheap, simple, and less likely to be a source of faults itself.
%% However, the monitor must be ensured not to become faulty, flooding the data-bus
%% with wrong messages.  In this architecture, the monitor must be constrained to
%% fail silently and prevented from being a ``babbling idiot'' on the bus~\cite{Rushby01:buscompare}.

%% BusMoP adapts this architecture to verify bus protocols~\cite{PellizzoniCR08A}.
%% In the case of BusMoP, the monitor is implemented in an FPGA that is interposed
%% between the peripheral hardware and the bus allowing it to sniff all activities
%% on the bus.


%% %% The class of faults this monitoring architecture is able to capture, however, is
%% %% limited: the monitor can only infer the health of a process from the messages
%% %% processes passes over the bus since monitors have no access to a process's
%% %% private state. Depending on the type of interconnect, the monitor may not
%% %% be able to report faults to processes if processes are able to
%% %% flood the data-bus.

%% \paragraph{A single monitor watching on a dedicated bus.}
%% An alternative architecture is one in which the monitor observes on a separate
%% dedicated monitoring bus from the one used by the target.  In this architecture,
%% each process is instrumented to send data to the monitor over the monitor bus,
%% and the single monitor can compare the incoming data.  As necessary, the monitor
%% may signal the processes if a fault is detected.

%% %% The modification required of the target to implement this architecture is
%% %% ostensibly minor; each process needs only to be instrumented to broadcast select
%% %% messages that it receives or sends over another bus, helping to ensure that the

%% In this architecture, all state information must be explicitly passed to the
%% monitors via dedicated messages over the monitor's bus.  By using a separate
%% bus, there is less chance the monitor causes disruptions on the target's data
%% bus, including timing violations or other faults.  Moreover, faults in the
%% target's bus are independent of the monitor's bus.  The primary disadvantage of
%% this architecture, similar to the previous one, is that the monitor is a
%% single point of failure.

%% Architectures such as TTTech's TTEthernet~\cite{ttethernet} provide interesting
%% possibilities to safely combine monitoring traffic with the target's bus
%% traffic.  TTEthernet logically partitions time-triggered rate-constrained, and
%% best-effort Ethernet traffic on the same bus.


%% % In some cases, it might be possible to conceptually combine an architecture in
%% % which a monitor has a dedicated bus with the previous architecture described, in
%% % which a monitor watches a shared bus.  For example, if the interconnect is
%% % , monitor messages might be sent between
%% % distributed monitors over the network as conventional Ethernet traffic, which is
%% % guaranteed by the TTEthernet not to interfere with safety-critical messages sent
%% % over the same network.


%% \paragraph{Distributed monitors.}
%% The final architecture we consider is one in which monitors are distributed with
%% the distributed processes of the target.  In Figure~\ref{fig:dist}, the monitors
%% $M_0$, $M_1$, $\ldots$, $M_n$ corresponding to each process $x_0$, $x_1$,
%% $\ldots$, $x_n$ in the target.

%% %% Monitor $M_i$ may be implemented on the same
%% %% hardware as process $x_i$ and sharing the same interconnect as is depicted in
%% %% Figure~\ref{fig:dist}, or the monitor could be physically separated and possibly
%% %% have a dedicated interconnect; the former is cheaper (in terms of hardware
%% %% resources), but the latter provides less chance of a monitor sharing the
%% %% fault-modes of its associated process.

%% Distributed monitors provide a way to ``bolt on'' fault-tolerance to an existing
%% distributed system; we give an example of monitors implementing fault-tolerance
%% in Section~\ref{sec:case-study}.  This architecture is similar to the
%% architecture implemented in~\cite{SenVAR04}, where the task of monitoring
%% distributed systems is distributed among monitors at each node, but local
%% monitors require information be transmitted from other nodes over the
%% network. The architecture described in~\cite{bauer} has local monitors that
%% verify safety properties of local state, but transmits information to a
%% centralized diagnosis module that obtains a global view of the system. In both
%% cases, the monitors appear to share a bus or network with the system under
%% observation, which given the traffic between monitors may compromise hard
%% real-time deadlines.

%% %% As compared to the single process-monitor architecture in
%% %% Figure~\ref{fig:onemonbus}, this architecture has the advantage that it supports
%% %% a potentially more reliable monitor since the monitors are distributed.  So even
%% %% if one or more $M_i$'s fail, the monitor may assist the system.  On the other
%% %% hand, this architecture is one of the most expensive in terms of additional
%% %% size, weight, and power.


%% % A distributed monitor also has an additional benefit in that an individual
%% % monitor $M_i$ can serve as a guardian to its corresponding process $x_i$.  In
%% % the single process-monitor architecture, a ``babbling'' process could prevent
%% % other processes from sending messages to the monitor or prevent the monitor from
%% % signaling the processes in the case of a detected error.

%% % On the other hand, a distributed monitor is more complicated, and that
%% % complexity may lead to less reliability in the monitor.  Furthermore, the
%% % distributed process-monitor architecture may be nearly as expensive in terms of
%% % processes and interconnects as the SUO itself.  A distributed architecture may
%% % not be feasible if the monitor has stringent cost, size, weight, or energy
%% % consumption constraints.


%% % redundancy and fault-tolerance, forming a ``virtual bus''; for example, in
%% % NASA's Reliable Optical Bus (ROBUS), the interconnect is a bipartite graph used
%% % to provide Byzantine fault-tolerance~\cite{spider_design}.


%% %\subsection{Bus-Monitor Architecture}\label{arch:singlebus}



%% \section{Copilot: a Language for Ultra-Critical RV}\label{sec:copilot}


%% To answer the challenge of RV in the context of fault-tolerant systems, we have
%% developed a stream language called \emph{Copilot}.\footnote{Copilot is released
%%   under the BSD3 license and pointers to the compiler and libraries can be found
%%   at \url{http://leepike.github.com/Copilot/}.}  Copilot is designed to achieve the
%% ``FaCTS'' constraints described in Section~\ref{sec:constraints}.

%% \begin{figure}[ht]
%% %  \centering
%% \begin{framed}
%% \noindent

%% %\todo{find ambiguous ``compiler'' reference}

%% If the majority of the three engine temperature probes has exceeded 250 degrees, then
%% the cooler is engaged and remains engaged until the temperature of the majority
%% of the probes drop to 250 degrees or less.  Otherwise, trigger an immediate
%% shutdown of the engine.
%% \begin{code}
%% engineMonitor = do
%%   trigger "shutoff" (not overHeat) [arg maj]
%%   where 
%%   vals     = map externW8 ["tmp_probe_0", "tmp_probe_1", "tmp_probe_2"]
%%   exceed   = map (< 250) vals
%%   maj      = majority exceed
%%   checkMaj = aMajority exceed maj
%%   overHeat = (extern "cooler" || (maj && checkMaj)) `since' not maj
%% \end{code}

%% %% \begin{tabular}[t]{l|@{$\;\;\;$}r}
%% %% \begin{minipage}{0.45\textwidth}
%% %% \begin{code}
%% %% -- external variables
%% %% t0       = extW8 "temp_probe_0"
%% %% t1       = extW8 "temp_probe_1"
%% %% t2       = extW8 "temp_probe_2"
%% %% cooler   = extB  "fan_status"
%% %% -- Copilot variables
%% %% maj      = varW8 "maj"
%% %% check    = varB  "maj_check"
%% %% overHeat = varB  "over_heat"
%% %% monitor  = varB  "monitor"
%% %% \end{code}
%% %% %$<$\emph{continued on right}$>$
%% %% \end{minipage} &

%% %% \begin{minipage}{0.45\textwidth}
%% %% $<$\emph{continued from left}$>$
%% %% \begin{code}
%% %% engineMonitor = do
%% %%   let temps = map (< 250) [t0, t1, t2]
%% %%   maj      .= majority temps
%% %%   check    .= aMajority temps maj
%% %%   overHeat `ptltl`
%% %%     ((cooler || maj && check)
%% %%        `since' not maj)
%% %%   monitor .= not overHeat
%% %%   trigger monitor "shutoff" void
%% %% \end{code}
%% %% \end{minipage}
%% %% \end{tabular}
%% \end{framed}
%%  \caption{A safety property and its corresponding Copilot monitor specification.}
%%   \label{fig:engine}
%% \end{myfig}

%% While a preliminary description of the language has been
%% presented~\cite{copilot}, significant improvements to the language have been
%% made and the compiler has been fully reimplemented.  In any event, the focus of
%% this paper is the unique properties of Copilot for implementing hardware
%% fault-tolerance and software monitoring in the context of an ultra-critical
%% system.  Copilot is a language with stream semantics, similar to languages like
%% Lustre~\cite{lustre-intro}; we mention advantages of Copilot over Lustre in
%% Section~\ref{sec:related}.  

%% To briefly introduce Copilot, we provide an example Copilot specification in
%% Figure~\ref{fig:engine}.  A Copilot monitor program is a sequence of
%% \emph{triggers}.  A trigger is comprised of a name ({\tt shutoff}), a Boolean
%% guard ({\tt not overHeat}), and a list of arguments (in this case, one argument,
%% {\tt maj}, is provided).  If and only if the condition holds is the function
%% {\tt shutoff} called with the arguments.  What a trigger does is
%% implementation-dependent; if Copilot's C code generator is used, then a raw C
%% function with the prototype
%% \begin{code}
%% void shutoff(uint8_t maj);
%% \end{code}
%% should be defined.  Within a single Copilot program, triggers are scheduled to
%% fire synchronously, if they fire at all.  Outside of triggers, a Copilot monitor
%% is side-effect free with respect to non-Copilot state.  Thus, triggers are used
%% for other events, such as communication between monitors, as described in
%% Section~\ref{sec:distributed}.

%% %% In addition to Copilot's ability to sample C variables (and functions and
%% %% arrays), the language supports \emph{triggers}.  The purpose of triggers is to
%% %% provide a mechanism for Copilot monitors to make stateful changes to its
%% %% environment.  Outside of triggers, a Copilot monitor is side-effect free with
%% %% respect to non-Copilot state.  A trigger calls a function if a Boolean Copilot
%% %% variable is ever true.  For example, in the trigger in Figure~\ref{fig:engine},
%% %% the C function {\tt shutoff} is called with no arguments (triggers can be
%% %% defined to take an arbitrary number of arguments) if the Copilot variable {\tt
%% %%   monitor} is true.  It is the responsibility of programmer to write a
%% %% corresponding {\tt shutoff} C function.  Within a single monitor specification,
%% %% triggers are scheduled to fire synchronously, if they fire at all, after each
%% %% stream has generated a new value.  We can also use triggers for communication
%% %% between monitors, as described in Section~\ref{sec:distributed}.


%% A trigger's guard and arguments are \emph{stream}
%% expressions.  Streams are infinite lists of values.  The syntax for defining
%% streams is nearly identical to that of Haskell list expressions; for example,
%% the following is a Copilot program defining the Fibonacci sequence.
%% \begin{code}
%% fib = [0, 1] ++ fib + drop 1 fib
%% \end{code}
%% In Copilot streams, operators are automatically applied point-wise; for example,
%% negation in the expression {\tt not overHeat} is applied point-wise over the
%% elements of the stream {\tt overHeat}.  In Figure~\ref{fig:engine}, the streams
%% are defined using library functions.  The functions {\tt majority}, {\tt
%%   aMajority}, and {\tt `since'} are all Copilot library functions.  The
%% functions {\tt majority} (which determines the majority element from a list, if
%% one exists---e.g., {\tt majority [1, 2, 1, 2, 1] == 1}) and {\tt aMajority}
%% (which determines if any majority element exists) come from a majority-vote
%% library, described in more detail in Section~\ref{sec:ft}.  The function {\tt
%%   `since'} comes from a a past-time linear temporal logic library.  Libraries
%% also exist for defining clocks, linear temporal logic expressions, regular
%% expressions, and simple statistical characterizations of streams.

%% %% In the operator or a specialized library operator (e.g., {\tt `ptltl'}, denoting
%% %% a past-time linear temporal logic expression).  Copilot variables, like {\tt
%% %%   maj}, are variables that hold values produced by stream expressions.  Copilot
%% %% variables can be built over the same types as external variables.  

%% Copilot is a typed language, where types are enforced by the Haskell type system
%% to ensure generated C programs are well-typed.  Copilot is \emph{strongly typed}
%% (i.e., type-incorrect function application is not possible) and \emph{statically
%%   typed} (i.e., type-checking is done at compile-time).  We rely on the type
%% system to ensure the Copilot compiler is type-correct.  The base types are
%% Booleans, unsigned and signed words of width 8, 16, 32, and 64, floats, and
%% doubles.  All elements of a stream must belong to the same base type.

%% To sample values from the ``external world'', Copilot has a notion of
%% \emph{external variables}.  External variables include any value that can be
%% referenced by a C variable (as well as C functions with a non-void return type
%% and arrays of values).  In the example, three external variables are sampled:
%% {\tt tmp\_probe\_0}, {\tt tmp\_probe\_1}, {\tt tmp\_probe\_2}.  External
%% variables are lifted into Copilot streams by applying a typed ``extern''
%% function.  For example, An expression {\tt externW8 "var"} is a stream of values
%% sampled from the variable {\tt var}, which is assumed to be an unsigned 8-bit
%% word.

%% Copilot is implemented as an \emph{embedded domain-specific language} (eDSL).
%% An eDSL is a domain-specific language in which the language is defined as a
%% sub-language of a more expressive host language.  Because the eDSL is embedded,
%% there is no need to build custom compiler infrastructure for Copilot---the host
%% language's parser, lexer, type system, etc. can all be reused.  Indeed, Copilot
%% is \emph{deeply embedded}, i.e., implemented as data in the host language that
%% can be manipulated by ``observer programs'' (in the host language) over the
%% data, implementing interpreters, analyzers, pretty-printers, compilers, etc.
%% Copilot's host language is the pure functional language
%% Haskell~\cite{Haskell98}.  In one sense, Copilot is an experiment to answer the
%% question, ``To what extent can functional languages be used for ultra-critical
%% system monitoring?''

%% One advantage of the eDSL approach is that Haskell acts as a powerful macro
%% language for Copilot.  For example, in Figure~\ref{fig:engine}, the expression
%% \begin{code}
%% map externW8 ["tmp_probe_0", "tmp_probe_1", "tmp_probe_2"]
%% \end{code}
%% is a \emph{Haskell} expression that maps the external stream operator {\tt externW8} over a
%% list of strings (variable names).  We discuss macros in more detail in Section~\ref{sec:ft}.  

%% Additionally, by reusing Haskell's compiler infrastructure and type system, not
%% only do we have stronger guarantees of correctness than we would by writing a
%% new compiler from scratch, but we can keep the size of the compiler
%% infrastructure that is unique to Copilot small and easily analyzable; the
%% combined front-end and core of the Copilot compiler is just over two thousand
%% lines of code.  Our primary back-end generating C code is around three thousand
%% lines of code.

%% %% We have executed Copilot-generated specifications on an ATmega328 microprocessor
%% %% as well as on an ARM~Cortex~M3.
%% %% The Copilot compiler, using Atom~\cite{} as a
%% %% backend, generates C99 code, so any processor for which a C compiler exists is a
%% %% potential target.  The code generated is a strict subset of C99, without
%% %% non-array pointers or dynamic memory allocation.  Function bodies are generated
%% %% in static single assignment form~\cite{}, simplifying worst-case execution time
%% %% analysis.  However, the program's hard real-time guarantees depend on various
%% %% hardware environmental assumptions; e.g., a cache will change timing properties.

%% Copilot is designed to integrate easily with multiple back-ends.  Currently, two
%% back-ends generate C code.  The primary back-end uses the Atom eDSL~\cite{atom}
%% for code generation and scheduling.  Using this back-end, Copilot compiles into
%% constant-time and constant-space programs that are a small subset of C99.  By
%% \emph{constant-time}, we mean C programs such that the number of statements
%% executed is not dependent on control-flow\footnote{We do not presume that a
%%   constant-time C program implies constant execution time (e.g., due to
%%   hardware-level effects like cache misses), but it simplifies execution-time
%%   analysis.} and by \emph{constant-space}, we mean C programs with no dynamic
%% memory allocation.

%% %% Function bodies are generated in approximate
%% %% static single assignment form~\cite{ssa}, simplifying worst-case execution time
%% %% analysis; control-flow does not affect the execution time of Copilot-generated
%% %% code.

%% The generated C is suitable for compiling to embedded microprocessors: we have
%% tested Copilot-generated specifications on the AVR (ATmega328 processor) and
%% STM32 (ARM~Cortex~M3 processor) micro-controllers.  Additionally, the compiler
%% generates its own static periodic schedule, allowing it to run on bare hardware
%% (e.g., no operating system is needed).  The language follows a sampling-based
%% monitoring strategy in which variables or the return values of functions of an
%% observed program are periodically sampled according to its schedule, and
%% properties about the observations are computed.


%% %% Copilot monitors are functions that do not affect the state of
%% %% the programs they are composed with, unless some monitored condition is
%% %% violated.  Generated monitors are C functions with their own hard real-time
%% %% schedules that can be scheduled as a task in the overall system design.


%% % \todo{delete or move
%% % %\paragraph{Haskell Syntax.}

%% % To understand Copilot's syntax, it helps to understand a little Haskell syntax.
%% % In Haskell,, and the
%% % empty list is denoted {\tt []}.

%% % \begin{itemize}
%% %   \item {\tt drop 2 [6, 7, 8]} evaluates to {\tt [8]}.
%% %   \item {\tt [2, 3] ++ [6, 7]} evaluates to {\tt [2, 3, 6, 7]}.
%% % \end{itemize}

%% % \noindent

%% % \begin{quote}
%% % \begin{verbatim}
%% % let x = [0] ++ map (+ 2) y
%% %     y = x
%% % \end{verbatim}
%% % {\tt x} evaluates to {\tt [0, 2, 4, ...]}, and
%% % {\tt y} evaluates to {\tt [0, 2, 4, ...]}.
%% % \end{quote}

%% % \noindent
%% % The lists generated are infinite lists, or \emph{streams}.
%% % }


%% % \paragraph{Copilot Syntax.}

%% \subsection{Easy Fault-Tolerance} \label{sec:ft}
%% Fault-tolerance is hard to get right.  The examples given in
%% Section~\ref{sec:motivation} can be viewed as fault-tolerance algorithms that
%% failed; indeed, as noted by Rushby, fault-tolerance algorithms, ostensibly
%% designed to prevent faults, are often the source of systematic faults
%% themselves~\cite{rushbyss10}!  One goal of Copilot is to make fault-tolerance
%% easy---easy for experts to specify algorithms without having to worry about
%% low-level programming errors and easy for users of the functions to integrate
%% the algorithms into their overall system.  While Copilot cannot protect against
%% a designer using a fault-tolerant algorithm with a weak fault-model, it
%% increases the chances of getting fault-tolerance right as well as decoupling the
%% design of fault-tolerance from the primary control system.  Finally, it
%% separates the concerns of implementing a fault-tolerance algorithm from
%% implementing the algorithm as a functionally correct, memory-safe, real-time C
%% program.

%% As noted, because Copilot is deeply embedded in Haskell, Haskell acts as a meta-language
%% for manipulating Copilot programs.  For example, the streams {\tt maj}, {\tt check}, and
%% {\tt overHeat} in Figure~\ref{fig:engine} are implemented by Haskell functions
%% that generate Copilot programs.

%% %% (Stream {\tt overHeat} is defined using a
%% %% past-time linear temporal logic (ptLTL) library).  The {\tt majority} function
%% %% takes a list of streams and computes the point-wise majority element of the
%% %% streams.}

%% %% For example, consider the following stream values and the majority
%% %% values for them:
%% %% \begin{code}
%% %% s0       => 0, 1, 2, 4, ...
%% %% s1       => 0, 1, 3, 2, ...
%% %% s2       => 1, 2, 3, 4, ...
%% %% majority => 0, 1, 3, 4, ...
%% %% \end{code}
%% %% \noindent

%% To see this in more detail, consider the Boyer-Moore Majority-Vote Algorithm,
%% the most efficient algorithm for computing a majority element from a
%% set\footnote{Due to space limitations, we will not describe the algorithm here,
%%   but an illustration of the algorithm can be found at
%%   \url{http://www.cs.utexas.edu/~moore/best-ideas/mjrty/example.html}.}~\cite{mjrty}.
%% The {\tt majority} library function implements this algorithm as a Copilot macro
%% as follows: 
%% \begin{code}
%% majority (x:xs) = majority' xs x (1 :: Stream Word32)
%%   where 
%%   majority' []     candidate _   = candidate
%%   majority' (x:xs) candidate cnt = 
%%     majority' xs (if cnt == 0 then x else candidate)
%%                  (if cnt == 0 || x == candidate then cnt+1 else cnt-1)
%% \end{code}

%% %% \begin{code}
%% %% majority ls = majority' ls (const unit) 0
%% %%   where
%% %%   majority' [] candidate _       = candidate
%% %%   majority' (x:xs) candidate cnt =
%% %%     majority' xs (mux (cnt == 0) x candidate)
%% %%                  (mux (cnt == 0 || x == candidate) (cnt + 1) (cnt - 1))
%% %% \end{code}
%% \noindent
%% %% The function recursively generates a tree of {\tt if-then-else} Copilot
%% %% expressions.
%% \noindent
%% The macro specializes the algorithm for a fixed-size set of streams at compile-time to
%% ensure a constant-time implementation, even though the algorithm's
%% time-complexity is data-dependent.  (Our library function ensures sharing is
%% preserved to reduce the size of the generated expression.)


%% %% \footnote{For readability, we omit local variable bindings that
%%   %% reduce the size of the expression.  See
%%   %% \url{https://github.com/seni/copilot-libraries/blob/master/src/Copilot/Library/Voting.hs}
%%   %% for the actual implementation.}

%% As an informal performance benchmark, for the {\tt majority} algorithm voting
%% over five streams of unsigned 64-bit words, we compare C code generated from
%% Copilot and constant-time handwritten C.  Each program is compiled using {\tt
%%   gcc -O3}, with a {\tt printf} statement piped to {\tt /dev/null} (to ensure
%% the function is not optimized away).  The hand-written C code is approximately
%% nine percent faster.

%% While the Boyer-Moore algorithm is not complicated, the advantages of the
%% Copilot approach over C are (1) {\tt majority} is a polymorphic library function
%% that can be applied to arbitrary (Copilot-supported) data-types and sizes of voting sets; with
%% (2) constant-time code, which is tedious to write, is generate
%% automatically; (3) the Copilot verification and validation tools (described in
%% Section~\ref{sec:assurance}) can be used. 

%% %% We provide a fault-tolerance library; moreover, other Copilot libraries implement other
%% %% common monitor specification languages such as bounded linear temporal logic
%% %% (LTL), ptLTL, and bounded regular expressions.

%% %% \subsection{Triggers}

%% %% A trigger is a function of the form
%% %% \begin{code}
%% %% trigger b "fn" (a0 <> a1 <> ... <> an)
%% %% \end{code}
%% %% \noindent
%% %% where {\tt b} is a Copilot variable or a constant, {\tt fn} is string containing
%% %% the name of a C function, and {\tt a0 ... an} are Copilot variables or
%% %% constants, or it is of the form
%% %% \begin{code}
%% %% trigger b "fn" void
%% %% \end{code}
%% %% \noindent
%% %% if it takes no arguments.  A trigger is fired only if the Boolean {\tt b} is
%% %% true.  When it is true, the function {\tt fn} is called with the appropriate
%% %% arguments; {\tt fn} is expected to have a {\tt void} return type.




%% \subsection{Distributed Monitoring}\label{sec:distributed}
%% Our
%% case study presented in Section~\ref{sec:case-study} implements distributed monitors.  In a
%% distributed monitor architecture, monitors are replicated, with specific
%% parameters per process (e.g., process identifiers).  The meta-programming
%% techniques described in Section~\ref{sec:ft} can be used to generate distributed
%% monitors by parameterizing programs over node-specific data, reducing a tedious
%% task that is traditionally solved with makefiles and C macros to a few lines of Haskell.

%% %% This is because
%% %% the Copilot compiler, along with other tools, is just a Haskell function.

%% Copilot remains agnostic as to how the communication between distinct processes
%% occurs; the communication can be operating system supported (e.g., IPC) if the
%% monitors are processes hosted by the same operating system, or they can be raw
%% hardware communication mechanisms (e.g., a custom serial protocol and processor
%% interrupts).  If the monitors are on separate processors, the programmer needs
%% to ensure either that the hardware is synchronized (e.g., by using a shared clock
%% or by executing a clock synchronization protocol).  Regardless of the method,
%% triggers, described above, are also used to call C functions that implement the
%% platform-specific protocol.  Incoming values are obtained by sampling external
%% variables (or functions or arrays).


