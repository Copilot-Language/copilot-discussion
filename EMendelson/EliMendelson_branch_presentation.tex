% $Header$

\documentclass[xcolor={dvipsnames}]{beamer}
\usepackage{alltt}

\mode<presentation>
{
  \usetheme{Warsaw}
  % or ...

  \setbeamercovered{transparent}
  % or whatever (possibly just delete it)
}


\usepackage[english]{babel}
% or whatever

\usepackage[latin1]{inputenc}
% or whatever

\usepackage{times}
\usepackage[T1]{fontenc}
% Or whatever. Note that the encoding and the font should match. If T1
% does not look nice, try deleting the line with the fontenc.


\title[Implementing Structs \& Copilot Insights] % (optional, use only with long paper titles)
{Structuring Copilot Code}

\subtitle
{Implementing Structs and Other Insights in Copilot} % (optional)

\author[Mendelson] % (optional, use only with lots of authors)
{Eli~Mendelson}
% - Use the \inst{?} command only if the authors have different
%   affiliation.

\institute[NASA Langley Research Center] % (optional, but mostly needed)
{
  Safety-Critical Avionics: Systems Branch\\
  NASA Langley Research Center}
% - Use the \inst command only if there are several affiliations.
% - Keep it simple, no one is interested in your street address.

\date[Intern Presentation] % (optional)
{August 6, 2015 / Intern Presentation}

\subject{Intern Presentation}
% This is only inserted into the PDF information catalog. Can be left
% out. 



% If you have a file called "university-logo-filename.xxx", where xxx
% is a graphic format that can be processed by latex or pdflatex,
% resp., then you can add a logo as follows:

% \pgfdeclareimage[height=0.5cm]{university-logo}{university-logo-filename}
% \logo{\pgfuseimage{university-logo}}
\pgfdeclareimage[height=1.25cm]{nasa-logo}{NASA_logo}
\logo{\pgfuseimage{nasa-logo}}

% Delete this, if you do not want the table of contents to pop up at
% the beginning of each subsection:
\AtBeginSubsection[]
{
  \begin{frame}<beamer>{Outline}
    \tableofcontents[currentsection,currentsubsection]
  \end{frame}
}


% If you wish to uncover everything in a step-wise fashion, uncomment
% the following command: 

%\beamerdefaultoverlayspecification{<+->}


\begin{document}

\begin{frame}
  \titlepage
\end{frame}

\begin{frame}{Outline}
  \tableofcontents[pausesections]
\end{frame}


% Since this a solution template for a generic talk, very little can
% be said about how it should be structured. However, the talk length
% of between 15min and 45min and the theme suggest that you stick to
% the following rules:  

% - Exactly two or three sections (other than the summary).
% - At *most* three subsections per section.
% - Talk about 30s to 2min per frame. So there should be between about
%   15 and 30 frames, all told.

\section{Copilot Overview}

\subsection[Haskell eDSL]{High-Assurance Compilation}

\begin{frame}{Copilot Pipeline}{New Back Ends are Available (CompCert, Frama-C)}
\centering\includegraphics[width=7.0cm, height=6.0cm]{Copilot_architecture}
\end{frame}

\begin{frame}{Copilot Back Ends}
  \begin{itemize}
  \item
    All back ends compile to C99 - \alert{hard real-time} systems
    \begin{itemize}
    \item
      Atom
      \begin{itemize}
        \item
        Heavily relies on arbitrary changes of state
      \end{itemize}
    \item
      SBV
      \begin{itemize}
        \item
        Modular
        \item
        Used to \alert{implement structs}
      \end{itemize}
    \end{itemize}
  \end{itemize}
\end{frame}

\subsection[High-Assurance Monitors]{High-Assurance Meets Avionics}

\begin{frame}{Homer Got It Right}{Converting Sensory Input to Safety}
  \centering\includegraphics[width=7.5cm, height=6.0cm]{Homer_Simpson}
\end{frame}
\begin{frame}{New Back Ends}{CompCert, Frama-C}
  New back ends \alert{verify correctness} of \textbf{monitor code}
  \begin{itemize}
  \item
    CompCert
    
    \includegraphics[width=9.0cm, height=1.65cm]{CompCert_Pipeline}
  \item
    \includegraphics[width=3.75cm, height=1.5cm]{FramaC_logo}
  \end{itemize}
\end{frame}

\section{Implementing Structs}
\subsection{Coding with Structs}
\begin{frame}{What is a Struct (in C)?}
  A struct is\dots
  \begin{itemize}
  \item
    Collection of values (no singular \emph{field type})
  \item
    \alert{Contiguous block} of memory
  \item
    Type-less
  \item
    Portable
  \end{itemize}
\end{frame}
\subsection{Monitors in Avionics}
\begin{frame}{Monitors}{How Can They Be Applied to Structs?}
  We need monitors to\dots
  \begin{itemize}
  \item
    Keep track of sensory input
  \item
    Embedded systems \alert{use structs} to keep track of sensory values in a specification
  \item
    Ensure that a flight system abides by a \alert{specification}
  \end{itemize}
\end{frame}
\begin{frame}{Structs in Monitor Code}{Structuring Sensor Values}
    \begin{alltt}
    \begin{columns}
      \begin{column}{0.25\textwidth}
        \scriptsize\par{\textcolor{Peach}{uint32\_t count;\\
        float    sum;\\
        float    average;\\
        float    correction;\\
        float    algo\_erro\_check;\\
        float    min;\\
        float    max;\\
        bool     start\_sampling;\\
        bool     have\_correction;}}
      \end{column}
      \makebox[1in]{\rightarrowfill}
      \begin{column}{0.5\textwidth}
        \setlength{\parindent}{10pt}
        \scriptsize\par{\noindent{\textcolor{Emerald}{struct NeutralThrustEstimation} \{ \\}
        \textcolor{Peach}{uint32\_t count;\\
        float    sum;\\
        float    average;\\
        float    correction;\\
        float    algo\_erro\_check;\\
        float    min;\\
        float    max;\\
        bool     start\_sampling;\\
        bool     have\_correction;}\\
        \};\\
        \noindent{static \textcolor{Emerald}{struct NeutralThrustEstimation} neutralThrustEst;}}
      \end{column}
    \end{columns}
    \end{alltt}
\end{frame}

\section{Discoveries}
\subsection[(In)Compatibility]{Language (In)Compatibility \& Portability}
\begin{frame}{Haskell Meets Embedded-C}
  \begin{itemize}
  \item
    Haskell
    \begin{itemize}
    \item
      Functional language
    \item
      Easy to implement \alert{formal methods}
      \begin{itemize}
      \item
        Verification
      \item
        Model Checking
      \end{itemize}
  \end{itemize}
  \item[]
    vs.
  \item
    Embedded C
    \begin{itemize}
    \item
      \alert{Constant} time/memory
    \item
      Output C99 code similar to MISRA
    \end{itemize}
  \end{itemize}
\end{frame}
\begin{frame}{(Dis)Advantages}
  \begin{itemize}
  \item
    Goals/Uses of Haskell and Embedded-C are \alert{not aligned}
    \begin{itemize}
    \item
      Leverage the features of both languages
    \item
      Two sets of limitations
    \end{itemize}
  \item
    Compiler produces monitor code
    \begin{itemize}
    \item
      Avionics programmer \alert{must know Haskell}
    \end{itemize}
  \item
    New back ends only support subset of C99
    \begin{itemize}
    \item
      CompCert does not support long doubles
    \end{itemize}
  \end{itemize}
\end{frame}
\subsection[Struct Limitations]{Limitations of Copilot Structs}
\begin{frame}{Examples in Copilot}{Haskell Code}
  \begin{alltt}
    \small{
    \setlength{\parindent}{-10pt}
    
    
    simple :: Stream \textbf{Bool} \\
    simple = \alert{externStruct} "simple" [(\textcolor{Plum}{"example"}, arg \textcolor{Green}{example})] \\
    \setlength{\parindent}{0pt}
    where \\
    \setlength{\parindent}{10pt}
    \textcolor{Green}{example} :: Stream Word32 \\
    \textcolor{Green}{example} = 1 \\
    \setlength{\parindent}{-15pt}
    \dots\\
    \dots\\
    \dots\\
    \setlength{\parindent}{-10pt}
    run :: Stream Word32 \\
    run = simple\#\textcolor{Plum}{"example"}
    }
  \end{alltt}
\end{frame}
\begin{frame}{Examples in Copilot}{C Code (for monitor)}
  \begin{alltt}
    \begin{columns}
      \begin{column}{0.5\textwidth}
        \scriptsize\par{\textbf{.h File}\\
        \textcolor{Emerald}{struct simple} \{ \\}
        \setlength{\parindent}{10pt}
        \textcolor{Peach}{uint\_32t example;}\\
        \noindent{\};}
      \end{column}
      \begin{column}{0.5\textwidth}
        \setlength{\parindent}{10pt}
        \scriptsize\par{\noindent{\textbf{.c File}\\
        \dots\\
        \dots\\
        \textcolor{Peach}{.. simple.example ..}\\
        \dots\\
        \dots}}
      \end{column}
    \end{columns}

  \end{alltt}
\end{frame}
\begin{frame}{Semantic Problems}
  \begin{itemize}
  \item
    Structs need to be of \alert{type Bool}
  \item
    Redundant way of implementing structs in Haskell specification
    \begin{itemize}
    \item
      All fields must be declared as \texttt{Extern} variables, including structs
    \end{itemize}
  \item
    Complications surrounding \alert{typing structs} in Haskell
    \begin{itemize}
    \item
      Difficult to \alert{nest} structs
    \end{itemize}
  \end{itemize}
\end{frame}

\section*{Summary}

\begin{frame}{Summary}

  % Keep the summary *very short*.
  \begin{itemize}
  \item
    Haskell and C serve vastly \alert{different purposes} as languages
  \item
    Structs are necessary for monitors, but suffer from the disparity of Haskell and C
  \item
    Structs are not done being implemented, but \alert{significant progress} has been achieved
  \end{itemize}
  
  % The following outlook is optional.
  \vskip0pt plus.5fill
  \begin{itemize}
  \item
    Outlook
    \begin{itemize}
    \item
      \textbf{Copilot} will be very powerful as it becomes more robust.
    \item
      Extending Copilot's use to other UAV software
    \end{itemize}
  \end{itemize}
\end{frame}


\end{document}


