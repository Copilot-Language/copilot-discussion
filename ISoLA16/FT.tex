\section{Fault-Tolerant RV}~\label{sec:ft}

%\noindent\fbox{%
%    \parbox{\textwidth}{%
% Question: Under what fault model do you expect the runtime verification system to function?
%    }%
%}
 
High assurance  RV must meet the high level of reliability that is expected of
the of the system.  Often mandated by regulatory
guidelines~\cite{SAE4761}, sfety engineers employ a range of established
methods to identify hazards and failure modes~\cite{Levenson}. The level of desired
reliability determines what faults the system must be designed to
tolerate. If RV is to be the guarantor of safety, then it must be
designed to be \emph{fault-tolerant}~\cite{butler-faults}, meaning it
that continues to provide its required functionality in the presence
of faults.  A fault-tolerant system must not contain a \emph{single
  point of failure} such that if the single subsystem fails, the
entire system fails.


 Ideally, the RV and the SUO should not be subject to common
 modes of failure.  For instance, software errors in the SUO such as
 numerical overflows an memory leaks that can render a system
 inoperable should not affect the RV.  A fault tolerant system must
 be robust in the presence of hardware faults such as sensor failures
 and voltage spikes as well.  A \emph{fault-containment region} (FCR)
 is a region in a system designed to ensure faults do not propagate to
 other regions~\cite{Rushby01:buscompare}.   The 
 easiest way to ensure this is to physically isolate one FCR from
 another.  However, FCRs may need to communicate, hence they share
 channels. An FCR containing a monitor may need to share a channel
 with the SUO.  Care must be taken to ensure faults cannot propagate
 over these channels. 


\paragraph{Challenge}   \emph{The RV  should not be rendered inoperable by the
 same failure conditions that cause hazards to the SUO.}


\paragraph{Copilot Approach} Fault-tolerant RV has been a ongoing
topic of investigation for the Copilot research group. The avionics
industry has been migrating away from federated systems toward the use
of integrated modular avionics that provide fault-tolerance as a
service.  The Aeronautical Radio, Incorporated (ARINC)
653~\cite{ARINC653} compliant hard real-time operating systems RTOS
provides temporal and spatial partitioning guarantees so applications
can safely share resources.  The Copilot group has been investigating
design pattern for implementing fault-tolerant RV on such
platforms~\cite{Kaveh15}. Monitors run on the same nodes as the
software being monitored, but in separate partitions. Monitoring tasks
executing in a separate partition is done through very circumscribed
channels that preserve the isolation guarantees.  The spacial and
temporal protections provided by ARINC 653 keep the monitors safe from
other programs running on the same system, redundancy is required to
tolerate more pernicious faults.


Systems that need be ultrareliable typically have employ redundancy to
tolerate the most pernicious faults such as \emph{Byzantine fault}
(i.e., a fault in which different nodes interpret a single broadcast
message differently.  There are numerous documented incidents in the
aerospace industry where sensors failed in a Byzantine fashion with
the potential to affect a vehicle's safety.  The aircraft horizontal separation criteria in
Section~\ref{sec:req} depends on reliably sensing the position and
velocity of both systems. In earlier work~\cite{pike-isse-13}, we have
tried to address this issue in a case study where a system had
redundant and Copilot monitors performed Byzantine exchange and a
majority voting to create a system that could tolerate a single
Byzantine fault~\cite{pike-isse-13}.  Fault injection testing was
performed along with flight tests.  The hardware used in these
experiments were commodity microprocessors, but we have recently
bought Mil-Spec hardened processors that are more reliable when operating
under varying environmental conditions.


% Generally,
%physical faults in separate FCRs are statistically independent, but under
%exceptional circumstances, simultaneous faults may be observed in FCRs.  For
%example, widespread high-intensity radiation may affect multiple FCRs.


%\noindent\fbox{%
%   \parbox{\textwidth}{%
% Question: Are the monitors isolated from system under observation?
%    }%
% }
