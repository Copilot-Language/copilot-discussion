\section{Reliabiity}~\label{sec:ft}

\noindent\fbox{%
    \parbox{\textwidth}{%
 Question: Under what fault model do you expect the runtime verification system to function?
    }%
}
 
If RV is to enable IA in safety-critical systems, than the RV must
meet the high level of reliability that is expected of the of the
system. Hence we content that assured RV must be \emph{fault-tolerant}
~\cite{butler-faults}, meaning it that continues to provide its
required functionality in the presence of faults.  A fault-tolerant
system must not contain a \emph{single point of failure} such that if
the single subsystem fails, the entire system fails. 


%As part of the
%overall systems analysis, we can assume that safety analysis has been
%conducted to at least identify hazards~\cite{XX} and common mode
%failures~\cite{XXX} that can lead compromise safety. 


A \emph{fault-containment region} (FCR) is a region in a system
designed to ensure faults do not propagate to other
regions~\cite{Rushby01:buscompare}.  The easiest way to ensure this is
to physically isolate one FCR from another.  However, FCRs may
need to communicate, hence  they share channels. An FCR
containing a monitor may need to share a channel with the SUO.  Care must be taken to
ensure faults cannot propagate over these channels.  

In the course of our research, we have investigated several approaches
to creating fault-tolerant Copilot monitors.  Motivated by several
incidents arising from failures of aircraft pitot tubes, we
constructed a redundant hardware stack and implemented Copilot
libraries so as to tolerate a single Byzantine fault of the pitot
tubes~\cite{pike-isse-13}. Each node sampled an independently
redundant sensor and performed a Byzantine exchange and a majority
voting.  Fault injection testing was performed along with flight
tests.  The hardware used in these experiments were commodity
microprocessors, but we have recently bought Mil-Spec hardened
processors are far more reliability given the operational conditions.
It may not always ne necessary to execute one's monitors on separate
hardware. We recently investigated several possible architectural
patterns for implementing fault-tolerant Copilot monitors on
integrated modular avionics systems running 653~{XXXX} compliant hard
real-time operating systems~\cite{Kaveh15}. Here, monitors run on the
same nodes as the software being monitored, but 
in separate partitions that provide fault-isolation guarantees.








% Generally,
%physical faults in separate FCRs are statistically independent, but under
%exceptional circumstances, simultaneous faults may be observed in FCRs.  For
%example, widespread high-intensity radiation may affect multiple FCRs.


\noindent\fbox{%
    \parbox{\textwidth}{%
 Question: Are the monitors isolated from system under observation?
    }%
}

%\begin{itemize} 
%\item Complex safety-critical systems often have Hazard analysis,
 % FMEA, .... mandated.
%\item  While control engineers often
%  equate safety with stability, structural engineers with structural
%  integrity, and  s/w engineers with software correctness, it is
%  really a systems issues. 
%\item Each discipline's contribution is often vital to ensuring safety
%  and will provide critical evidence to a safety case that the
%  system is safe. 
%\item Given our focus on safety-critical systems we can, we can assume
%  that hazard analysis, FMEA, .... analysis are done in the
%  requirements phase.
%\begin{itemize}
%\item DO 178B 2.2.3 Monitor should not be rendered inoperable by the
%  same failure conditions that cause hazards.
%\item DO 178B 2.2.3  Dictates that the RV should be ``assigned the
%  same software level as most severe failure condition''.
%\item Assured RV for autonomous systems must respect a fault model so
%  as to not fall victim to the hazards and common mode failures
%  detected in analysis. 
%\end{itemize}
%\item Extensive safety analysis (hazard analysis, failure mode effects
%  analysis (FEMA), zonal analysis, etc. 
%\item RV should be able to handle faults allowable under fault model.
%\item Redundancy and independence as well as associated redundancy
%  management must be included in the RV system.
%\item We have investigated how to engineer fault-tolerant monitors to
%  withstand known hazards of sensor failure (pitot tube) failure. 
%\item Redundant HW stack, consensus to withstand 1 Byzantine fault. 
%\item Conducted experiments not only to demonstrate the FT, but to
%  see how the system responded when the defined safety threshold of
%  one fault is surpassed. 
%\item Monitors and steering code need to be isolated from faults in
%  the SUO. 
% \begin{itemize}
%\item We have investigated both separate/redundant RV HW stack 
%\item We have investigated how to architect the monitor system so as
%  run monitors and steering code in separate partitions on a VX-Works
%  653 compliant system. 
%\end{itemize} 
%\end{itemize} 