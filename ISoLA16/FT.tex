\section{Fault-Tolerant RV}~\label{sec:ft}

%\noindent\fbox{%
%    \parbox{\textwidth}{%
% Question: Under what fault model do you expect the runtime verification system to function?
%    }%
%}
 
Assured RV must meet the high level of reliability that is expected of
the of the system. Safety engineers employ a range of established
methods to identify hazards and failure modes. The level of desired
reliability determines what faults the system must be designed to
tolerate. If RV is to be the guarantor of safety, then it must be
designed to be \emph{fault-tolerant} ~\cite{butler-faults}, meaning it
that continues to provide its required functionality in the presence
of faults.  A fault-tolerant system must not contain a \emph{single
  point of failure} such that if the single subsystem fails, the
entire system fails.

\paragraph{Challenge}   The RV  should not be rendered inoperable by the
 same failure conditions that cause hazards to the SUO. 

 Ideally, the RV should and the SUO should not be subject to common
 model of failure.  For instance, software errors in the SUO such as
 numerical overflows an memory leaks that can render a system
 inoperable should not affect the RV.  A fault tolerant systems must
 be robust in the presence of hardware faults such as sensor failures
 and voltage spikes as well.  A \emph{fault-containment region} (FCR)
 is a region in a system designed to ensure faults do not propagate to
 other regions~\cite{Rushby01:buscompare}.   TheL
 easiest way to ensure this is to physically isolate one FCR from
 another.  However, FCRs may need to communicate, hence they share
 channels. An FCR containing a monitor may need to share a channel
 with the SUO.  Care must be taken to ensure faults cannot propagate
 over these channels.


 \paragraph{Copilot Approach} Fault-tolerant RV has been a ongoing topic of
 investigation for the Copilot research group. The avionics industry
 has been migrating away from federated systems toward the use of
 integrated modular avionics that provide fault-tolerance as a
 service.  The Aeronautical Radio, Incorporated (ARINC)
 653~\cite{ARINC653} compliant hard real-time operating systems (RTOS)
 ~\cite{Kaveh15} provides temporal and spatial partitioning guarantees
 so applications can safely share resources.  The Copilot group has
 been investigating design pattern for implementing fault-tolerant RV
 on such platforms. Monitors run on the same nodes as the software
 being monitored, but in separate partitions. Monitoring tasks
 executing in a separate partition is done through very circumscribed
 channels that preserve the isolation guarantees.  The spacial and
 temporal protections provided by ARINC 653 keep the monitors safe
 from other programs running on the same system, redundancy is
 required to tolerate more pernicious faults.


 Monitors for safety-critical cyber-physical systems typically involve
 sensor data. The aircraft horizontal separation criteria in
 Section~\cite{sec:spec} depends on reliably sensing the position and
 velocity of both systems.  In aerospace, there are numerous
 documented incidents that had the potential to impact safety.  In
 earlier work~\cite{pike-isse-13}, we have tried to address this issue
 in a case study where a system had redundant and Copilot monitors
 performed Byzantine exchange and a majority voting to create a system
 that could tolerate a single Byzantine fault~\cite{pike-isse-13}.
 Fault injection testing was performed along with flight tests.  The
 hardware used in these experiments were commodity microprocessors,
 but we have recently bought Mil-Spec hardened processors are more
 reliable when operating under varying environmental conditions.

Much more work in fault-tolerant RV remains to be done and we hope
that the broader RV community will take up the challenge.



% Generally,
%physical faults in separate FCRs are statistically independent, but under
%exceptional circumstances, simultaneous faults may be observed in FCRs.  For
%example, widespread high-intensity radiation may affect multiple FCRs.


%\noindent\fbox{%
%   \parbox{\textwidth}{%
% Question: Are the monitors isolated from system under observation?
%    }%
% }
