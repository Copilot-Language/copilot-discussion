\section{Observability}\label{sec:observ} 

Technically speaking, an RV monitor takes a logical specification 
$\phi$ and execution trace $\tau$ of state information of the SUO and
decides wheter $\tau$ satisfies $\phi$.  Ensuring observability of
both the system state and the environment as required by a
specification is one of the principal engineering challenges of RV.

  If monitoring the state of an executing program, is it possible that
  the  monitor fails to detect a state change?  It is often necessary
  to read sensor data to obtain the required   state data (e.g. aircraft
  pitch and  vehicle position)  or environmental data
  (e.g. temperature).  If it  is  raw sensor  data, do  we  apply
  filters  before feeding  the data  into  the monitors?  Is the  data
  available   in  the  same  coordinate   systems  demanded   of  the
  specification?  Can we ensure the  integrity and provenance  of the
  data  being observed?

\paragraph{Challenge} \emph{Are the state and environmental variables in the
specification  observable.}  


 
  \paragraph{Copilot Approach} 
  How an RV framework obtains state data impacts the properties that
  can be monitored.  Many of RV frameworks such as MAC~\cite{KimLKS04}
  and MOP~\cite{ChenR05} instrument the software and hardware of the
  SUO so that it emits events of interest to the monitor.  While
  attractive from the veiwpoint of maximizing state observability, the
  additional overhead may affect execution time affectiing worst case
  execution times and consequently the scheduling; in addition
  regulatory guidelines may require recertificaiton of that software.
  Copilot and several other RV
  frameworks~\cite{sampling,Kane15,borzoo} opt to sacrifice complete
  observability by sampling events.  Copilot monitors run as dedicated
  thread and sample program variables and state information via shared
  memory.  Currently, we rely on scheduling analysis with a good
  amount of experimentation to ensure that we sample values at a
  sufficient rate that specification violations are detected. This has
  been very successful when the implementation platform is running a
  real-time operating system (RTOS) with deterministic scheduling
  guarantees, but we cannot make strong guarantees running on less
  specialized systems.

  The aircraft safe separation criteria specification introduced in
  Section~\ref{sec:req} requires the monitor observe state data for
  both the aircraft the monitor is executing on as well as state for
  the ``intruder'' aircraft which is obtained by
  communication. Meaning monitors required sampling data from
  executing programs (planned maneuver), onboard positioning sensors,
  and data broadcast from other vehicles.

\paragraph{Opportunities}  There are ample opportunities for targeting
static analysis tools. At times it can be a daunting task to simply
ascertain if the values in the specification are present in the act If the RV approach involves
instrumenting code, then static analysis can both assist in the
instrumentation and prove that the instrumentation did not affect the
correctness of the code. If sampling, static analysis has the
potential to inform when
to sample. 
  

%\begin{itemize} 
%\item Does the monitor possess access to the necessary sensor data to
 % monitor state of the SUO?
%\item Do those sensors even exist?
%\item Need to preform traditional signal processing (filtering etc)
%\item  If sampling:
%\begin{itemize}
%\item Can you estimate missed values
%\item Can you figure out when to sample.
%\end{itemize}
%\item static analysis tools can help with code instrumentation 
%\item static analysis can help with determining when to sample based
 % on decision points in code. etc timing analysis for hard realtime
 % systems.
%end{itemize} 
