\section{Observability}\label{sec:observ} 
Guaranteeing that all the data required by the specification is
actually \emph{observable} is one of the principal engineering
challenges of RV. In embedded systems, the RV specification often
involves data from a number of different types of data sources,
including state data of executing programs, sensor data, as well as
data that is communicated from other systems.  The systems safety
properties of cyber-physical systems are often formulated by aerospace
and automobile engineers that are domain experts, but can have
varying degrees of understanding of the computing systems so the RV
engineer needs to be very proactive in addressing the observability
issue. In embedded systems, the closed nature of the hardware
platforms and propriety issues can make it impossible to observe the
information required in the specification.  Additional sensors may be
needed or communication data formats changed.  At times it is
necessary to change the specification so that it only depends on
observable data.  The observability issue may seem like an
``engineering detail'', but based on our experience it is often a
significant obstacle resulting in  delays, frustration, and sometimes
preventing progress altogether.


\paragraph{Challenge:} \emph{Are the state and environmental variables in the
specification  observable?}  


  \paragraph{Copilot Approach:} How an RV framework obtains state data impacts the properties that
  can be monitored. Many of RV frameworks such as MAC~\cite{KimLKS04}
  and MOP~\cite{ChenR05} instrument the software and hardware of the
  SUO so that it emits events of interest to the monitor.  While
  attractive from the viewpoint of maximizing state observability, the
  additional overhead may affect execution time affectiing worst case
  execution times and consequently the scheduling; 
  regulatory guidelines may  also require recertification of that software.
  Copilot, and several other RV frameworks~\cite{sampling,Kane15,borzoo}, opt to sacrifice complete
  observability by sampling events.  Copilot monitors run as dedicated
  thread and sample program variables and state information via shared
  memory.  Currently, we rely on scheduling analysis with a good
  amount of experimentation to ensure that we sample values at a
  sufficient rate that specification violations are detected. This has
  been very successful when the implementation platform is running a
  real-time operating system (RTOS) with deterministic scheduling
  guarantees, but we cannot make strong guarantees running on less
  specialized systems.

A critical lesson learned in the course of conducting many case studies
is to ask questions about observability early and often. 
If monitoring the state of an executing program, is it possible that
the monitor fails to detect a state change?  It is often necessary to
read sensor data to obtain the required state data (e.g. aircraft
pitch and vehicle position) or environmental data (e.g. temperature).
If it is raw sensor data, do we apply filters before feeding the data
into the monitors?  Is the data available in the same coordinate
systems demanded of the specification?  Can we ensure the integrity
and provenance of the data being observed? 


  The aircraft safe separation criteria specification introduced in
  Section~\ref{sec:req} requires the monitor to  observe state data for
  both the aircraft the monitor is executing on as well as the
  ``intruder'' aircraft.  Hence the monitors must sample data from
  executing programs (planned maneuver), onboard positioning sensors,
  and data broadcast from other vehicles.  For our experiments, the
  aircraft position data had to be converted from World Geodetic
  System latitude and longitude to ECEF.


%\item Does the monitor possess access to the necessary sensor data to
 % monitor state of the SUO?
%\item Do those sensors even exist?
%\item Need to preform traditional signal processing (filtering etc)
%\item  If sampling:
%\begin{itemize}
%\item Can you estimate missed values
%\item Can you figure out when to sample.
%\end{itemize}
%\item static analysis tools can help with code instrumentation 
%\item static analysis can help with determining when to sample based
 % on decision points in code. etc timing analysis for hard realtime
 % systems.
%end{itemize} 
