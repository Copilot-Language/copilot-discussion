\section{Introduction}~\label{sec:intro} 

The consequences of failure of safety-critical systems such as nuclear
reactor shutdown systems, railway switching systems, and civil air
transportation are considered especially dire. The prescribe safety
margin of a catastrophic fault for avionics in civil aircraft is
$10^{-9}$ or one in a billion.  The justification for the requirement
is to show that failures resulting in a catastrophic effect are ``so
unlikely that it is not anticipated to occur during the operational
life of an entire system or fleet''\cite{FAA2000}.  The safety record
in civil aviation has been so outstanding the last several decades
that it is often considered the gold standard for engineering safe and
reliable systems. Strict certification regime imposed on hardware and
software as well as licensing requirements on human operators in the
system is a significant factor in the .  New technologies can take
many years to transition onto civil aviation due to the need to
demonstrate that safety is not being negatively impacted. Yet
technological advances are enabling the development of increasingly
autonomous (IA) unmanned aircraft systems (UAS) that modify their
behavior in response to the external environment and learn from their
experience.  In practice, the certification of avionics software
relies heavily on it being predicable and existing standards do not
have provisions to assure systems whose behavior is not predicable
during certification.  The US National Academies study ``Autonomy
Research for Civil Aviation''~\cite{NRC14} identifies the verification
and validation of IA systems as considerable hurdle to the adaption.


% Federal Aviation Administration (FAA) regulations
%govern the certification of aircraft and engines including the
%software. 
%Strict government regulations
%govern everything from the certification of aircraft to the management
%of the airspace . 

Runtime verification (RV), where monitors detect and respond to
property violations at runtime, has the potential to enable the safe
operation of safety-critical systems that are too complex to formally
verify or fully test using conventional test such as IA systems. The
\emph{Simplex Architecture}~\cite{simplex} provides an architectural
pattern. A monitor checks that the executing system under observation
(SUO) satisfies a specification and If the property is violated, the
RV system will the switch to system to the control of a more
conservative component that can be assured using conventional means
that \emph{steers} the system into a safe state.   


Since 2007, my colleague Lee Pike and I, with considerable help for a
bevy of talent students, have been working on program aimed at
creating a framework for \emph{high assurance RV} of safety-critical
hard real-time systems primarily in the context of the Copilot RV
framework.  In order to be used in ultra-critical environments,
high-assurance RV must:
\begin{enumerate}
\item \label{req:a} Provide evidence for a safety case that the RV
  enforces safety guarantees.
\item \label{req:d}  Be reliable in the presence of faults.  
\item \label{req:b} Support verification that the specification of the monitors
  is correct.
\item \label{req:c} Ensure that monitor code generated implements the specification of the
monitor.
\end{enumerate} 
In addition to building the Copilot framework, we have
been conducting a number of case studies.


\paragraph{Contributions} 
A considerable challenge his been discovering what were the right
questions to ask for assured RV.  This  paper is organized around a
series of questions that we feel anyone trying to apply RV in
ultracrtical IA systems must address. We will discuss how these
questions have been addressed in the context of Copilot. While far
from being a complete, we  believe
that these questions can aide others in the RV community in building
their own frameworks or conducting case studies. In keeping with the
theme of the workshop, we will highlight our use of static analysis
tools in the hope that we will not only encourage others in the
community to apply them in their own RV frameworks, but encourage the
static analysis community to collaborate with the RV researchers in
creating tools that will enable assured RV.

%Although building a safety
%case~\ciite{Kelly98arguingsafety} in the spirit
%of Rushby's~\cite{rvRushby,RushbyAIAA09} remains a challenge,  we
%believe that 
 

%To achieve this, the
%system must be architected so that the RV component can observe the
%state of the system  and prevent the autonomous decision making form
%taking unsafe actions. 





 

