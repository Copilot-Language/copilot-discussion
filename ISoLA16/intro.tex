\section{Introduction}~\label{sec:intro} 
Safety-critical systems, such as aircraft, automobiles, and medical
devices, are those systems whose failure could result
in loss of life, significant property damage, or damage to the
environment~\cite{Knight2002}.  The  grave consequences of failure have compelled
industry and regulatory authorities  to adapt conservative design
approaches and exhaustive verification and validation (V\&V) procedures
to prevent mishaps. In addition, strict licensing requirements are
often demanded on human operators of  many safety-critical systems.
In practice, the verification and validation of avionics
and other safety-critical software systems relies heavily on it being
predicable and existing regulatory guidance  such as
DO-178~\cite{DO178B}  do not  have provisions to
assure safety-critical systems that are not predictable at certification. 
Yet technological advances are enabling the development of increasingly
autonomous (IA) cyber-physical  systems that modify their behavior in response
to the external environment and learn from their experience.  While
unmanned aircraft systems (UAS) and self-driving cars have the
potential of transforming society in many beneficial ways, they also
pose new dangers to public safety. The algorithmic methods such as
machine learning that enable autonomy lack the salient feature of
being predictable since the system's behavior depends on what it has
learned.  Consequently, the  problem  of assuring safety-critical IA is
both a barrier to industrial use of  as well as a 
systems is a considerable research challenge~\cite{NRC14} .


\emph{Runtime verification} (RV)~\cite{monitors}, where monitors
detect and respond to property violations at runtime, has the
potential to enable the safe operation of safety-critical systems that
are too complex to formally verify or fully test.  Technically
speaking, an RV monitor takes a logical specification $\phi$ and
execution trace $\tau$ of state information of the system under
observation (SUO) and decides wheter $\tau$ satisfies $\phi$. The
\emph{Simplex Architecture}~\cite{simplex} provides a model
architectural pattern for RV, where a monitor checks that the
executing SUO satisfies a specification and If the property is
violated, the RV system will switch control to a more conservative
component that can be assured using conventional means that
\emph{steers} the system into a safe state.

\emph{High-assurance} RV provides an assured level of safety even when
the SUO itself cannot be verified  by conventional
means. Copilot~\cite{copilot, pike-isse-13} is an RV framework
targeted at safety-critical hard real-realtime system, which we have
recently applied to our program of high-assurance RV.  


\paragraph{Contributions:} During the course of our research on we
have been guided by the general question of ``what will take to
convince someone in authority to use high-assurance RV so safeguard a
system that cannot be otherwise assured.''  While we are a long way
from the point of convincing a regulatory authority, we believe we
have identified a number of challenges impede actualizing
high-assurance RV.  We discuss how we have approached these challenge
in Copilot with the help of a running example. A general theme of our
work is the application of lightweight formal methods to achieve
high-assurance.  Safety is a systems property so the scope of the
challenge is necessarily broad ranging from how one arrives at the
specification to be monitored to necessary capabilities the RV
framework must meet. We will highlight many opportunities for applying
light-weight formal methods tools to high-assurance RV in hopes of
fostering greater collaboration between the RV and tool communities.

