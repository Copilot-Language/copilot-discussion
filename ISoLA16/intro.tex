\section{Introduction}~\label{sec:intro} 
Safety-critical systems, such as aircraft, automobiles, and medical
devices, are those systems whose failure could result
in loss of life, significant property damage, or damage to the
environment~\cite{Knight2002}.  The  grave consequences of failure have compelled
industry and regulatory authorities  to adapt conservative design
approaches and exhaustive verification and validation (V\&V) procedures
to prevent mishaps. In addition, strict licensing requirements are
often demanded on human operators of  many safety-critical systems.
In practice, the verification and validation of avionics
and other safety-critical software systems relies heavily on it being
predicable and existing regulatory guidance  such as
DO-178~\cite{DO178B}  do not  have provisions to
assure safety-critical systems that are not predictable at certification. 
Yet technological advances are enabling the development of increasingly
autonomous (IA) cyber-physical  that modify their behavior in response
to the external environment and learn from their experience.  While
unmanned aircraft systems (UAS) and self-driving cars have the
potential of transforming society in many beneficial ways, they also
pose new dangers to public safety. The algorithmic methods such as
machine learning that enable autonomy lack the salient feature of
being predictable since the system's behavior depends on what it has
learned.  Consequently, the  problem  of assuring safety-critical IA is
both a barrier to industrial use of  as well as a 
systems is a considerable research challenge~\cite{NRC14} .


\emph{Runtime verification} (RV)~\cite{monitors}, where monitors
detect and respond to property violations at runtime, has the
potential to enable the safe operation of safety-critical systems that
are too complex to formally verify or fully test.  Technically
speaking, an RV monitor takes a logical specification $\phi$ and
execution trace $\tau$ of state information of the SUO and decides
wheter $\tau$ satisfies $\phi$. The \emph{Simplex
  Architecture}~\cite{simplex} provides an architectural pattern for
RV. A monitor checks that the executing system under observation (SUO)
satisfies a specification and If the property is violated, the RV
system will the switch to system to the control of a more conservative
component that can be assured using conventional means that
\emph{steers} the system into a safe state. Although there have been
considerable progress in the RV community, there are considerable
barriers to realizing RV that is trusted to safeguard safety-critical
system.

Copilot~\cite{copilot, pike-isse-13}  is an RV framework  targeted at
safety-critical hard real-realtime systems. Recently, Copilot research
has  focused on a program on \emph{high-assurance} RV targeting IA
applications that cannot be assured by conventional means.


\paragraph{Contributions'} We will present a number of challenges to
achieving high-assurance RV that have been identified during the
course of our investigation.  When appropriate, we will survey how we
have approached these challenge in Copilot.  A running example is  taken
from our research on using RV to ensure that aircraft  maintain safe
distance of separation. An general theme of our
work is the application of lightweight formal methods to achieve
high-assurance.  Safety is a systems property so the scope of the  challenge
is necessarily broad ranging from how one arrives at the specification
to be monitored to necessary capabilities the  RV framework must meet.
RV components are considerably simpler than the system as a whole and
consequently we have found static analysis and other light-weight
formal methods tools can play a very effective role in high assurance
RV.  We will highlight many opportunities for applying light-weight
formal methods tools to high-assurance RV in hopes of fostering
greater collaboration between the RV and tool communities.
