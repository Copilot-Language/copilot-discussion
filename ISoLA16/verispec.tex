\section{Monitor Specification Correctness}~\label{sec:verispec}

As RV matures it will be employed to
verify increasingly complex properties such as checking complex
stability properties of a control system or ensuring that a critical
system is fault-tolerant. As RV is applied to more complex systems, the
monitors themselves will become increasingly sophisticated and as prone to error
as the system being monitored.  Applying formal verification
tools to the monitors to ensure they are correct can help safeguard
that the last line of defense is actually effective. 

\paragraph{Challenge}  \emph{Assuring the correctness of the monitor
specification.}   

Ideally, specification verification capabilities should be integrated
into the RV framework so engineers could write specifications, verify
their correctness, and generate monitors.  Proof tools based on
Satisfiability Modulo Theories (\textsc{smt}) and similar advanced
automated proof techniques have the potential to be effective at
proving properties about specifications. 


\paragraph{Copilot Approach}  
 Copilot supports automated proofs of properties of specification
 properties  through its {\tt Copilot.Theorem} module.  
``synchronous observer'' approach~\cite{amast93}, properties \emph{about}
Copilot programs are specified \emph{within} Copilot itself. In particular,
properties are encoded with standard Boolean streams and Copilot streams are
sufficient to encode past-time linear temporal logic~\cite{ptltl}.

 A proposition is a Copilot value of type \texttt{Prop Existential}
 or \texttt{Prop Universal}, which can be introduced using \texttt{exists} and
\texttt{forall}, respectively. These are functions taking as an argument a
normal Copilot stream of type \lstinline{Stream Bool}. Propositions can be added
to a specification using the \texttt{prop} and \texttt{theorem} functions,
where \texttt{theorem} must also be passed a tactic for automatically proving
the proposition. The Copilot prover was first introduced
in~\cite{pike-rv-15}, where its utility was demonstrated in assuring
notoriously subtle voting algorithms to be used for fault-tolerant RV.

In the course of the analysis of the separation criteria, a team of
NASA domain experts and mathematicians used the PVS interactive prover
to prove theorems that characterize the correctness of the
criteria. Some these proofs required extensive expert human
interaction to provide mathematical intuition that remains beyond the
scope of fully automated proof techniques. As often is the case, many
of the properties were not so mathematically sophisticated.  We were
able to apply the Copilot prover  using Z3~\cite{XXX}  to prove
many of the criteria correctness properties within the Copilot
framework. Among the properties proved about the horizontal separation
criteria are

\begin{eqnarray*}
\texttt{horiz\_criteria}(\bm{sx},  \bm{sy},
\epsilon,  \bm{vx},  \bm{vy}) & \Longleftrightarrow &
\texttt{horiz\_criteria}(\bm{-sx},  \bm{-sy},
\epsilon, \bm{-vx},  \bm{-vy})   \\
( \texttt{horiz\_criteria}(\bm{sx},\bm{sy},
\epsilon, \bm{vx},\bm{vy}) ) \wedge &&  \\ 
\texttt{horiz\_criteria} (\bm{sx},  \bm{sy},
\epsilon,  \bm{wx}, \bm{wy})& \Longrightarrow&
\texttt{horiz\_criteria} (\bm{sx}, \bm{sy},
\epsilon,\bm{vx},\bm{vy})\\
( \texttt{horiz\_criteria}(\bm{sx},  \bm{sy},
\epsilon,  \bm{vx},  \bm{vy})  \wedge  &&\\
\texttt{horiz\_criteria}(\bm{sx},  \bm{sy},
\epsilon',  \bm{vx},  \bm{vy}) )  &\Longrightarrow & \epsilon = \epsilon'
\end{eqnarray*}

\paragraph{Opportunities} RV monitors offer a rich target for
researchers building proof engines and static analysis tools.  In
spite of tremendous improvements in SMT solvers, the continuous math
used in cyber-physical systems .e have not yet found an automated
proof engine that can prove properties of formula with trigonometric
functions. Static analysis techniques, such as abstract interpretation,
could be applied to show a floating point error might arise before the
monitor is even generated. Finally, there are opportunities for
investigating integrating interactive provers in the RV framework. 

%\begin{lstlisting}[frame=single]
%\lstinine{ horizont(sx, sy, e, vx vy) 
% <==>  horizontalCriterionForConflictResolution -sx -sy e -v'x -v'y} 
%\end{lstlisting}
%
%and
%
%\begin{lstlisting}[frame=single]
%\lstinine{ horizontalCriterionForConflictResolution sx sy e v'x v'y
%  &&  horizontalCriterionForConflictResolution sx sy e w'x w'y ==>
%  horizontalCriterionForConflictResolution sx sy e (v'x+w'x) (v'y + w'y)} 
%\end{lstlisting}




%\begin{itemize}
%\item Monitor specs are safety properties. 
%\item Spec needs to be validated against the requirements. Maybe spec
%  language needs to be closer to  language of domain.  Copilot
%  libraries. 
%\item Many LTL specs may be very strait forward, but we found that the
%  specs in FT and airspace management were subtle and easy to get
%  wrong. 
%\item Modern verification tools such as SMT solvers can be integrated
%% with the RV framework. 
%\item An example from Airspace management. 
%\end{itemize} 