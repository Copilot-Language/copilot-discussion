\section{Requirements}~\label{sec:req}
\noindent\fbox{%
    \parbox{\textwidth}{%

Question: Do your monitor specifications derive from system level
requirements and safety analyses? 
    }%
}

Sound systems engineering practices as well as regulatory guidelines
mandate ultra-critical systems have detailed written requirements and
as well as a through safety assessment.  Building a safety
case~\cite{Kelly98arguingsafety} for any ultra-critical system is a 
challenge. Constructing a safety case for  IA  systems  depend on RV to ensure
safety is still the subject of research.  Although the work is  investigations are in
the spirit of Rushby's
proposal~\cite{rvRushby,RushbyAIAA09}  for runtime certification. It
is crucial that the monitor specification be derived from the safety
analyses in the safety case.  A good safety case should fully enumerate the
set of assumptions on which safe operation of the system rests and the
monitor needs to test for the violation of these assumptions as well
as other safety criteria. 

While the Copilot project has not performed a case study on an
industrial scale IA airborne system, we have been collaborating
closely with NASA researchers working on new concepts that will enable
aircraft to perform autonomous flight by self-optimizing their
four-dimensional trajectories while conforming to constraints such as
required times of arrival  generated by air-traffic service
providers on the ground. Proposals being investigated  use genetic
algorithms~\cite{KarrVRC} and other concepts for artificial
intelligence and do not always behave with the predictability of
conventional systems. While these algorithms may greatly improve the
efficiency of the airspace it is not possible to ensure that safe
separation among aircraft is maintained. Ensuring  that these algorithms do
not compromise the safety of the aircraft appeared to us to be an
ideal case study for assured RV.  Failure to get  the right  monitor
specification and articulate the assumptions could have catastrophic
consequences.  Fortuitously, colleagues at NASA have discovered analytical
formula,  called \emph{criteria} that characterize those resolution
maneuvers that both  ensure safe separation when one aircraft maneuvers
and ensures separation when two conflicting aircraft both
maneuver~\cite{NMH14ATIO}. These are expressed as a set of formula involving
the position and velocity vectors of the two aircraft. Researchers
have also  enumerated a number of assumptions that must hold for the formula
to hold. The criteria
have been extensively validated through simulation as well as
mathematical proof using the Prototype Verification System (PVS)
theorem prover. 

 Working with domain experts can be a challenge especially in
 ultracritical areas such as aerospace.  We computer scientists
 typically lack intuition in the physical sciences and the domain
 experts are often equally frustrated when confronted with the
 discrete nature of the computing systems. Yet realistic case studies
 are necessary to mature assured RV in order to apply it to the  IA
 systems that are bound to be out there. In a small attempt to bridge
 the gap, between RV and the cyber-physical domain experts we
 are currently  constructing libraries to make it easier to write
 specifications involving matrices and vectors.  

 








%\begin{itemize}
%\item Safety-Critical that that are considered ultra-critical systems
% typically have detailed requirements

%\item established that only ``safety properties'' can be checked with
%  RV 
% \item For adaptive systems where RV is the entrusted repository of
%  predictability and guarantor of safety we must make sure that we are
%  actually checking properties based on formal requirements.
% \item Need to make sure that the RV safety properties are derived
%  from requirements. 
% \item Criteria based separation example an excellent example in that
%  has undergone extensive analysis both simulation and formal
%  analytical proof.  Assumptions are well spelled out. 
% \item NASA is investigating sophisticated AI based  flight management
%  software. Such systems cannot be certified by traditional means, but
%  the criteria can an can serve as the basis for RV. 
%\item EDSL we strive to build libraries allow specification in this
%  sort of notation. 
%\end{itemize}


