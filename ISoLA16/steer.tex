\section{Steering} \label{sec:steer}

\noindent\fbox{%
    \parbox{\textwidth}{
Challenge:   When the monitor detects a problem, does the RV system steer the system so as to preserve safety. 
    }
}

The problem of what to do when a specification has been violated is
one of the most thorny problems in RV and almost completely
application dependent.  Most case studies the author has seen simply
log the violation or raise an alarm for humans to intervene. This has
been the case in our Copilot case studies. The the case of an
autonomous robot there putting the system into a quiescent state may
be a safe default operation depending on the operating environment.
The research aircraft conflict detection mentioned above has produced
an algorithm for generating resolution maneuvers that would
reestablish safe aircraft separation.  How to build a safety case that
would allow this to be used for steering UAS remains an open question.
Our experience in participating in UAS flight tests illustrate how
difficult the steering problem can be.  Under certain conditions,
trained human operators sometime bring down a UAS in a controlled
within the designated zone of operations to prevent an aircraft
causing harm.  Automating this might seem as simple as killing power
to the engines; yet aircraft can glide for some distance. The steering
software may be able to direct the control surfaces to direct the
vehicle into the ground, but consideration of aircraft dynamics is
likely needed to close this can be safely done to the boundary of the
operating zone.  In many domains, the challenge in constructing an
assured safe steering algorithm may be as difficult as constructing
the adaptive autonomous algorithm.
