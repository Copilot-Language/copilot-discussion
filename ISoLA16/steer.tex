\section{Additional Challanges }\label{sec:future} 
The presentation so far has examined challenges in assured RV that
have been addressed in our research. In this section, we will
raise three of the key additional challenges that we have identified
as critical to address in future work.

\paragraph{Safe Steering:} The problem of what to do when a
specification has been violated is one of the most thorny problems in
high-assurance RV and almost completely application dependent. The
simplest action is to log the violation for further analysis or raise
an alarm for humans to intervene, but in many cases the RV system must
take proactive steps to preserve safety. For an autonomous robot,
putting the system into a quiescent state may be a safe default
operation depending on the operating environment.  In the case of an
adaptive control system, the RV framework may switch to a conventional
controller, but whether this re-establishes safety depends on many
factors.  In many domains, the challenge in constructing an assured
safe steering algorithm may be as difficult as constructing the
adaptive autonomous algorithm itself.

\paragraph{Challenge:}  \emph{Verify the correctness and safety of the
steering performed  from any viable system state once a specification violation is detected.}
 

\paragraph{Predictive Monitors:} Applying RV to application domains
that have strict timing constraints, such as an adaptive control
system, raises many technical challenges.  It is imperative that the
monitor detects that the adaptive controller is about to lose
stability in time to switch to a safe controller. In the case of our
running example, the RV system needs to detect that two aircraft are
about to lose separation in time for them to take corrective action.
Assured predictive monitors are needed, but much work remains to be
done.  Johnson et. al.~\cite{johnson2015tecs} is a promising approach
to predictive monitoring for controllers, but the general problem is
very domain specific. Assured predictive monitoring remains a research
challenge for the RV community.

\paragraph{Challenge:} \emph{The monitor should detect impending violations of
  the specification and invoke the safety controller in time to
  preserve safe operation.}


\paragraph{Secured RV:} Adding more software or hardware to a system
has the potential to introduce vulnerabilities that that can be
exploited by an attacker.  Copilot and many other RV frameworks
generate monitors that have constant time and constant space execution
footprints and while this eliminates some common attacks it does not
provide general protection. Every sensor and unauthenticated message
may contribute to the attack surface. If an attacker can trick the
systems into a monitor specification violation, the triggered steering
behavior may itself constitute a denial-of-service attack.  Future
research is needed in identifying attack surfaces and suitable
techniques to thwart adversaries from turning the RV meant to protect
a system into a liability that exposes the system to attack.

\paragraph{Challenge:} \emph{Assured RV should not introduce security
vulnerabilities into a system.} 



%Our experience in participating in UAS flight tests
%illustrate how difficult the steering problem can be as simply
%directing the vehicle into the ground, so that it does not leave a
%test range, is require the need to consider aircraft dynamics. 