\section|Future Work}\label{sec:future} 
We  so far examined challenges and opportunities in assured  RV that
have been addressed in our own research. In this section, we will
raise  three of the key additional  challenges that we believe need to be
addressed in future work. 


\paragraph{Challenge:}  Verifying the correctness and safety of the
steering performed once a specification violation is detected. 
  
The problem of what to do when a specification has been violated is
one of the most thorny problems in RV and almost completely
application dependent.  Most case studies the author has seen simply
log the violation or raise an alarm for humans to intervene. This has
been the case in our Copilot case studies. The the case of an
autonomous robot there putting the system into a quiescent state may
be a safe default operation depending on the operating environment.
The research aircraft conflict detection mentioned above has produced
an algorithm for generating resolution maneuvers that would
reestablish safe aircraft separation. Our experience in participating
in UAS flight tests illustrate how difficult the steering problem can
be. Automating this might seem as simple as killing power to the
engines; yet aircraft can glide for some distance. The steering
software may be able to direct the control surfaces to direct the
vehicle into the ground, but consideration of aircraft dynamics is
likely needed to close this can be safely done to the boundary of the
operating zone.  In many domains, the challenge in constructing an
assured safe steering algorithm may be as difficult as constructing
the adaptive autonomous algorithm.

\paragraph{Challenge} 

Complex  real-time cyber-physical systems  are a natural target for
RV.  Applying RV to  functions such as an  adaptive control system
that have to act in strict time windows posses  technical
challenges. The monitor needs to detect  that the adaptive
controller is about to loose stability in time to  switch to a safe
controller. In the case of aircraft flight management, the RV system
needs to detect that  two airplanes are about to lose separation in
time for the aircraft to take corrective action.   Assured predictive
monitors are needed, but much work remains to be done. A promising
approach to monitoring controllers against a loss of stability is
given in~\cite{}. Researchers using Copilot have shown how  timing
analysis and  hard real-time scheduling of monitors can ensure that
monitors are executed at a set time.  Assured predictive monitoring
remains a research challenge for the RV community.



\paragraph{Challenge} Assured RV should not introduce security
vulnerabilities into a system. 

Adding more software or  hardware to a system  has the potential to  
introduce vulnerabilities that that can be exploited by  an
attacker.  Copilot and many other RV frameworks generate monitors that
have constant time and constant space execution footprints and
while this eliminates some common attacks it does not provide general
protection. If an attacker can trick the systems into a monitor
specification violation, the triggered steering behavior may itself
constitute an attack.  Future research is needed in  Identifying attack surfaces
and suitable techniques to thwart adversaries from turning the RV
meant to protect a system into a liability that exposes the system to
attack.



