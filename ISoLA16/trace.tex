\section{Traceability}~\label{sec:trace}
 To ensure that the requirements and safety analyses performed early in
systems development are reflected throughout the lifecycle, many
guidelines for safety-critical software, such as DO-178C, require
documentation of traceability from requirements to object
code. Consequently, to promote the acceptance of high-assurance RV the
monitor generation frameworks should produce documentation that
supports traceability from specification to monitor code.

 \paragraph{Challenge:} \emph{Support  traceability from the requirements  and system level analysis to the actual monitor code.}

 \paragraph{Copilot Approach:} Using SBV to generate C monitors may
 create many small files and it can be quite difficult to relate this to
 the specification. The code generation module has recently been
 revised to generate documentation that improves traceability. The
 user can insert labels in their specifications that flow down to the
 documentation. The translation process creates C header files with
 documentation formatted to be processed by the plain text graph
 description language processor DOT~\cite{dot}. Each C function has
 accompanying auto-generated graphical documentation.

 In the case of the following example:
\begin{lstlisting}[frame=single]
hor_rr sx sy  = (label "?hor_rr_dividend" $ sqrt $ (normsq2dim sx sy) - (minHorSep * minHorSep)) / (label "?hor_rr_divisor" $ minHorSep)
\end{lstlisting}  
the SBV translation breaks this relatively simple expression into
numerous small C functions and function parameters get instantiated
with the variables being monitored.  The auto-generated documentation for one
of these files  appears similar to   Figure~\ref{fig:trace2}, where the labels have
the names of the program variables being monitored. 


%\begin{figure}[ht!]
%        \centering
%        \resizebox {0.4\textwidth} {!} {
%\begin{tikzpicture}[->, node distance=1.5cm, auto, shorten >=1pt, bend angle=45 ]
%\tikzstyle{every state}=[rectangle]
%\node[state] (fil1) {ext\_ident\_double\_10\_arg0};
%\node[state](la)[below of = fil1]{label:hor\_rr\_dividend};  
%\node[state](sqrt) [below of = la]{ext\_sqrt9};
%        \tikzstyle{every node}=[]
%        \path 
%        (fil1) edge [->, anchor=east] node {} (la)
%        (la) edge [->, anchor=east] node  {}(sqrt);
%        \end{tikzpicture}
%}
%        \caption{Autogenerated }
%       \label{fig:trace}%        \end{figure}

\begin{figure}[ht!]
        \centering
        \resizebox {0.4\textwidth} {!} {
\begin{tikzpicture}[->, node distance=2.3cm, auto, shorten >=1pt, bend angle=45 ]
\tikzstyle{every state}=[rectangle]
\node[state] (fil2) {ext\_sqrt};
\node[state](opmin)[right of = fil2]{OP2:-};  
\node[state](v1)[below left of = opmin]{ext\_ident\_double\_8};  
\node[state](opmul)[below right of = opmin]{OP2:*};
\node[state](mv1)[below left of =opmul]{ext\_min\_hor\_sep}; 
\node[state](mv2)[below right of =opmul]{ext\_min\_hor\_sep}; 
 
        \tikzstyle{every node}=[]
        \path 
        (fil2) edge [->, anchor=east] node {} (opmin)
        (opmin) edge [->, anchor=east] node  {}(v1)
        (opmin) edge [->, anchor=west] node  {}(opmul)
        (opmul) edge [->, anchor=east] node  {}(mv1)
        (opmul) edge [->, anchor=west] node  {}(mv2);
        \end{tikzpicture}
}
        \caption{Autogenerated Documentation }
        \label{fig:trace2}
        \end{figure}


%\begin{figure}[ht]
%\begin{center}
%\includegraphics[width=\textwidth,height=10cm]{rrfig2.pdf}
% \label{fig:label}
%\end{center}
%\end{figure}



%\begin{itemize}
%\item Many conventional processes based  guidance
% documents require ``traceability'' from the requirements to the
%  executable. 
%\item In assured RV we should be able to trace from requirements to
%  monitor spec to monitor. 
%\item We have developed a feature for adding labels to Copilot specs
 %  facilitating traceability.  
%\item  We leveraged the GNU dot tool to help visualize what is going
%  on. 

%\end{itemize}
