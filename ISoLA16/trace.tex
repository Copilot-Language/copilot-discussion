\section{Traceability}~\label{sec:trace}

Safety is a systems property and most of the analysis ensuring a
design is safe is conducted during requirements and design.  To ensure
the safety analysis is carried through the design and development
process many guidelines for safety-critical software, such as DO-178C,
requires documentation of traceability from requirements to object
code. While industry has well established means for establishing
traceability from source code to object code.  To facilitate the
acceptance of RV in these areas, RV frameworks should facilitate
traceability from the monitor to the specification.

 \paragraph{Challenge} 
To  traceability from the reqirememts and system level analysis to the actual monitor code. 

\paragraph{Example} Copilot's monitor generation process can create many small C
files. The code generation module has recently been revised to
generate code that is more readable and traceable.  We require the
user to insert labels in their specifications. The translation process
creates C header files with documentation formatted to be processed by
the plan text graph description language processor  DOT~\cite{ZZZZ}
giving  visual representation of the code  that includes the labels
making it easy to determine the correspondence between the monitor and 
specification. 

 
%\begin{itemize}
%\item Many conventional processes based  guidance
% documents require ``traceability'' from the requirements to the
%  executable. 
%\item In assured RV we should be able to trace from requirements to
%  monitor spec to monitor. 
%\item We have developed a feature for adding labels to Copilot specs
%  facilitating traceability.  
%\item  We leveraged the GNU dot tool to help visualize what is going
%  on. 

%\end{itemize}
